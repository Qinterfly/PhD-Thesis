
\section{Научная новизна}

\begin{frame}{Научная новизна}
	\begin{enumerate}
		\item Разработан новый подход к коррекции расчетных динамических моделей путем добавления корректирующих конечных элементов, параметры которых определяются из решения задачи оптимизации по целевым модальным характеристикам.
		\item Обоснована методика формирования глобальной матрицы демпфирования конструкций по результатам испытаний.
		\item Развита методика синтеза достоверной расчетной модели ЛА из полноразмерных моделей составных частей, скорректированных по результатам модальных испытаний.
		\item Разработан способ освобождения расчетных моделей от наложенных связей.
		\item Исследованы сходимость алгоритма и чувствительность методики коррекции расчетных моделей к погрешностям эксперимента. 
	\end{enumerate}
\end{frame}

\section{Основные выводы и заключение}

\begin{frame}{Основные выводы и заключение}
	\begin{enumerate}
		\item С применением методов оптимизации разработана методика коррекции, основанная на дополнении исходной расчетной модели корректирующими конечными элементами. Проработана возможность коррекции как упругих характеристик, так и параметров демпфирования. Для формирования начального приближения матрицы демпфирования используется гипотеза Е.\,С.~Сорокина. Разработанный подход использован для решения практических задач коррекции  расчетных моделей.
		\item Методом статистического моделирования исследована сходимость алгоритма коррекции по отношению к погрешностям в результатах модальных испытаний. На модельных задачах показано, что результаты коррекции устойчивы во всем диапазоне погрешностей, вносимых в целевые значения частот собственных колебаний.
	\end{enumerate}
\end{frame}

\begin{frame}{Основные выводы и заключение}
	\begin{enumerate}
		\setcounter{enumi}{2}
		\item Развита методика синтеза глобальных конечно-элементных моделей конструкций на основе верифицированных по результатам модальных испытаний моделей составных частей. Обоснована программа проведения экспериментов, позволяющая получать наиболее полные характеристики составных частей конструкций, в том числе при наложении дополнительных связей. Разработано программное обеспечение, реализующие полный цикл операций для решения задачи синтеза.
		\item Создан комплекс программ, позволяющий проводить обработку и представление результатов модального анализа непосредственно в процессе испытаний. Используя разработанное программное обеспечение, продемонстрированы практические результаты обнаружения производственно-конструктивных дефектов в конструкциях ЛА по результатам модальных испытаний.
	\end{enumerate}
\end{frame}