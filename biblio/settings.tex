% --------------------------------------------------------- %
% --- Настройки библиографии посредством движка bibtex8 --- %
% --------------------------------------------------------- %

% --- Пакеты --- %

\usepackage{cite} % Ссылки на литературу

% --- Стили --- %

% Оформляем библиографию по ГОСТ 7.1 (ГОСТ Р 7.0.11-2011, 5.6.7)
\bibliographystyle{ugost2008} 

% Заменяем библиографию с квадратных скобок на точку
\makeatletter
\renewcommand{\@biblabel}[1]{#1.}   
\makeatother

% --- Цитирование --- %

% Разделение ; при перечислении ссылок (ГОСТ Р 7.0.5-2008)
\renewcommand\citepunct{;\penalty\citepunctpenalty\hskip.13emplus.1emminus.1em\relax} 

% Чтобы примеры цитирования, рассчитанные на biblatex, не вызывали ошибок при компиляции в bibtex
\newcommand*{\autocite}[1]{}  

% --- Создание команд для вывода списка литературы --- %

\newcommand*{\insertbibliofull}{\bibliography{biblio/external, biblio/author}}
\newcommand*{\insertbiblioauthor}{\bibliography{biblio/author}}
\newcommand*{\insertbiblioexternal}{\bibliography{biblio/external}}

% Счётчик использованных ссылок на литературу, обрабатывающий с учётом неоднократных ссылок
\newtotcounter{citenum}
\def\oldcite{}
\let\oldcite=\bibcite
\def\bibcite{\stepcounter{citenum}\oldcite}
