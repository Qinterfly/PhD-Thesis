% -------------------------------------- %
% --- Задание диссертационного стиля --- %
% -------------------------------------- %

% --- Путь к изображениям --- %

\graphicspath{{images/partReview/}{images/partModalAnalysis/}{images/partVerification/}{images/partAprobation/}{images/appendix/}}

% --- Интервалы --- %

% Для исправления интервала в таблицах ставится *
\OnehalfSpacing* % Полуторный интервал (ГОСТ Р 7.0.11-201)

% --- Макет страницы (ГОСТ 7.0.11-2011, 5.3.7) --- %

% Выставляем значения полей 
\geometry{a4paper, top = 2cm, bottom = 2cm, left = 2.5cm, right = 1cm, nofoot, nomarginpar} 
\setlength{\topskip}{0pt}     % Размер дополнительного верхнего поля
\setlength{\footskip}{12.3pt} % Снимет warning, согласно https://tex.stackexchange.com/a/334346

% --- Выравнивание и переносы --- %
%% http://tex.stackexchange.com/questions/241343/what-is-the-meaning-of-fussy-sloppy-emergencystretch-tolerance-hbadness
%% http://www.latex-community.org/forum/viewtopic.php?p=70342#p70342

\tolerance 1414
\hbadness 1414
\emergencystretch 1.5em % В случае проблем регулировать в первую очередь
\hfuzz 0.3pt
\vfuzz \hfuzz
%\raggedbottom
%\sloppy                % Избавляемся от переполнений
\clubpenalty=10000      % Запрещаем разрыв страницы после первой строки абзаца
\widowpenalty=10000     % Запрещаем разрыв страницы после последней строки абзаца
\brokenpenalty=4991     % Ограничение на разрыв страницы, если строка заканчивается переносом

% --- Блок управления параметрами для выравнивания заголовков в тексте --- %

\newlength{\otstuplen}
\setlength{\otstuplen}{\theotstup\parindent}

% Выравнивание заголовков в тексте (ГОСТ Р 7.0.11-2011, 5.3.5)
\ifnumequal{\value{headingalign}}{0}{
    \newcommand{\hdngalign}{\centering} % По центру
    \newcommand{\hdngaligni}{}
    \setlength{\otstuplen}{0pt}
}{
    \newcommand{\hdngalign}{} % По левому краю
    \newcommand{\hdngaligni}{\hspace{\otstuplen}}
} 

% --- Оглавление --- %

% Задание заполнителя между названием и нумерацией
\renewcommand{\cftchapterdotsep}{\cftdotsep} 

% Запрет переноса заголовков в оглавлении (ГОСТ Р 7.0.11-2011, 5.3.5)
\setrmarg{2.55em plus1fil}

% Текстовое выделение
\renewcommand{\cftchapterpagefont}{\normalfont}                  % Нежирные номера страниц у глав в оглавлении
\renewcommand{\cftchapterleader}{\cftdotfill{\cftchapterdotsep}} % Нежирные точки до номеров страниц 
																 %  у глав в оглавлении
%\renewcommand{\cftchapterfont}{}                                % Нежирные названия глав в оглавлении

% Определение разделителей
% Точка с пробелом в номере раздела
\ifnumgreater{\value{headingdelim}}{0}{
    \renewcommand\cftchapteraftersnum{.\space} 
}{}
% Точка с пробелом в номерах подразделов и их производных
\ifnumgreater{\value{headingdelim}}{1}{
    \renewcommand\cftsectionaftersnum{.\space}   
    \renewcommand\cftsubsectionaftersnum{.\space} 
    \renewcommand\cftsubsubsectionaftersnum{.\space}
    \setsecnumformat{\csname the#1\endcsname.\space}
}{}

% Включение приложения в оглавление
\renewcommand*{\cftappendixname}{\appendixname\space}

% --- Колонтитулы --- %

% Порядковый номер страницы печатают на середине верхнего поля страницы (ГОСТ Р 7.0.11-2011, 5.3.8)
\makeevenhead{plain}{}{--~\rmfamily\thepage~--}{}
\makeoddhead{plain}{}{--~\rmfamily\thepage~--}{}
\makeevenfoot{plain}{}{}{}
\makeoddfoot{plain}{}{}{}
\pagestyle{plain}

% Добавить Стр. над номерами страниц в оглавлении
\newif\ifendTOC

\newcommand*{\tocheader}{
\ifnumequal{\value{pgnum}}{1}{
	\ifendTOC\else\hbox to \linewidth%
      {\noindent{}~\hfill{Стр.}}\par%
      \ifnumless{\value{page}}{3}{}{%
        \vspace{0.5\onelineskip}
      }
      \afterpage{\tocheader}
    \fi
}{}
}

% --- Оформление заголовков глав, разделов, подразделов --- %

% Работа должна быть выполнена ... размером шрифта 12-14 пунктов (ГОСТ Р 7.0.11-2011, 5.3.8). То есть не должно быть надписей шрифтом более 14. Так и поставим.
% Эти установки будут давать одинаковый результат независимо от выбора базовым шрифтом 12 пт или 14 пт
\newcommand{\basegostsectionfont}{\fontsize{14pt}{16pt}\selectfont\bfseries}

\makechapterstyle{thesisgost}{
    \chapterstyle{default}
    \setlength{\beforechapskip}{0pt}
    \setlength{\midchapskip}{0pt}
    \setlength{\afterchapskip}{\theintvl\curtextsize}
    \renewcommand*{\chapnamefont}{\basegostsectionfont\large}
    \renewcommand*{\chapnumfont}{\basegostsectionfont\large}
    \renewcommand*{\chaptitlefont}{\basegostsectionfont\itshape\large}
    \renewcommand*{\chapterheadstart}{}
    \ifnumgreater{\value{headingdelim}}{0}{
        \renewcommand*{\afterchapternum}{.\space} % Добавляет точку с пробелом после номера главы
    }{
        \renewcommand*{\afterchapternum}{\quad}   % Добавляет \quad после номера главы
    }
    \renewcommand*{\printchapternum}{\hdngaligni\hdngalign\chapnumfont \thechapter}
    \renewcommand*{\printchaptername}{}
    \renewcommand*{\printchapternonum}{\hdngaligni\hdngalign}
}

\makeatletter
\makechapterstyle{thesisgostchapname}{
    \chapterstyle{thesisgost}
    \renewcommand*{\printchapternum}{\chapnumfont \thechapter}
    \renewcommand*{\printchaptername}{\hdngaligni\hdngalign\chapnamefont \@chapapp}
}
\makeatother

% Выбор стиля глав
\chapterstyle{thesisgost}

% Задание отступов всех разделов и их производных
\setsecheadstyle{\basegostsectionfont\hdngalign}
\setsecindent{\otstuplen}

\setsubsecheadstyle{\basegostsectionfont\hdngalign}
\setsubsecindent{\otstuplen}

\setsubsubsecheadstyle{\basegostsectionfont\hdngalign}
\setsubsubsecindent{\otstuplen}

% Центрирование заголовков с учетом номера
\sethangfrom{\noindent #1} 

% Написание слова "Глава" перед каждым номером раздела в оглавлении
\ifnumequal{\value{chapstyle}}{1}{
    \chapterstyle{thesisgostchapname}
    \renewcommand*{\cftchaptername}{\chaptername\space}
}{}

% --- Интервалы до и после заголовков (ГОСТ Р 7.0.11-2011, 5.3.5) --- %

% Отделение от текста до и после заголовков интервалами
\setbeforesecskip{\theintvl\curtextsize}
\setaftersecskip{\theintvl\curtextsize}
\setbeforesubsecskip{\theintvl\curtextsize}
\setaftersubsecskip{\theintvl\curtextsize}
\setbeforesubsubsecskip{\theintvl\curtextsize}
\setaftersubsubsecskip{\theintvl\curtextsize}

% --- Дополнительное задание размеров заголовков --- %

% Пропорциональные заголовки и базовый шрифт 14 пт
\ifnumequal{\value{headingsize}}{1}{
    \renewcommand{\basegostsectionfont}{\bfseries}
    \renewcommand*{\chapnamefont}{\large\bfseries}
    \renewcommand*{\chapnumfont}{\large\bfseries}
    \renewcommand*{\chaptitlefont}{\large\itshape\bfseries}
}{}

% --- Счётчики --- %

% Упрощённые настройки шаблона диссертации: нумерация формул, таблиц, рисунков
% * Убираем связанность номера формулы с номером главы/раздела
\ifnumequal{\value{contnumeq}}{1}{
    \counterwithout{equation}{chapter} 
}{}
\ifnumequal{\value{contnumfig}}{1}{
    \counterwithout{figure}{chapter}
}{}
\ifnumequal{\value{contnumtab}}{1}{
    \counterwithout{table}{chapter}
}{}

% --- Регистрация счётчиков в системе totcounter --- %

\AfterEndPreamble{
    \regtotcounter{totalcount@figure}
    \regtotcounter{totalcount@table}       % Если иным способом поставить в преамбуле то ошибка в числе таблиц
    \regtotcounter{TotPages}               % Если иным способом поставить в преамбуле то ошибка в числе страниц
    \newtotcounter{totalappendix}
    \newtotcounter{totalchapter}
}

% --- Определение новых сокращений --- %

% Формулы
\newcommand{\rarr}{$\rightarrow$}             % Стрелка вправо
\newcommand{\mat}[1]{\mathbf{#1}}             % Жирные матричные символы
\newcommand{\set}[1]{\mathbb{#1}}             % Множество
\newcommand{\trans}[1]{#1 ^ \mathsf{T}}       % Транспонирование
\newcommand{\rbrackets}[1]{\left( #1 \right)} % Круглые скобки
\newcommand{\sbrackets}[1]{\left[ #1 \right]} % Квадратные скобки
\newcommand{\abs}[1]{\left| #1 \right|}       % Модуль числа
\newcommand{\cbrackets}[1]{\{ #1 \}}		  % Фигурные скобки
\newcommand{\imag}{\operatorname{Im}}         % Мнимая часть числа
\newcommand{\real}{\operatorname{Re}}         % Действительная часть числа
% Частная производная n-го порядка по одному аргументу
\newcommand{\partDer}[3]{ 
\ifthenelse{ \equal{#3}{1} \OR \equal{#3}{} }
	{ \frac{\partial #1}{\partial #2} }
	{ \frac{\partial^{#3} #1}{\partial #2^{#3}} }
}

% Названия
\newcommand{\name}[1]{\texttt{#1}} 

% Рисунки
\newcommand{\figref}[1]{(рисунок~\ref{#1})} 

% Таблицы
\newcommand{\tabref}[1]{(таблица~\ref{#1})}

% --- Начертание математических выражений --- %

% Русская традиция начертания математических знаков
\renewcommand{\le}{\ensuremath{\leqslant}}
\renewcommand{\leq}{\ensuremath{\leqslant}}
\renewcommand{\ge}{\ensuremath{\geqslant}}
\renewcommand{\geq}{\ensuremath{\geqslant}}
\renewcommand{\emptyset}{\varnothing}

% Русская традиция начертания математических функций
\renewcommand{\tan}{\operatorname{tg}}
\renewcommand{\cot}{\operatorname{ctg}}
\renewcommand{\csc}{\operatorname{cosec}}

% Русская традиция начертания греческих букв (греческие буквы вертикальные, через пакет upgreek)
\renewcommand{\epsilon}{\ensuremath{\upvarepsilon}}
\renewcommand{\phi}{\ensuremath{\upvarphi}}
%\renewcommand{\kappa}{\ensuremath{\varkappa}}
\renewcommand{\alpha}{\upalpha}
\renewcommand{\beta}{\upbeta}
\renewcommand{\gamma}{\upgamma}
\renewcommand{\delta}{\updelta}
\renewcommand{\varepsilon}{\upvarepsilon}
\renewcommand{\zeta}{\upzeta}
\renewcommand{\eta}{\upeta}
\renewcommand{\theta}{\uptheta}
\renewcommand{\vartheta}{\upvartheta}
\renewcommand{\iota}{\upiota}
\renewcommand{\kappa}{\upkappa}
\renewcommand{\lambda}{\uplambda}
\renewcommand{\mu}{\upmu}
\renewcommand{\nu}{\upnu}
\renewcommand{\xi}{\upxi}
\renewcommand{\pi}{\uppi}
\renewcommand{\varpi}{\upvarpi}
\renewcommand{\rho}{\uprho}
%\renewcommand{\varrho}{\upvarrho}
\renewcommand{\sigma}{\upsigma}
%\renewcommand{\varsigma}{\upvarsigma}
\renewcommand{\tau}{\uptau}
\renewcommand{\upsilon}{\upupsilon}
\renewcommand{\varphi}{\upvarphi}
\renewcommand{\chi}{\upchi}
\renewcommand{\psi}{\uppsi}
\renewcommand{\omega}{\upomega}