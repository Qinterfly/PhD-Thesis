\chapter{Результаты модальных испытаний как исходные данные для верификации расчетных моделей конструкций}

\section{Методика определения модальных параметров}

\fixme{Здесь нужно использовать описание методики, изложенное в изобретении}

\section{Погрешности экспериментального модального анализа}

\fixme{Знание погрешностей необходимо для оценки <<глубины>> коррекции}
 
\section{Первичная обработка результатов испытаний}

\fixme{Эта обработка необходима в том числе и для устранения некоторых погрешностей}

\section{Диагностика дефектов конструкций по результатам испытаний}

\fixme{Дефектов нет в расчетных моделях, поэтому их нужно обнаруживать в реальной конструкции, а затем либо устранять, либо учитывать}

\subsection{Использование нелинейных искажений портретов колебаний}

\fixme{Привести таблицу сравнений чувствительности портретов и собственных частот к дефектам на примере панели}

\subsection{Обнаружение трещин}

\subsection{Контроль люфтов и зазоров}

\subsection{Выявление повышенного трения в подвижных соединениях}

\section{Обработка и представление результатов в процессе испытаний}

\fixme{Вот здесь и пригодилась твоя программа экспресс обработки результатов испытаний, так как между испытаниями и первым вылетом нет времени для составления полновесного отчета. Но главное не это! Обработка и представление результатов испытаний непосредственно в процессе испытаний позволит оперативно составить заключение о полноте экспериментальных данных, необходимых для коррекции расчетной модели объекта испытаний.}

\section{Выводы по главе \thechapter}