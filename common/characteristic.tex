
{\actuality} 

Решение проблемы безопасной и эффективной эксплуатации авиационной и космической техники начинается на этапе проектирования. Для этой цели разрабатываются различные расчетные модели летательных аппаратов (ЛА). Так, например, динамические расчетные модели используются для обеспечения аэроупругой устойчивости самолетов и управляемости космических аппаратов, определении реакции ЛА на динамическое воздействие. Расчетные модели, построенные по технической документации изделий, позволяют сделать первоначальную оценку динамических характеристик ЛА. Невозможность в полной мере учесть в расчетах особенности реальной конструкции приводит к необходимости экспериментального определения модальных параметров ЛА с последующей коррекцией расчетных моделей, поэтому разработка методов коррекции моделей по результатам модальных испытаний является актуальной задачей.

Целью модальных испытаний ЛА является определение характеристик собственных тонов (мод) колебаний конструкций (собственных частот и форм, обобщенных масс и демпфирования). Они проводятся на всех этапах создания ЛА. Испытаниям подвергаются динамически подобные модели ЛА, опытные и серийные образцы авиационной и космической техники. Этап экспериментальных исследований динамических характеристик предполагает испытания не только ЛА в целом, но и их составных частей. Скорректированные по результатам испытаний расчетные модели позволяют повысить эффективность работ по доводке изделий исходя из требований их безопасной и эффективной эксплуатации. 

В диссертации разработан метод коррекции конечно-элементных (КЭ) моделей динамических систем по результатам модальных испытаний. Целью коррекции является изменение спектра частот собственных колебаний. Метод заключается в изменении матрицы жесткости  посредством добавления корректирующей КЭ-модели, построенной на узлах исходной модели в соответствии с существующими взаимосвязями между линейными степенями свободы. В качестве параметров коррекции, подлежащих определению, выступают жесткости элементов корректирующей модели. Целевой функцией является взвешенная сумма квадратов разностей между целевыми (найденными экспериментально) и текущими обобщенными жесткостями. Метод не имеет ограничений по числу степеней свободы КЭ-моделей и не нарушает симметрию матриц. Он позволяет выделять составные части конструкции и проводить поэтапную коррекцию, в ходе которой в качестве целевых принимаются различные группы частот.

{\aim} данной работы является \ldots

Для~достижения поставленной цели необходимо было решить следующие {\tasks}:
\begin{enumerate}[beginpenalty = 10000] 
	\item Исследовать, разработать, вычислить и~т.\:д. и~т.\:п.
	\item Исследовать, разработать, вычислить и~т.\:д. и~т.\:п.
	\item Исследовать, разработать, вычислить и~т.\:д. и~т.\:п.
	\item Исследовать, разработать, вычислить и~т.\:д. и~т.\:п.
\end{enumerate}

{\novelty}
\begin{enumerate}[beginpenalty = 10000] 
	\item Предложен новый метод коррекции конечно-элементных моделей. 
	\item Создана программная платформа, позволяющая верифицировать математические модели конструкций по результатам модальных испытаний их составных частей.
	\item Предложена методика декомпозиции сигналов виброускорений для определения модальных параметров исследуемых конструкций.
	\item Разработаны вычислительные алгоритмы и их программные реализации для идентификации дефектов по типу трещин, люфтов и зазоров на основе нелинейных искажений портретов колебаний.
	\item Результаты применения методов верификации, идентификации и диагностирования дефектов к аэрокосмическим конструкциям.
\end{enumerate}

{\influence} 

{\methods} 

{\defpositions}
\begin{enumerate}[beginpenalty = 10000] 
	\item Первое положение
	\item Второе положение
	\item Третье положение
	\item Четвертое положение
\end{enumerate}

{\reliability} 



{\contribution} 

Все исследования, изложенные в диссертационной работе, проведены лично соискателем в процессе научной
деятельности. Из совместных публикаций в диссертацию включен лишь тот материал, который непосредственно принадлежит соискателю, заимствованный материал обозначен в работе ссылками.

{\pasport}

\fixme{2.5.14}

{\publications} 

Основные результаты по теме диссертации изложены в~XX~печатных изданиях, X~из которых изданы в журналах, рекомендованных ВАК, X~---~в тезисах докладов.
