
\section{Ответы на замечания}

\subsection{Ведущей организации}

\small

\begin{frame}
	\beginSkip
	\highlight{Ведущая организация}: Федеральное автономное учреждение <<Центральный аэрогидродинамический институт имени профессора Н.Е.\,Жуковского>>. \\
	Отзыв подписан: \\
	\highlight{Михаил Чеславович Зиченков}~---~заместитель генерального директора~--~начальник центра прочности ЛА, к.т.н., доцент. \\
	\highlight{Сергей Эмильевич Парышев}~---~начальник~отделения норм прочности, нагрузок и аэроупругости, к.т.н. \\
	\begin{comtblr}{14em}
		1. Из результатов коррекции видно, что точность достижения целевых значений частот собственных колебаний для ряда конструкций значительно превышает погрешность их экспериментально определения. Насколько оправдан с практической точки зрения?
		& 
		Удовлетворение экспериментальных значений в рамках доверительных границ погрешностей частот оправдано только для глобальных расчетных моделей. Применение такого подхода к коррекции ассемблируемых моделей составных частей приводит к погрешностям в частотах глобальной модели, которые существенно выходят за пределы исходного доверительного интервала. Таким образом, происходит множественное накопление погрешностей.
	\end{comtblr}
\end{frame}

\begin{frame}
	\vspace{0.3em}
	\begin{comtblr}{15em}
		2. Размер расчетной модели динамически-подобной модели (ДПМ) самолёта \mbox{Ту-204}, схематизация которой предполагает органичное моделирование балочными элементами, кажется избыточным для подобного рода конструкций. Является ли такое количество степеней свободы необходимым условием для успешного завершения процедуры коррекции?
		& 
		Нет, это не является необходимым условием. По результатам экспериментального модального анализа установлено, что балочная модель ДПМ недостаточно точно описывает динамическое поведение узлов сочленения агрегатов планера, имеющих люфт. Поэтому использована объемная конечно-элементная модель, моделирующая контактными элементами ограниченно подвижные соединения. Последнее является причиной значительного числа степеней свободы. \\
		3. Не даны рекомендации по выбору численных значений весовых коэффициентов и параметра регуляризации в задаче коррекции.
		&
		С замечанием согласен. На каждом шаге коррекции весовые коэффициенты определялись как величины, обратные невязкам корректируемых тонов колебаний. Параметр регуляризации полагался равным нулю для улучшения сходимости алгоритма коррекции в случае существенных исходных погрешностей. 
	\end{comtblr}
\end{frame}

\begin{frame}
	\vspace{1em}
	\begin{comtblr}{}
		4. Отсутствуют данные о минимальном размере дефекта каждого типа, который может быть идентифицирован по распределению параметра искажений портретов колебаний.
		&
		С замечанием согласен. В общем случае минимальный размер дефекта определяется габаритами исследуемого объекта. Исходя из имеющегося опыта применения методики установлено, что минимальный размер обнаруживаемых люфтов и зазоров измеряется микрометрами, а трещин~---~пятью миллиметрами. \\
		5. В таблицах, показывающих сходимость расчетных частот к целевым значениям, показаны только исходные частоты, а результаты итераций представлены уже только в виде отклонений от целевых значений в процентах. Желательно давать значения самых частот в процессе сходимости.
		& 
		С замечанием частично согласен. Значения частот в ходе коррекции не представлены, так как они могут быть пересчитаны на основе исходных частот и отклонений на каждом шаге коррекции. 
	\end{comtblr}
\end{frame}

\subsection{Организаций}

\begin{frame}
	\vspace{0.5em}
	\highlight{Бехер Сергей Алексеевич}, заведующий научно-исследовательской лабораторией физических методов контроля качества ФГБОУ ВО <<Сибирский государственный университет путей сообщения>>, д.т.н., доцент. \\
	\highlight{Красовский Валерий Викторович}, заместитель главного технолога филиала ПАО <<ОАК>>~---~<<НАЗ им. В.П.~Чкалова>>, к.т.н. \\
	\highlight{Социховский Аркадий Борисович}, технический директор производства военной авиационной техники ПАО <<ОАК>>~---~<<НАЗ им. В.П.~Чкалова>>. \\
	\begin{comtblr}{12em}
		1. Каким параметром оцениваются искажения портретов вынужденных колебаний конструкций при появлении дефектов?
		& 
		Параметр искажения в каждой точке измерения равен нормированному максимуму остатка сигнала после вычитания из него первой гармоники. Более подробное описание методики приведено в диссертации в подразделе 3.2.1. \\
		2. Как чувствительность представленного в работе алгоритма обнаружения дефектов соотносится с известными? 
		&
		Проведенные исследования показали, что известный метод вибродиагностики по изменению частот собственных колебаний обладает чувствительностью в несколько процентов, в то время как параметр искажений портретов вынужденных колебаний увеличивается в тысячу раз.
	\end{comtblr}
\end{frame}

\begin{frame}
	\beginSkip
	\begin{comtblr}{12em}
		3. Каковы минимальные размеры выявляемых дефектов?
		&
		Минимальные размеры обнаруживаемых дефектов озвучены при ответе на отзыв ведущей организации. \\
	\end{comtblr}
\end{frame}

\begin{frame}
	\beginSkip
	\highlight{Каракешишев Владимир Александрович}, заместитель главного конструктора филиала ПАО <<Авиационный комплекс имени С.В.~Илюшина>>~---~<<Экспериментальный машиностроительный завод имени В.М.~Мясищева>>.
	\begin{comtblr}{}
		В автореферате не приведены данные о результатах визуального и инструментального контроля дефектов, обнаруженных по изменению характера портретов колебаний.
		& 
		С замечанием согласен. Данные о соответствии места обнаружения дефекта максимуму параметра искажений портретов колебаний приведены в диссертации на примере рефлектора антенны космического аппарата (стр.~78).
	\end{comtblr}
\end{frame}

\begin{frame}
	\beginSkip
	\highlight{Шкода Александр Васильевич}, главный конструктор по прочности~---~начальник научно-исследовательского отделения филиала ПАО <<Объединенная авиастроительная корпорация>>~---~<<Опытно-конструкторское бюро П.О.~Сухого>>. \\
	\highlight{Пара Александр Владимирович}, заместитель начальника отдела нагрузок и аэроупругости ПАО <<Объединенная авиастроительная корпорация>>~---~<<Опытно-конструкторское бюро П.О.~Сухого>>. \\
	\begin{comtblr}{15em}
		В автореферате представлены результаты коррекции расчетной модели динамически-подобной модели самолёта Ту-204. Если ранее коррекция этой модели уже производилась, то как ее результаты соотносятся с полученными диссертантом?
		&
		Действительно, коррекция расчетной модели уже производилось посредством введения фиктивного груза массой $ 7 $ кг, что позволило получить частное приближение к результатам модального анализа для испытываемой конфигурации. Однако результаты такой коррекции нельзя назвать объективными в виду отсутствия физической взаимосвязи между внесенным изменением и реальным конструктивным составом модели. 
	\end{comtblr}
\end{frame}

\begin{frame}
	\beginSkip
	\highlight{Кручинин Михаил Михайлович}, технический руководитель конструкторского бюро внешних нагрузок, аэроупругости и земного резонанса АО <<Национальный центр вертолётостроения имени М.Л.~Миля и Н.И.~Камова>>. \\
	\begin{comtblr}{15em}
		1. Исследования проведены только для одного типа летательных аппаратов~---~самолётов, хотелось бы оценить возможности методики при модальном анализе вертолетной техники.
		&
		С замечанием согласен. Этот пробел планируется восполнить в дальнейшем. \\
		2. Изменение конечно-элементной модели с помощью корректирующих элементов неизбежно приведет к изменению местных форм и частот колебаний в различных зонах при совпадении с испытаниями форм и частот всего летательного аппарата.
		&
		С замечанием согласен~---~это закономерное следствие изменения упругих характеристик. Однако с практической точки зрения аэроупругая устойчивость определяется частотами и формами собственных колебаний всего летательного аппарата.
	\end{comtblr}
\end{frame}

\begin{frame}
	\beginSkip
	\highlight{Паймушин Виталий Николаевич}, профессор кафедры прочности конструкций ФГБОУ ВО <<Казанский национальный исследовательский технический университет им. А.Н. Туполева~--~КАИ>>, д.ф.-м.н, академик Академии наук Республики Татарстан. \\
	\begin{comtblr}{}
		В работе отсутствует результаты исследования аэроупругой устойчивости с использованием скорректированных моделей. Известно, что некорректные задачи имеют множество решений, в том числе и таких, которые обеспечивают достижения целей коррекции, являясь при этом физически несогласованными.
		&
		С замечанием согласен. Косвенно о сходимости к предельному решению свидетельствуют результаты исследования устойчивости алгоритма коррекции на модельных задачах. Исследование аэрупругой устойчивости по скорректированным моделям планируется в ближайшей перспективе.
	\end{comtblr}
\end{frame}

\begin{frame}
	\beginSkip
	\highlight{Пономарев Сергей Васильевич}, заведующий отделом механики деформируемого твердого тела Обособленного структурного подразделения <<Научно-исследовательский институт прикладной математики и механики Томского государственного университета>>, д.ф.-м.н, с.н.с. \\
	\begin{comtblr}{15em}
		Из рисунка 1 следует, что корректирующие элементы являются независимыми, в том числе и для агрегатов планера, расположенных зеркально. Как это соотносится с тем, что реальные летательные аппараты, по меньшей мере, обладают симметрией в расположении отдельных агрегатов?
		&
		При введении корректирующих элементов учитываются геометрические особенности конструкции, такие как наличие конструктивно-идентичных элементов и плоскостей симметрии. В этом случае число независимых параметров коррекции сокращается посредством введения дополнительных взаимосвязей.
	\end{comtblr}
\end{frame}

\begin{frame}
	\beginSkip
	\highlight{Фролов Антон Сергеевич}, заместитель начальника проектно-конструкторского центра <<Прочность>> ПАО <<Туполев>>. \\
	\highlight{Гонин Владимир Михайлович}, начальник бригады ПАО <<Туполев>>. \\
	\begin{comtblr}{15em}
		Результаты работы метода коррекции расчетных моделей показаны с учетом относительно ограниченного набора собственных тонов, в то время как практические расчетные задачи аэроупругости могут требовать учета существенно большего базиса собственных тонов. При этом должна обеспечиваться удовлетворительная сходимость как по значениям частот, так и по формам собственных колебаний.
		&
		С замечанием согласен. Набор корректируемых тонов собственных колебаний, ответственных за флаттер, оценивается на этапе предварительного расчета. Учет расхождения расчетных и экспериментальных форм собственных колебаний планируется реализовать в дальнейшем посредством введения соответствующего штрафного слагаемого в целевую функцию.
	\end{comtblr}
\end{frame}

\begin{frame}
	\beginSkip
	\highlight{Петрова Татьяна Владимировна}, заведующий кафедрой № 24 <<Авиационной техники и диагностики>> ФГБОУ ВО <<Санкт-Петербургский государственный университет гражданской авиации имени А.А.~Новикова>>, к.т.н., доцент.
	\begin{comtblr}{15em}
		В качестве нулевого приближения используется гипотеза Е.C.\,Сорокина, однако возможность её использования не раскрыта, не указаны неточности, к которым может привести использование этой гипотезы.
		&
		С замечанием согласен. Однако гипотеза Е.C.~Сорокина используется только для формирования начального приближения матриц демпфирования, которые впоследствии уточняются введением корректирующих элементов. В этом случае допустимо использование любой гипотезы, которая обеспечивает ортогональность собственных векторов в метрике матрицы демпфирования. 
	\end{comtblr}
\end{frame}

\begin{frame}
	\beginSkip
	\highlight{Халиманович Владимир Иванович}, директор отраслевого центра крупногабаритных трансформируемых механических систем~---~заместитель главного конструктора по механическим системам АО <<Информационные спутниковые системы имени академика М.Ф.~Решетнева>>, к.ф.-м.н, профессор, член-корреспондент Российской инженерной академии.
	\begin{comtblr}{18em}
		В качестве недостатка работы следует отметить то, что автором не контролируется изменение напряжённо-деформируемого состояния расчетной модели в ходе коррекции. Может ли произойти так, что изменения упругих расчетных характеристик окажутся несовместимыми с требованиями статической прочности, предъявляемыми к анализируемой конструкции?
		&
		Методика коррекции оперирует динамическими моделями, которые упрощаются для проведения параметрических исследований аэроупругости, поэтому оценка статической прочности по ним не проводится. 
	\end{comtblr}
\end{frame}

\begin{frame}
	\beginSkip
	\highlight{Яковлев Алексей Борисович}, заведующий кафедрой <<Авиа- и ракетостроение>> ФГАОУ ВО <<Омский государственный технический университет>>, к.т.н., доцент. \\
	\highlight{Жариков Константин Игоревич}, доцент кафедры <<Авиа- и ракетостроение>> ФГАОУ ВО <<Омский государственный технический университет>>, к.т.н. \\
	\begin{comtblr}{}
		1. В автореферате диссертации в п.\,3 научной новизны заявлено обоснование методики формирования глобальной матрицы демпфирования конструкции по результатам испытаний её составных частей, но далее по тексту автореферата обоснования методики не представлено.
		&
		Методика состоит в двухэтапном построении и ассемблировании матриц демпфирования моделей составных частей, используя результаты нескольких экспериментов при различных условиях закрепления. \\
		2. Из текста неясно в чем заключается <<методика контроля зазоров в технических изделиях по искажениям портретов вынужденных колебаний>>, указанная на стр.~14 автореферата.
		&
		Описание методики приведено в тексте диссертации в подразделе 3.2.1. 
	\end{comtblr}
\end{frame}

\begin{frame}
	\beginSkip
	\highlight{Шеремет Михаил Александрович}, заведующий кафедрой теоретической механики ФГАОУ ВО <<Национальный исследовательский Томский государственный университет>>, д.ф.-м.н., профессор.
	\begin{comtblr}{}
		1. В автореферате указано, что для решения задачи минимизации целевой функции применяется метод сопряженных градиентов. Существует большое количество методик расчета коэффициентов сопряжения. Следовало указать используемый подход.
		&
		С замечанием согласен. Используется метод Флетчера-Ривса совместно с процедурой обновления Пауэлла. \\
		2. В четвертой главе при апробации методики коррекции на динамически-модели самолёта Ту-204 соискатель средствами Ansys создал конечно-элементную модель, имеющую 752000 степеней свободы. Из автореферата неясно, что послужило причиной выбора такого числа степеней свободы.
		&
		Обоснование степени дискретизации расчетной модели дано при ответе на отзыв ведущей организации. Также оно содержится в тексте диссертации в подразделе 4.1 (стр.~99). 
	\end{comtblr}
\end{frame}

\begin{frame}
	\beginSkip
	\highlight{Нагорнов Андрей Юрьевич}, начальник отдела аэроупругости отделения прочности АО <<Уральский завод гражданской авиации>>, к.т.н.
	\begin{comtblr}{17em}
		В тексте автореферата приведены изменения жесткостей по поверхности модели космического аппарата (рисунок 5) без указания порядка исходных величин. Соискателю стоило пояснить насколько существенно изменилась модель для того, чтобы удовлетворить данным эксперимента.
		&
		С замечанием согласен. Исходная максимальная узловая жесткость панелей солнечных батарей составляла $ 10 ^ {13} \ \nicefrac{\text{Н}}{\text{м}} $. В результате коррекции она увеличилась на $ 10 ^ 4 \ \nicefrac{\text{Н}}{\text{м}} $, что свидетельствует об относительной малости вносимых в модель изменений.
	\end{comtblr}
\end{frame}

\begin{frame}
	\beginSkip
	\highlight{Бужинский Валерий Алексеевич}, начальник отдела АО <<Центральный научно-исследовательский институт машиностроения>>, д.ф.-м.н.
	\begin{comtblr}{19em}
		1. Предложенная методика коррекции конечно-элементных моделей основывается на согласовании расчетных и полученных при модальных испытаниях собственных частотах. Для решения такого типа задач, относящихся к некорректным математическим задачам, требуется привлечение дополнительных сведений, которые четко не оговорены, в частности, не предъявлены требования к точности изначальной конечно-элементной модели, требуемой для эффективности применения методики.
		&
		С замечанием согласен. Предполагается, что расчетная модель обладает точным распределением инерционных характеристик, качественно описывает корректируемые тона колебаний и содержит все конструктивные элементы, представленные в исследуемой конструкции. \\
	\end{comtblr}
\end{frame}

\begin{frame}
	\beginSkip
	\begin{comtblr}{}
		2. Матрица масс конечно-элементной модели конструкции принимается точной и не корректируется, что ограничивает применение методики для легких крупногабаритных космических конструкций при испытаниях в лабораторных условиях из-за влияния воздушной среды, приводящей к присоединенной массе, а также к увеличению демпфирования. 
		&
		С замечанием согласен. Для такого рода конструкций необходимо проведение дополнительных экспериментальных исследований с целью оценки влияния воздушной среды на инерционные и диссипативные характеристики.
	\end{comtblr}
\end{frame}

\subsection{Оппонентов}

\begin{frame}
	\beginSkip
	\highlight{Официальный оппонент}: профессор кафедры аэрокосмических систем ФГБОУ~ВО <<Московский государственный технический университет имени Н.Э.~Баумана (национальный исследовательский университет)>>, д.т.н, профессор \\
	\highlight{Щеглов Георгий Александрович} \\
		\begin{comtblr}{18em}
		1. В разработанной автором методики коррекция модели производится только по собственным частотам колебаний, полученным в ходе модальных испытаний. В тексте диссертации нет данных, насколько хорошо удается обеспечить коррекцию собственных форм предложенной методики.
		&
		С замечанием согласен. Скорректированные расчетные и экспериментальные формы собственных колебаний сопоставлялись визуально. Получено удовлетворительное согласование. \\
		2. Отсутствуют данные о влиянии начального приближения параметров корректирующих элементов на результаты коррекции.
		&
		С замечанием согласен. По результатам исследований установлено, что наилучшие результаты коррекции достигаются при нулевом начальном приближении варьируемых параметров. 
	\end{comtblr}
\end{frame}

\begin{frame}
	\vspace{0.3em}
	\begin{comtblr}{}
		3. Предложенный автором способ освобождения от наложенных связей применим только для малых угловых скоростей вращений свободной конструкции, поскольку в представленных уравнениях системы (2.48) фигурируют только угловые ускорения, а влияние центробежных и кориолисовых сил, зависящих от угловой скорости вращения динамической системы не учитывается. Следует отметить, что обозначение углов греческой буквой <<омега>> не очень удачно, поскольку часто этой буквой обозначается угловая скорость. Автором в разделе 2.3.2 рассмотрен только пример поступательно движущийся системы. Было бы желательно также привести пример вращения освобожденной конструкции.
		&
		С замечанием частично согласен. В разделе 2.3.2 диссертационной работы содержится пример освобождения балочной модели самолёта (стр.~49), обладающей как линейными, так и угловым степенями свободы. \\
	\end{comtblr}
\end{frame}

\begin{frame}
	\vspace{0.5em}
	\begin{comtblr}{12em}
		4. Не обоснован выбор метода сопряженных градиентов в качестве решателя задачи оптимизации.
		&
		Метод сопряженных градиентов позволяет решать задачи многомерной безусловной минимизации без построения матрицы вторых частных производных, что существенно сокращает временные затраты и требования к памяти при решении задач коррекции полноразмерных конечно-элементных моделей, характеризующихся миллионами неизвестных. 
	\end{comtblr}
\end{frame}

\begin{frame}
	\beginSkip
	\highlight{Официальный оппонент}: профессор кафедры автоматических систем энергетических установок ФГАОУ ВО <<Самарский национальный исследовательский университет имени академика С.П. Королева>>, д.т.н., доцент \\
	\highlight{Иголкин Александр Алексеевич} \\
	\begin{comtblr}{}
		1. Недостаточно раскрыты достоинства разработанных методики коррекции конечно-элементных моделей ЛА, способа определения частот и форм собственных колебаний и методики испытаний составных частей ЛА.
		&
		С замечанием согласен. \\ 
		2. Не приведены примеры, свидетельствующие в пользу корректности методики формирования глобальной матрицы демпфирования для полноразмерных моделей 
		&
		С замечанием согласен. Этот пробел планируется восполнить в дальнейшей работе. \\
	\end{comtblr}
\end{frame}

\begin{frame}
	\beginSkip
	\begin{comtblr}{12em}
		3. В подразделе 2.3.2 описывается тестирование разработки на системе масс на пружинках, но никаких выводов в результате не делается
		&
		С замечанием не согласен. На странице 48 текста диссертации указано, что характеристики тонов собственных колебаний, определенные для освобожденных систем (2.66)~---~(2.68), совпадают с частотами и формами собственных колебаний свободной системы. Сопоставление численных результатов в данном случае не проводилось по причине относительной простоты рассматриваемой системы. \\
		4. В подразделе 3.2.2 описана методика диагностирования зазоров и люфтов, однако, из текста диссертации не понятно как эта методика связана с защищаемыми пунктами научной новизны и не приведены практические рекомендации
		& 
		При наличии дефектов в летательном аппарате его динамические характеристики могут существенно отличаться от расчетных. Эти дефекты должны либо оперативно устраняться, либо учитываться в расчетной модели. Без этого уточнение упругих характеристик расчетной модели является некорректным по причине неполного соответствия исследуемому объекту. Поэтому на практике сначала проводится обнаружение дефектов, а только затем коррекция.
	\end{comtblr}
\end{frame}

\begin{frame}
	\vspace{0.5em}
	\begin{comtblr}{12em}
		5. На странице 125 в таблице 4.6 вызывает вопрос полученная погрешность $ 0 $ \%.
		&
		Оценка доверительных интервалов экспериментальных погрешностей частот существенно зависимости от исследуемого тона колебаний. В рамках диссертации с целью определения эффективности созданных вычислительных алгоритмов требовалось абсолютное достижение целей коррекции. В дальнейшем планируется учет погрешностей определения целевых значений в качестве весовых коэффициентов в целевой функции. \\
		6. На некоторых рисунках отсутствуют размерности, а подписи сделаны на английском языке.
		&
		С замечанием согласен. \\
		7. На некоторых рисунках с результатами численного моделирования не указан масштаб <<цветового распределения>> полученных величин.
		&
		С замечанием согласен, хотя в большей степени оно относится к распределениям параметра искажений портретов колебаний, служащих для определения месторасположения возможных дефектов, которые затем оцениваются методами инструментального контроля.
	\end{comtblr}
\end{frame}

\normalsize