
\newcommand{\listPublications}{\pdfbookmark[0]{Список основных научных публикаций по теме диссертации}{listPublications}\textbf{Список основных научных публикаций по теме диссертации}}

{\listPublications}

\ul{Статьи в рецензируемых научных изданиях, входящих в список ВАК, в том числе, входящих в международные реферативные базы данных Web of Science и Scopus:}

\begin{enumerate}
	\item Метод коррекции конечно-элементных моделей динамических систем~/ Д.\,А.~Красноруцкий, П.\,А.~Лакиза, В.\,А.~Бернс, Е.\,П.~Жуков // Вестник Пермского национального исследовательского политехнического университета. Механика.~--- 2021.~--- №~3.~--- С.~84--95. 
	\item Контроль зазоров в конструкциях технических изделий в процессе вибрационных испытаний~/ Н.\,А.~Тестоедов, В.\,А.~Бернс, Е.\,П.~Жуков,
Е.\,А.~Лысенко, П.\,А.~Лакиза~// Обработка металлов (технология, оборудование, инструменты).~--- 2021.~--- Т.~23, №~2.~--- С.~40--53.
	\item Метод освобождения динамической расчетной модели летательного аппарата~/ Д.\,А.~Красноруцкий, В.\,А.~Бернс, П.\,А.~Лакиза, В.\,Е.~Левин~// Научный журнал <<Известия Самарского научного центра РАН>>.~--- 2019.~--- Т.~21, №~1.~--- С.~31--38.
	\item Исследования достоверности диагностирования трещин по искажениям портретов вынужденных колебаний / В.\,А.~Бернс, Е.\,П.~Жуков, П.\,А.~Лакиза, E.\,А.~Лысенко // Обработка металлов (технология, оборудование, инструменты).~--- 2019.~--- Т.~21, №~2.~--- С.~26--39.
\end{enumerate}

\ul{Патент на изобретение:}

\begin{enumerate}
	\item Пат.~2728329 Российская Федерация, МПК G01M7/00. Способ определения собственных частот и форм колебаний свободной конструкции по результатам испытаний этой конструкции с наложенными связями~/ Бернс~В.\,А., Жуков~Е.\,П., Красноруцкий~Д.\,А., Лакиза~П.\,А.~--- № 2019119278; заявл.~19.06.19; опубл.~29.06.20, Бюл.~№ 22.~--- 15~с.
\end{enumerate}

\ul{Свидетельства о государственной регистрации программ для ЭВМ:}

\begin{enumerate}
	\item Свидетельство 2023610282. Операционный модальный анализ летательных аппаратов <<FlightLab>>: программа для ЭВМ~/ П.\,А.~Лакиза, Д.\,А.~Красноруцкий, В.\,А.~Бернс (RU); правообладатель <<Новосибирский государственный технический университет>>; заявл.~10.01.2023; опубл.~10.01.2023.
	\item Свидетельство 2021663099. Контроль дефектов в процессе вибрационных испытаний <<DistortionFinder>>: программа для ЭВМ / П.\,А. Лакиза, В.\,А.~Бернс, Е.\,П.~Жуков (RU); правообладатель <<Новосибирский государственный технический университет>>; заявл.~02.08.2021; опубл.~11.08.2021.
 	\item Свидетельство 2021662816. Представление результатов модальных испытаний <<ResponseAnalyzer>>: программа для ЭВМ / П.\,А.~Лакиза, В.\,А.~Бернс, Е.\,П.~Жуков (RU); правообладатель <<Новосибирский государственный технический университет>>; заявл.~02.08.2021; опубл.~05.08.2021.
	\item Свидетельство 2021662965. Расчет обобщенных характеристик тонов собственных колебаний по результатам модальных испытаний <<GenCalc>>: программа для ЭВМ / П.\,А.~Лакиза, В.\,А.~Бернс, Е.\,П.~Жуков (RU); правообладатель <<Новосибирский государственный технический университет>>; заявл.~02.08.2021; опубл.~10.08.2021.
\end{enumerate}

\ul{Статьи в прочих изданиях:}

\begin{enumerate}
	\item Использование портретов колебаний в процессе контроля технического состояния летательных аппаратов~/ В.\,А.~Бернс, Е.\,А.~Лысенко, Е.\,П.~Жуков, П.\,А.~Лакиза, Д.\,О.~Душухин~// Общероссийский научно-технический журнал <<Полёт>>.~--- 2022.~--- №~2.~--- C.~64--71.
	\item Результаты модальных испытаний как исходные данные для коррекции расчетных моделей летательных аппаратов~/ В.\,А.~Бернс, Е.\,П.~Жуков, П.\,А.~Лакиза, Маленкова В. В., Д.\,О.~Душухин~// Общероссийский научно-технический журнал <<Полёт>>.~--- 2022.~--- №~2.~--- C.~49--56.
	\item Разработка расчетно-экспериментального метода модального анализа крупногабаритных трансформируемых космических конструкций~/ В.\,А.~Бернс, В.\,Е.~Левин, Д.\,А.~Красноруцкий, Д.\,А.~Маринин, Е.\,П.~Жуков, В.\,В.~Маленкова, П.\,А.~Лакиза~// Научный журнал <<Космические аппараты и технологии>>.~--- 2018.~--- C.~125--133.
\end{enumerate}