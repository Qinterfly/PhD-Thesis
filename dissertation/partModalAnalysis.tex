\chapter{Результаты модальных испытаний как исходные данные для верификации расчетных моделей конструкций}

\section{Методика определения модальных параметров по результатам экспериментального модального анализа}

С целью обеспечения возможности эффективного расчета обобщенных характеристик по результатам модальных испытаний, была разработана программная реализация \name{GenCalc}, позволяющая посредством графического интерфейса \figref{fig:gencalc-interface} гибко менять параметры расчета и исследовать зависимости получаемых характеристик по каждому из способов одновременно. Более того, для оценки качества выделения тона колебаний в программе заложена возможность построения частотного годографа \figref{fig:gencalc-godograph} и параметра монофазности \figref{fig:gencalc-monophase-parameter} колебательной системы.

\begin{figure}[!htb]
	\centerfloat
	\includegraphics[width = 0.85\linewidth]{gencalc-interface}
	\caption{Графический интерфейс программы} \label{fig:gencalc-interface}
\end{figure}

Необходимо отметить, что разработанная программная реализация обеспечивает прямое взаимодействие с результатами модальных испытаний, которые были получены с использованием комплекса \name{Simcenter Testlab}.

В рамках предлагаемого подхода, для расчетного диапазона необходимо задать минимальный и максимальный уровень амплитуды, число уровней, а также число точек для интерполяции сигнала отклика на каждом уровне. Кроме того, необходимо выбрать частоту амплитудного и фазового резонанса.

\begin{figure}[!htb]
	\centerfloat
	\includegraphics[width = 0.6\linewidth]{gencalc-monophase-parameter}
	\caption{Параметр монофазности по двум каналам возбуждения при колебаниях изделия по тону АГИКр2} \label{fig:gencalc-monophase-parameter}
\end{figure}

\begin{figure}[!htb]
	\centerfloat
	\includegraphics[width = 0.6\linewidth]{gencalc-godograph}
	\caption{Частотный годограф в точке отклика изделия при колебаниях по тону АГИКр2} \label{fig:gencalc-godograph}
\end{figure}

Программный функционал позволяет определять логарифмической декремент колебаний системы по каждому тону, используя четыре подхода:
\begin{enumerate}[topsep = 0pt, noitemsep]
	\item По ширине резонансного пика мнимой составляющей сигнала отклика.
	\item По ширине резонансного пика амплитуды колебаний.
	\item По наклону реальной составляющей сигнала отклика.
	\item Посредством точного решения \eqref{eq:generalModalSolution} системы нелинейных уравнений третьего порядка \eqref{eq:generalModalSystem} относительно обобщенных характеристик:
\end{enumerate}

\begin{equation}
	\begin{aligned}
		a ^ 3 \sum_{k = 1} ^ M y_k ^ 4 \omega_k ^ 8 - 3 a ^ 2 c \sum_{k = 1} ^ M y_k ^ 4 \omega_k ^ 6 + a \sum_{k = 1} ^ M \left[ y_k ^ 4 \omega_k ^ 4 \left(3 c ^ 2 + h ^ 2 \right) - Q_k ^ 2 y_k ^ 2 \omega_k ^ 4 \right] + \\
		+ \ c \sum_{k = 1} ^ M Q_k ^ 2 y_k ^ 2 \omega_k ^ 2 - c ^ 3 \sum_{k = 1} ^ M y_k ^ 4 \omega_k ^ 2 - c h ^ 2 \sum_{k = 1} ^ M y_k ^ 4 \omega_k ^ 2 = 0, \\
		a ^ 3 \sum_{k = 1} ^ M y_k ^ 4 \omega_k ^ 8 - 3 a ^ 2 c \sum_{k = 1} ^ M y_k ^ 4 \omega_k ^ 6 + a \sum_{k = 1} ^ M \left[ y_k ^ 4 \omega_k ^ 4 \left(3 c ^ 2 + h ^ 2 \right) - Q_k ^ 2 y_k ^ 2 \omega_k ^ 4 \right] + \\
		+ \ c \sum_{k = 1} ^ M Q_k ^ 2 y_k ^ 2 - c ^ 3 \sum_{k = 1} ^ M y_k ^ 4 - c h ^ 2 \sum_{k = 1} ^ M y_k ^ 4 = 0, \\
		a ^ 2 h \sum_{k = 1} ^ M y_k ^ 4 \omega_k ^ 4 - 2 a c h \sum_{k = 1} ^ M y_k ^ 4 \omega_k ^ 2 - h \sum_{k = 1} ^ M Q_k ^ 2 y_k ^ 2 + c ^ 2 h \sum_{k = 1} ^ M y_k ^ 4 + h ^ 3 \sum_{k = 1} ^ M y_k ^ 4 = 0. 
	\end{aligned}
	\label{eq:generalModalSystem}
\end{equation}

Эту систему уравнений удается решить точно:
\begin{equation}
	\begin{gathered}
		a = b ^ {\sfrac{1}{2}}, \\
		c = -\frac{b d_1 + d_3}{d_2 b ^ {\sfrac{1}{2}}}, \\
		h = \left[ \left( \sum_{k = 1} ^ M Q_k ^ 2 y_k ^ 2 - c ^ 2 \sum_{k = 1} ^ M y_k ^ 4 - a ^ 2 \sum_{k = 1} ^ M y_k ^ 4 \omega_k ^ 4 + 2 a c \sum_{k = 1} ^ M y_k ^ 4 \omega_k ^ 2 \right) / \sum_{k = 1} ^ M y_k ^ 4 \right] ^ {\sfrac{1}{2}}, \\ 
		f_1 = \sum_{i, j = 1} ^ M y_i ^ 4 y_j ^ 4 \omega_j ^ 4 \left( \omega_j ^ 4 - \omega_i ^ 4 \right), \ d_1 = \sum_{i, j = 1} ^ M y_i ^ 4 y_j ^ 4 \omega_i ^ 4 \left( w_i ^ 4 - \omega_j ^ 4 \right), \\
		f_2 = \sum_{i, j = 1} ^ M y_i ^ 4 y_j ^ 4 \omega_j ^ 4 \left( \omega_j ^ 2 - \omega_i ^ 2 \right), \ d_2 = 2 \sum_{i, j = 1} ^ M y_i ^ 4 y_j ^ 4 \omega_i ^ 2 \left( w_j ^ 2 - \omega_i ^ 2 \right), \\
		d_3 = \sum_{i, j = 1} ^ M y_i ^ 2 y_j ^ 2 \omega_j ^ 2 \left( y_i ^ 2 Q_j ^ 2 - y_j ^ 2 Q_i ^ 2 \right), \ f_3 = \sum_{i, j = 1} ^ M y_i ^ 2 y_j ^ 2 \omega_i ^ 4 \left( y_i ^ 2 Q_j ^ 2 - y_j ^ 2 Q_i ^ 2 \right), \\
		b = \frac{f_2 d_3 - f_3 d_2}{f_1 d_2 - f_2 d_1}.
	\end{gathered}
	\label{eq:generalModalSolution}
\end{equation}

В случае последнего подхода будем дополнительно определять обобщенное демпфирование, обобщенные жесткость и массу, и представлять их в виде графической зависимости от уровня амплитуд.

Для выбранной точки отклика конструкции, которая, как правило, располагается вблизи точки возбуждения, выберем некоторый диапазон значений в окрестности резонансной частоты для которого будет проводиться расчет по каждому из подходов. Заметим, что логарифмический декремент изменяется по мере изменения амплитуды воздействия, поэтому, с целью определения характера этого изменения и его предельных значений, предлагается
строить графические зависимости определяемых характеристик от амплитуды отклика.

Рассмотрим каждый из расчетных способов в отдельности. Для отыскания логарифмического декремента колебаний $ \delta_I $ по ширине резонансного пика мнимой составляющей (№1) воспользуемся следующей формулой:

\begin{equation}
	\delta_I = \pi \Delta \overline{f} \sqrt{\frac{\imag \overline{a}}{1 - \imag \overline{a}}}, \label{eq:decrementWidthImaginaryPeak}
\end{equation}
где $ \imag \overline{a} = \frac{\imag a}{\imag a_{\max}} $~---~относительное значение мнимой составляющей сигнала, $ \Delta \overline{f} = \frac{\Delta f}{f_{\imag}}$, $ \Delta f = f_2 - f_1 $~---~разность характерных частот амплитудной частотной характеристики (АЧХ). Значения $ f_1 $ и $ f_2 $ равны абсциссам где ординаты АЧХ достигают (в долях от максимальной амплитуды) характерное значение уровня.

Логарифмический декремент колебаний $ \delta_{II} $ по ширине резонансного пика амплитуды колебаний (№2) рассчитывается следующим образом:
\begin{equation}
	\delta_{II} = \pi \Delta \overline{f} \frac{\overline{A}}{\sqrt{1 - \overline{A} ^ 2}}, \label{eq:decrementWidthAmplitudePeak}
\end{equation}
где $ \overline{A} = \frac{A}{A_{\max}}$~--~относительная характерная амплитуда уровня, $ A = \sqrt{(\real a) ^ 2 + (\imag a) ^ 2} $.

Расчет  $ \delta_{III} $ по наклону реальной составляющей (№3) производится по следующей формуле:
\begin{equation}
	\delta_{III} = \pi \Delta \overline{f},
	\label{eq:decrementAngleReal}
\end{equation}
где значения $ f_1 $ и $ f_2 $ равны абсциссам тех точек, где ординаты АЧХ достигают экстремальных значений. Для определения этих значений предлагается использовать первую производную интерполированной действительной составляющей.

Для определения обобщенных характеристик конструкции из решения системы нелинейных уравнений \eqref{eq:generalModalSystem} необходимо произвести расчет обобщенной силы $ Q $. Для этого воспользуемся выражением:
\begin{equation}
	Q_i = \frac{\sum\limits_{k\,=\,1} ^ p \vline F_i ^ {(k)} \vline \cdot \imag a_i ^ {(k)}}{\imag a_i ^ {ref}}, \ i = 1 \hdots n,
\end{equation}
где $ n $~---~число отсчетов сигнала, $ \vline F_i ^ {(k)} \vline $~---~амплитуда воздействия в $ k $-ой точке, $ \imag a_i ^ {(k)} $~---~мнимая составляющая отклика сигнала $ k $-ой точке, $ \imag a_i ^ {ref} $~---~мнимая составляющая отклика сигнала в опорной точке.

Отметим, что расчеты по каждому из способов \eqref{eq:generalModalSolution}, \eqref{eq:decrementWidthImaginaryPeak}~--~\eqref{eq:decrementAngleReal} являются независимыми, поэтому осуществляются параллельно. Такой подход позволяет существенно ускорить производительность вычислений при высокой дискретизации сигнала отклика по уровню амплитуды.

По результатам вычислений было замечено, что изменение длины интерполяции на каждом расчетном уровне вне зависимости от расчетного подхода слабо влияет на результирующие значения обобщенных характеристик.

Также отметим, что собственная частота колебаний, определенная по обобщенным характеристикам в рамках четвертого способа, претерпевает малые изменения по мере роста относительного значения расчетного уровня \figref{fig:gencalc-natural-frequency}. Зависимости обобщенной массы, жесткости и демпфирования для этого расчетного случая приведены на рисунке~\ref{fig:gencalc-general-params}.

\begin{figure}[H]
	\centering
	\begin{subfigure}{0.49\textwidth}
		\includegraphics[width = 1\textwidth]{gencalc-general-stiffness} 
		\caption{Обобщенная жесткость}
	\end{subfigure}
	\hfill
	\begin{subfigure}{0.49\textwidth}
		\includegraphics[width = 1\textwidth]{gencalc-general-mass} 
		\caption{Обобщенная масса}
	\end{subfigure}
	\begin{subfigure}{0.49\textwidth}
		\includegraphics[width = 1\textwidth]{gencalc-general-damping} 
		\caption{Обобщенное демпфирование}
	\end{subfigure}
     \caption{Пример определения обобщенных характеристик конструкции} \label{fig:gencalc-general-params}
\end{figure}

\begin{figure}[H]
	\centering
	\includegraphics[width = 0.6\linewidth]{gencalc-natural-frequency}
	\caption{Пример определения частоты собственных колебаний конструкции по обобщенным характеристикам} \label{fig:gencalc-natural-frequency}
\end{figure}

\section{Диагностика дефектов конструкций по результатам испытаний}

В конструкциях многих технических изделий имеются зазоры (люфты), которые можно условно разделить на два вида. Одни из них~---~зазоры в соединениях составных частей конструкций~---~вводятся для обеспечения нормального функционирования этих соединений. Величины таких зазоров обычно нормируются. Другой вид~---~люфты, возникающие в процессе эксплуатации. Поскольку нормированные зазоры увеличиваются, как правило, в процессе эксплуатации, то оба этих вида могут привести к повышенной нагруженности и износу деталей, изменению динамических характеристик и ухудшению технического состояния изделий. Поэтому зазоры, конечно же, контролируются. Так как большинство технических изделий подвергается вибрационным испытаниям (прочностным, модальным, испытаниям на виброустойчивость), то представляется целесообразной разработка методики диагностики зазоров в этих испытаниях. 

Техническая вибродиагностика машинного оборудования нашла широкое распространение в машиностроении для контроля механических передач, соединительных муфт и подшипников \cite{lib:defects:Tiwari, lib:defects:Bachschmid, lib:defects:Kostjukov, lib:defects:Balickij, lib:defects:Zhukov}. Эти вращающиеся элементы машин при наличии дисбалансов, люфтов, несоосности и изгибов валов генерируют механические колебания. Колебания, регистрируемые на корпусных деталях машин как вибрации, содержат информацию о динамических процессах, которые происходят в работающей машине. Из этого объема информации необходимо выделить такие данные, на основании которых можно идентифицировать дефекты машин и отслеживать развитие этих дефектов \cite{lib:defects:Zhuge, lib:defects:Lacey, lib:defects:Litak}.

Методы вибродиагностики технических изделий по результатам испытаний разделятся на три группы. К первой из групп относятся методы обнаружения дефектов по изменению параметров собственных тонов колебаний \cite{lib:defects:Kisilev, lib:defects:Postnov, lib:defects:Kosicyn, lib:defects:Perera, lib:defects:Dilena, lib:defects:Xu, lib:defects:Barbieri}. Необходимо отметить, что нередко даже относительно большие повреждения слабо сказываются на изменении основных модальных параметров: частот и форм собственных колебаний. Более того, однозначная идентификация дефекта затруднена тем, что модальные параметры являются интегральными характеристиками, а расположение и величина дефекта~---~дифференциальными \cite{lib:defects:Doebling}.

Методы контроля дефектов по параметрам распространения упругих волн образуют вторую группу \cite{lib:defects:Viktorov, lib:defects:Worlton:ultrasonic, lib:defects:Worlton:experimental, lib:defects:Kessler, lib:defects:Zaitsev}. Но неоднородности конструкции в виде отверстий и вырезов осложняют использование этих методов.

Если в техническом изделии, проектные характеристики которого соответствуют линейной динамической системе, возникают суб- и супергармонические резонансы, искажения фазовых и других видов портретов колебаний, например, фигур Лиссажу, то методы обнаружения дефектов по этим признакам можно отнести к третьей группе \cite{lib:defects:Bovsunovsky, lib:defects:Tsifanskiy, lib:defects:Diana, lib:defects:Berns:align, lib:defects:Berns:backlash, lib:defects:AlKhazali, lib:defects:Berns:gap, lib:defects:Berns:experience, lib:defects:Berns:cracks}.

Как показано в работе \cite{lib:defects:Berns:experience}, для обнаружения и оценки величины зазоров в узлах проводки управления отклоняемыми поверхностями самолетов могут быть использованы нелинейные искажения портретов колебаний, которые определяются в модальных испытаниях.

В данном разделе излагается методика контроля зазоров в технических изделиях по искажениям портретов вынужденных колебаний в процессе любых вибрационных испытаний. Представлен способ поэтапного выявления всех зазоров в объекте испытаний, которые приводят к искажениям портретов колебаний. Это позволяет не только идентифицировать зазоры, но и оценивать их величины. В рамках описываемого подхода разработана и введена в программное обеспечение управления испытаниями подпрограмма анализа портретов колебаний. 

\subsection{Методика исследований}

Методика диагностирования дефектов в конструкциях летательных аппаратов по искажениям портретов колебаний заключается в следующем: на конструкции вблизи подвижных соединений и мест стыковки или крепления агрегатов и оборудования, а также в наиболее нагруженных местах устанавливаются датчики ускорений. Затем с помощью источников гармонических вибраций создаются вынужденные колебания конструкции. Эти колебания фиксируются акселерометрами и представляются в виде портретов: вертикальная развертка пропорциональна сигналу датчика, а горизонтальная~---~первой гармонике сигнала, сдвинутой по фазе на $ \sfrac{\pi}{2} $. Такой портрет колебаний для линейной динамической системы является окружностью. Наличие дефектов сопровождается нелинейными искажениями портретов колебаний из-за соударения элементов конструкции в зазорах, схлопывания трещин, трения в вершинах трещин и подвижных соединениях. Для численной оценки искажений из ряда Фурье для портрета колебаний вычитается первая гармоника, в остатке ряда определяется абсолютный максимум за период колебаний, величина которого $ \Psi $ принимается за параметр искажений. Величина параметра $ \Psi $ нормируется и обозначается как $ \xi $. Строится распределения $ \xi $ по объекту контроля. По расположениям локальных максимумов искажений определяются местоположения дефектов.

В расчетах параметра $ \xi $ используются два вида нормирования искажений $ \Psi $, условно названные глобальным и локальным. При глобальном нормировании величина $ \Psi $ относилась к амплитуде первой гармоники в контрольной точке конструкции. Предлагается принимать в качестве контрольной такую точку, в которой амплитуда колебаний первой гармоники наибольшая. В случае локального нормирования имеем:
\begin{equation}
	\xi_i = \frac{\max \ \vline \Psi_i \ \vline}{(A_1)_i},
\end{equation}
где $ (A_1)_i $~---~амплитуда колебаний первой гармоники, $ i $~---~номер канала измерений.

Глобальное нормирование необходимо для анализа распределения искажений портретов колебаний по всему изделию. Поскольку частоты вибрационного нагружения объектов испытаний находятся обычно в окрестности их собственных частот, то нужно исключить появление ложных локальных максимумов искажений. Это происходит из-за того, что некоторые акселерометры могут быть установлены вблизи узлов форм собственных колебаний конструкции.

Локальное нормирование искажений портретов колебаний используется для определения местоположений дефектов в отдельных агрегатах и узлах сопряжения конструкции. Такое нормирование позволяет сопоставить между собой проявления разных дефектов и отследить динамику изменения каждого из них в процессе испытаний или эксплуатации.

\subsection{Программная реализация}

Программа предназначена для контроля дефектов в процессе вибрационных испытаний, которые проводятся с использованием программного комплекса \name{Simcenter Testlab}. Для оценки состояния поврежденности конструкции проводится расчет и построение параметров искажений портретов колебаний, которые позволяют выявлять такие конструкционные дефекты, как зазоры (люфты), трещины и повышенное трение в подвижных соединениях. Исходными данными для расчета являются временные сигналы по выбранным каналам измерения. 

По команде экспериментатора~\figref{subfig:finder-interface-single} она осуществляет расчет параметров искажений портретов колебаний $ \xi $ параллельно по всем каналам измерений, строит распределения искажений по конструкции и запоминает такие распределения. Это позволяет контролировать проявление дефектов в течение вибропрочностных испытаний, а также эксплуатации конструкции путем сравнения полей параметра искажений~\figref{subfig:finder-interface-compare}, записанных для разных состояний изделий. Кроме того, в программе заложена возможность построения искажений портретов колебаний для отдельных агрегатов и узлов сопряжения конструкции, что необходимо, например, для поэтапного контроля дефектов.

\begin{figure}[!htb]
	\centering
	\begin{subfigure}{0.49\textwidth}
		\includegraphics[width = 1\textwidth]{finder-interface-single}
		\caption{Режим одиночного расчета} \label{subfig:finder-interface-single}
	\end{subfigure}
	\hfill
	\begin{subfigure}{0.49\textwidth}
		\includegraphics[width = 1\textwidth]{finder-interface-compare} 
		\caption{Режим сравнения} \label{subfig:finder-interface-compare}
	\end{subfigure}
    \caption{Графический интерфейс программы} 
\end{figure}

Для сохранения результирующих форм и портретов колебаний в навигаторе рабочего проекта~\figref{fig:finder-results-structure}, программа использует интерфейс \name{Testlab Automation}.

\begin{figure}[H]
	\centerfloat
	\includegraphics[width = 0.8\linewidth]{finder-results-structure}
	\caption{Расположение результатов работы программы в дереве навигатора проекта \name{Simcenter Testlab}} \label{fig:finder-results-structure}
\end{figure}

\subsection{Применение методики для диагностирования зазоров и люфтов}

Методика обнаружения зазоров по искажениям портретов колебаний была использована для диагностирования самолетов в процессе модальных испытаний, а также космических аппаратов открытого исполнения в технологических вибрационных испытаниях. 

На рисунках~\ref{fig:distortion-airplane-forewing}~--~\ref{fig:distortion-vertical-stabilizer} приведены примеры распределений искажений портретов колебаний, полученные в модальных испытаниях нескольких самолётов. Здесь и далее на рисунках красной цветовой гамме соответствуют области изделий с наибольшими искажениями, а синей~---~с наименьшими. 

\begin{figure}[!htb]
	\centerfloat
	\includegraphics[width = 0.5\linewidth]{distortion-airplane-forewing}
	\caption{Зазоры в узлах крепления переднего горизонтального оперения (ПГО). Глобальная нормировка искажений на частоте вращения ПГО} \label{fig:distortion-airplane-forewing}
\end{figure}

\begin{figure}[!htb]
	\centerfloat
	\includegraphics[width = 0.5\linewidth]{distortion-airplane-stabilizer}
	\caption{Зазоры в узлах крепления цельноповоротного стабилизатора. Глобальная нормировка искажений на частоте вращения стабилизатора} \label{fig:distortion-airplane-stabilizer}
\end{figure}

\begin{figure}[!htb]
	\centerfloat
	\includegraphics[width = 0.5\linewidth]{distortion-high-lift}
	\caption{Зазоры в проводках управления механизацией крыла самолёта. Глобальная нормировка искажений на частоте изгиба крыла} \label{fig:distortion-high-lift}
\end{figure}

На рисунке~\ref{fig:distortion-wiring-gap} представлены искажения портретов колебаний для самолёта с безбустерной системой управления (фюзеляж не показан). Видно, что максимумы искажений находятся на руле высоты и триммере из-за зазоров в проводках управления. Исключение этих искажений из рассмотрения приводит к локализации максимума искажений в соединении ручки управления с проводкой управления, где обнаружен повышенный люфт~\figref{fig:distortion-wiring-backlash}.

\begin{figure}[!htb]
	\centerfloat
	\includegraphics[width = 0.6\linewidth]{distortion-wiring-gap}
	\caption{Зазоры в проводке управления рулем высоты и триммером. Глобальная нормировка искажений портретов колебаний \\ a) датчики на ручке управления, b) искажения на руле высоты и триммере} \label{fig:distortion-wiring-gap}
\end{figure}

На рисунке~\ref{fig:distortion-vertical-stabilizer} показаны распределения искажений портретов колебаний вертикального оперения самолёта. В передних болтовых соединениях киля с фюзеляжем обнаружены повышенные зазоры в поперечном направлении. 

\begin{figure}[!htb]
	\centerfloat
	\includegraphics[width = 0.6\linewidth]{distortion-wiring-backlash}
	\caption{Люфт в соединении ручки управления с проводкой. Глобальная нормировка искажений портретов колебаний} \label{fig:distortion-wiring-backlash}
\end{figure}

\begin{figure}[!htb]
	\centerfloat
	\includegraphics[width = 0.3\linewidth]{distortion-vertical-stabilizer}
	\caption{Зазоры в передних узлах крепления киля. Локальная нормировка искажений} \label{fig:distortion-vertical-stabilizer}
\end{figure}

Космические аппараты (КА) в ходе создания подвергаются технологическим вибрационным испытаниям. Результаты используются для подтверждения качества спроектированной конструкции КА и обеспечения ее вибрационной прочности, в том числе для обнаружения производственно-технологических дефектов. Поскольку наибольшие вибрационные нагрузки воздействуют на КА во время его выведения на орбиту, то испытаниям подвергаются КА в стартовой конфигурации. На рисунке~\ref{fig:spacecraft-scheme} показана конструктивно-компоновочная схема КА открытого исполнения. Силовым каркасом является углепластиковой цилиндр с закрепленными на нем сотовыми плоскими панелями. Оборудование КА (антенны, солнечные батареи и т.д), а также астроплата с датчиками системы ориентации и стабилизации расположены на панелях. Для проведения испытаний КА устанавливается на адаптер, предназначенный для стыковки КА с ракетой-носителем.

Вибрационная диагностики КА проводится в несколько этапов \cite{lib:defects:Berns:experience}. На первом из этапов выполняется вибрационное нагружение низкой интенсивности с целью проверки соответствия динамических характеристик КА их проектным значениям. На втором этапе происходит нагружение КА нормированным вибрационным воздействием. В ходе нагружения могут возникать и развиваться дефекты, например, нарушаться межблочные связи за счет появления зазоров. Третий этап повторяет программу нагружения первого. На основании изменения параметров вибраций: резонансной частоты и амплитуды колебаний; появлению высокочастотных составляющих в отклике КА и сдвигу частотного спектра определяют местоположения и характер дефектов.

\begin{figure}[!htb]
	\centerfloat
	\includegraphics[width = 0.4\linewidth]{spacecraft-scheme}
	\caption{Конструктивно-компоновочная схема космического аппарата \\ 1~---~адаптер, 2~---~панель, 3~---~астроплата, 4~---~рефлекторы антенн, 5~---~панели солнечной батареи, 6~---~узел крепления солнечной батареи} \label{fig:spacecraft-scheme}
\end{figure}

В вибрационных испытаниях КА открытого исполнения используется как гармоническая, так и широкополосная случайная вибрация при акустическом нагружении.

На рисунках~\ref{fig:distortion-spacecraft} и~\ref{fig:distortion-antenna} представлены результаты обнаружения зазоров по искажениям портретов колебаний применительно к конструкциям двух КА. Необходимо отметить, что в испытаниях нагружение этих КА производилось синусоидальной вибрацией, частота которой изменялась по логарифмическому закону. Поскольку вынужденные колебания КА являлись нестационарным процессом, то в окрестностях резонансных частот объектов испытаний выделялись временные сегменты, для которых в глобальной нормировке вычислялись искажения портретов колебаний. Среди всех распределений выбирались те, которым соответствуют наибольшие значения искажений. 

На рисунке~\ref{fig:distortion-spacecraft} показаны распределения искажений портретов колебаний по поверхности одного из испытываемых КА. Это единственный вариант распределений в диапазоне частот колебаний от $ 20 $ до $ 100 $ Гц, в котором искажения портретов превышали погрешности их построения. А наибольшие искажения возникали вблизи узлов установки солнечных батарей, в которых имеются конструктивные зазоры. 

\begin{figure}[!htb]
	\centerfloat
	\includegraphics[width = 0.5\linewidth]{distortion-spacecraft}
	\caption{Проявление зазоров в узлах установки солнечных батарей} \label{fig:distortion-spacecraft}
\end{figure}

На рисунке~\ref{fig:antenna-scheme} представлена схема установки для вибрационных испытаний антенны другого КА и распределение искажений портретов колебаний рефлектора антенны. Точками на рисунке~\ref{fig:distortion-antenna} отмечены места установки датчиков ускорений на поверхности рефлектора. Стрелкой обозначено местоположение дефекта: разрушение клеевого соединения одной из опор рефлектора с его каркасом, в результате чего возник зазор. Этому месту соответствуют и наибольшие искажения портретов колебаний.

\begin{figure}[H]
	\centerfloat
	\includegraphics[width = 0.5\linewidth]{antenna-scheme}
	\caption{Установка для испытаний антенны \\ 1~---~каркас, 2~---~рефлектор, 3~---~вибростенд} \label{fig:antenna-scheme}
\end{figure}

\begin{figure}[!htb]
	\centerfloat
	\includegraphics[width = 0.5\linewidth]{distortion-antenna}
	\caption{Искажения портретов колебаний рефлектора антенны} \label{fig:distortion-antenna}
\end{figure}

\section{Обработка и представление результатов в процессе испытаний}

Одним из ключевых требований обеспечения непрерывности производственного процесса авиационной техники является сокращение времени между натурными испытаниями и первым вылетом изделия. Для удовлетворения этого условия необходимо осуществлять обработку и представление результатов модального анализа непосредственно в процессе испытаний. Это позволит оперативно составить заключение о полноте экспериментальных данных, необходимых для коррекции расчетной модели объекта испытаний. 

Для решения этой задачи авторами на языке программирования \name{C\#} была разработана программа \name{ResponseAnalyzer} для представления результатов модальных испытаний, проведенных с использованием программного комплекса \name{Simcenter Testlab}. Программная реализация использует интерфейс \name{Testlab Automation} для получения и обработки сигналов с датчиков акселерометров, геометрии и информации о ходе проведения эксперимента.

Посредством графического интерфейса \figref{fig:analyzer-interface} возможен выбор как отдельных геометрических точек конструкции, так и их комбинаций, для построения амплитудно-частотных характеристик при одном~\figref{fig:analyzer-response} и разных уровней нагружения ~\figref{fig:analyzer-multi-response}, форм колебаний~\figref{fig:analyzer-modeshape}, зависимостей резонансных частот собственных колебаний от амплитуд возбуждения \figref{fig:analyzer-frequency-dependency}. Данные пользовательского выбора сохраняются в виде бинарных шаблонов, которые могут использоваться для обработки результатов повторных испытаний рассматриваемой конструкции. Формой представления результатов работы программы являются электронные таблицы \name{Excel}. В соответствии с пользовательским форматированием этих таблиц происходит размещение результатов. Так, пользователь может поставить в соответствие графическим объектам типа <<Диаграмма>> различные экспериментальные данные, определив стиль и очередность их отображения: толщину и тип линий, параметры маркеров. С целью обеспечения пользовательского контроля результатов обработки, данные, использованные для построения этих графических объектов, наряду со служебной информацией о работе программы, приводятся в одном из разделов результирующих таблиц.

Для осуществления выбора сигналов используется раздел навигации \name{Simcenter Testlab} \figref{fig:analyzer-lms-structure}. При этом пользователю доступна загрузка результатов, соответствующих как одному, так и нескольким экспериментам. Кроме того, в случае, когда необходимо исключить выбросы в экспериментальных данных и/или испытания проведены в широком частотном диапазоне, пользователю доступен выбор значений частот для формирования результирующих таблиц.

Взаимодействие с геометрией исследуемого объекта осуществляется с помощью пользовательского ввода и контекстного меню \figref{fig:analyzer-geometry-selection}, в котором доступно для выбора: модель отображения конструкции (полигональная и сетчатая), модель отображения узлов (маркеры и имена), отображение отдельных конструкционных элементов, модель освещения, а также стандартные виды (изометрические и проективные).

\begin{figure}[H]
	\centerfloat
	\includegraphics[width = 0.9\linewidth]{analyzer-interface}
	\caption{Графический интерфейс программы} \label{fig:analyzer-interface}
\end{figure}

\begin{figure}[H]
	\centerfloat
	\includegraphics[width = 0.9\linewidth]{analyzer-geometry-selection}
	\caption{Выбор точек конструкции с помощью графического меню} \label{fig:analyzer-geometry-selection}
\end{figure}

\begin{figure}[H]
	\centerfloat
	\includegraphics[width = 0.9\linewidth]{analyzer-lms-structure}
	\caption{Выбор сигналов для построения графических объектов посредством дерева навигации \name{Simcenter Testlab}} \label{fig:analyzer-lms-structure}
\end{figure}

\begin{figure}[H]
	\centerfloat
	\includegraphics[width = 0.9\linewidth]{analyzer-modeshape}
	\caption{Построение формы колебаний изделия по тону СИК1} \label{fig:analyzer-modeshape}
\end{figure}

\begin{figure}[H]
	\centering
	\begin{subfigure}{0.49\textwidth}
		\includegraphics[width = 1\textwidth]{analyzer-response-imaginary}
		\caption{Мнимая составляющая}
	\end{subfigure}
	\hfill
	\begin{subfigure}{0.49\textwidth}
		\includegraphics[width = 1\textwidth]{analyzer-response-real}
		\caption{Действительная составляющая}
	\end{subfigure}
     \caption{Отклик в крыльевых точках при изгибе изделия по тону СИК1} \label{fig:analyzer-response}
\end{figure}

\begin{figure}[H]
	\centering
	\begin{subfigure}{0.49\textwidth}
		\includegraphics[width = 1\textwidth]{analyzer-multi-response-imaginary}
		\caption{Мнимая составляющая}
	\end{subfigure}
	\hfill
	\begin{subfigure}{0.49\textwidth}
		\includegraphics[width = 1\textwidth]{analyzer-multi-response-real}
		\caption{Действительная составляющая}
	\end{subfigure}
     \caption{Отклик изделия по тону вращения правого руля направления (ВрПрРН) при разных уровнях нагружения} \label{fig:analyzer-multi-response}
\end{figure}

\begin{figure}[H]
	\centerfloat
	\includegraphics[width = 0.6\linewidth]{analyzer-frequency-dependency}
	\caption{Зависимость частоты от амплитуды колебаний для тона ВрПрРН} \label{fig:analyzer-frequency-dependency}
\end{figure}

\section{Операционный модальный анализ}

\subsection{Методика декомпозиции виброускорений}

\fixme{Обоснование актуальности наличием дрейфа в показаниях датчиков. Приводится изложение метода для постоянных амплитуд. Оценивается чувствительность в зависимости от зашумленности сигнала}

Представим ускорения в каждый момент времени как:
\begin{equation}
	\alpha_i(\mat{p}) = \sum\limits_{k\,=\,1} ^ m A_k e ^ {-\eta_k (t_i - t_0)} \sbrackets{\rho_{1,i}(\mat{p}) + \rho_{2, i}(\mat{p})}, \ i = 1 \hdots n.
\end{equation}

Гармонические слагаемые, входящие в это разложение, запишутся:
\begin{equation}
	\begin{gathered}
		\rho_{1, i}(\mat{p}) = \cos \sbrackets{\varphi_k + \omega_k (t_i - t_0)} \rbrackets{\eta_k ^ 2 - \omega_k ^ 2}, \\
		\rho_{2, i}(\mat{p}) = 2 \eta_k \omega_k \sin \sbrackets{\varphi_k + \omega_k (t_i - t_0)},
	\end{gathered}
\end{equation}
где $ m $~---~число гармоник в разложении, $ n $~---~число временных отсчетов, $ \eta_k $~---~относительный коэффициент демпфирования, $ \varphi_k $ и $ \omega_k $~---~фаза и частота колебаний. 

Вектор варьируемых параметров разложения:
\begin{equation}
	\mat{p} = 
	\begin{Bmatrix}
		\varphi_k & A_k & \omega_k & \eta_k
	\end{Bmatrix}, \ k = 1 \hdots m.
\end{equation}

Ошибка представления $ \alpha_i(\mat{p}) $ экспериментальных ускорений $ a_i $ имеет вид:
\begin{equation}
	\mat{f(\mat{p})} = 
	\begin{Bmatrix}
		\alpha_0(\mat{p}) - a_0 & \hdots & \alpha_i(\mat{p}) - a_i & \hdots & \alpha_n(\mat{p}) - a_n
	\end{Bmatrix}.
\end{equation}

Таким образом, получаем задачу оптимизации:
\begin{equation}
	\begin{gathered}
		\mat{F}(\mat{p}) = \vert \vert \mat{f} \vert \vert ^ 2 \rightarrow \min_{\mat{p}}, \\
		\nabla \mat{F(\mat{p})} = 2 \trans{\mat{J}(\mat{p})} \mat{f(\mat{p})},
	\end{gathered}
\end{equation}
где $ \mat{J}(p) $~---~матрица Якоби по неизвестным параметрам.

\subsection{Тестирование на примере имитационной модели ЛА}

Рассмотрим имитационную модель беспилотного летательного аппарата \name{XQ-58 Valkyrie}~\figref{fig:x58-geometry} \cite{lib:misc:x58}. На основании геометрической модели планера была создана конечно-элементная модель \name{Ansys} \figref{fig:x58-mesh}. При этом жесткостные характеристики элементов планера подбирались таким образом, чтобы приблизить спектр частот собственных колебаний к тому, который наблюдается на летательных аппаратах схожей компоновки.

\begin{figure}[!htb]
	\centerfloat
	\includegraphics[width = 0.75\linewidth]{x58-geometry}
	\caption{Геометрическая модель беспилотного летательного аппарата} \label{fig:x58-geometry}
\end{figure}

На каждом из элементов планера была размещена сеть виртуальных датчиков, связанных между собой треугольными полигонами. Пространственная схема расположения датчиков, общее количество которых составило $ 101 $, приведена на рисунке \ref{fig:x58-sensors}.

Для определения модальных характеристик планера к законцовкам крыла было приложено однократное импульсное воздействие длительностью $ 0.04 $ с. Данные динамических откликов записывались с частотой дискретизации $ 2 $ кГц по каждому пространственному направлению во всех датчиках в течение $ 5 $ с. Временные зависимости ускорений вдоль направления Y, полученные в нескольких точках левой консоли крыла, приведены на рисунке~\ref{fig:x58-signals}. Распределение спектральной плотности мощности, соответствующей этим сигналам, показано на рисунке~\ref{fig:x58-spectrums}.

\begin{figure}[!htb]
	\centerfloat
	\includegraphics[width = 0.75\linewidth]{x58-mesh}
	\caption{КЭ-модель беспилотного летательного аппарата} \label{fig:x58-mesh}
\end{figure}

\begin{figure}[!htb]
	\centerfloat
	\includegraphics[width = 0.75\linewidth]{x58-sensors}
	\caption{Схема размещения датчиков} \label{fig:x58-sensors}
\end{figure}

\begin{figure}[!htb]
	\centerfloat
	\includegraphics[width = 1\linewidth]{x58-signals}
	\caption{Временные сигналы ускорений в нескольких точках левой консоли крыла} \label{fig:x58-signals}
\end{figure}

\begin{figure}[!htb]
	\centerfloat
	\includegraphics[width = 1\linewidth]{x58-spectrums}
	\caption{Спектральная плотность мощности в нескольких точках левой консоли крыла} \label{fig:x58-spectrums}
\end{figure}

Необходимо заметить, что не все методы операционного модального анализа, которые рассматриваются в настоящей работе, достаточно эффективны с численной точки зрения для одновременной обработки всего массива данных откликов. Так, посредством метода \name{ERA} удается обработать лишь один элемент планера одномоментно. Наиболее высокопроизводительным и точным применительно к рассматриваемой задачи оказался метод \name{SSI-COV}. В таблице~\ref{tab:x58-ssi-cov-results} сведены частоты $ f $ и логарифмические декременты колебаний $ \delta $, определенные по конечно-элементной модели и методом \name{SSI-COV}. По этим данным вычислены погрешности определения модальных характеристик, которые приведены в двух последних столбцах. При этом определенные формы колебаний \name{SSI-COV} совпадают с расчетными \name{Ansys}. Сравнение форм колебаний на примере пятого тона собственных колебаний приведено на рисунке~\ref{fig:x58-mode-compare}. 

\begin{longtblr}[
	caption = {Результат определения модальных характеристик методом SSI-COV}, 
	label = {tab:x58-ssi-cov-results}, 
]{
	colspec = {|c|c|c|c|c|c|c|},
	hlines
}
	\SetCell[r=2]{c} Тон & \SetCell[c=2]{c} Ansys && \SetCell[c=2]{c} SSI-COV && \SetCell[c=2]{c} Погрешность, \% & \\
	& $ f $, Гц & $ \delta $ & $ f $, Гц & $ \delta $ & $ \Delta \overline{f} $ & $ \Delta \overline{\delta} $ \\ \hline
	1 & 1.8343 & \SetCell[r=11]{c} 0.0628 & 1.8343 & 0.0628 & -0.0018 & 0.0018 \\ 
	2 & 7.0324 & & 7.0322 & 0.0628 & -0.0035 & 0.0050 \\ 
	3 & 10.2304 & & 10.2300 & 0.0628 & -0.0042 & -0.0141 \\ 
	4 & 21.7268 & & 21.7180 & 0.0628 & -0.0405 & -0.0730 \\ 
	5 & 25.9578 & & 25.9430 & 0.0628 & -0.0571 & -0.1016 \\ 
	6 & 30.8802 & & 30.8560 & 0.0627 & -0.0784 & -0.1732 \\ 
	7 & 35.5900 & & 35.5530 & 0.0627 & -0.1039 & -0.2035 \\ 
	8 & 38.4500 & & 38.4030 & 0.0627 & -0.1223 & -0.2369 \\ 
	9 & 40.3791 & & 40.3250 & 0.0626 & -0.1340 & -0.2942 \\ 
	10 & 40.4966 & & 40.4420 & 0.0627 & -0.1348 & -0.2608 \\ 
	11 & 52.6507 & & 52.5310 & 0.0626 & -0.2274 & -0.4486 \\ 
\end{longtblr}

\begin{figure}[!htb]
	\centering
	\begin{subfigure}{0.49\textwidth}
		\includegraphics[width = 1\textwidth]{x58-ansys-mode-5}
		\caption{\name{Ansys}}
	\end{subfigure}
	\begin{subfigure}{0.49\textwidth}
		\includegraphics[width = 1\textwidth]{x58-ssi-cov-mode-5}
		\caption{\name{SSI-COV}}
	\end{subfigure}
     \caption{Сопоставление расчетных и определенных форм собственных колебаний тона №5} \label{fig:x58-mode-compare}
\end{figure}

Остальные формы колебаний, определенные методом \name{SSI-COV}, приведены на рисунках~\ref{subfig:x58-ssi-cov-mode-1}~--~\ref{subfig:x58-ssi-cov-mode-4}.

\def\sfX58{0.48\textwidth}

\begin{figure}[!htb]
	\centering
	\begin{subfigure}[b]{\sfX58}
		\includegraphics[width = \textwidth]{x58-ssi-cov-mode-1}
		\caption{1.83 Гц (траекторный)} \label{subfig:x58-ssi-cov-mode-1}
	\end{subfigure}
	\hfill
	\begin{subfigure}[b]{\sfX58}
		\includegraphics[width = \textwidth]{x58-ssi-cov-mode-2}
		\caption{7.03 Гц}
	\end{subfigure}
	\begin{subfigure}[b]{\sfX58}
		\includegraphics[width = \textwidth]{x58-ssi-cov-mode-3}
		\caption{10.23 Гц}
	\end{subfigure}	
	\hfill
	\begin{subfigure}[b]{\sfX58}
		\includegraphics[width = \textwidth]{x58-ssi-cov-mode-4}
		\caption{21.71 Гц} \label{subfig:x58-ssi-cov-mode-4}
	\end{subfigure}	
	\caption{Пример форм колебаний имитационной модели по методу \name{SSI-COV} (a~--~г)}
\end{figure}

Для количественной оценки соответствия определенных форм колебаний их расчетным аналогам, воспользуемся критерием модального соответствия. Численная оценка качества выделения форм колебаний методом \name{SSI-COV} для первых двенадцати тонов колебаний приведена на рисунке~\ref{fig:x58-mac}. Из рисунка видно, что определенные формы колебаний практически совпадают с расчетными.

Стабилизационная диаграмма, построенная по методу \name{SSI-COV}, показана на рисунке~\ref{fig:x58-ssi-cov}.

\begin{figure}[H]
	\centerfloat
	\includegraphics[width = 1\linewidth]{x58-ssi-cov}
	\caption{Стабилизационная диаграмма по методу \name{SSI-COV}} \label{fig:x58-ssi-cov}
\end{figure}

\begin{figure}[!htb]
	\centerfloat
	\includegraphics[width = 1\linewidth]{x58-mac}
	\caption{Критерий модального соответствия расчетных форм колебаний их определенным аналогам} \label{fig:x58-mac}
\end{figure}

\subsection{Определение модальных характеристик по результатам акустических испытаний}

Определим модальные характеристики рефлектора \figref{fig:reflector-experiment} по результатам отклика на шумовое акустическое воздействие. Для записи откликов использовались акселерометры~\figref{fig:reflector-sensors}, размещенные на поверхности рефлектора в соответствии со схемой, приведенной на рисунке~\ref{fig:reflector-scheme}.

\begin{figure}[!htb]
	\centerfloat
	\includegraphics[width = 1\linewidth]{reflector-experiment}
	\caption{Рефлектор в сборе с испытательной оснасткой} \label{fig:reflector-experiment}
\end{figure}

\begin{figure}[!htb]
	\centerfloat
	\includegraphics[width = 0.9\linewidth]{reflector-sensors}
	\caption{Датчики ускорений, размещенные на поверхности рефлектора} \label{fig:reflector-sensors}
\end{figure}

\begin{figure}[!htb]
	\centerfloat
	\includegraphics[width = 0.9\linewidth]{reflector-scheme}
	\caption{Схема расстановки акселерометров по рефлектору} \label{fig:reflector-scheme}
\end{figure}

Длительность шумового воздействия составила $ 53 $ секунды при частоте дискретизации сигналов равной $ 12800 $ Гц. Пример временных сигналов акселерометров вдоль направления Y показан на рисунке~\ref{fig:reflector-signals}. Спектральная плотность мощности исследуемых сигналов приведена на рисунке~\ref{fig:reflector-spectrums}. Обработка сигналов осуществлялась последовательно вдоль каждого из пространственных направлений тремя методами операционного модального анализа: \name{SSI-COV}, \name{ERA} и \name{SSI-DD}. При этом сигнал шумового воздействия не использовался. Частоты и логарифмические декременты колебаний, полученные посредством каждого из методов, сведены в таблице~\ref{tab:reflector-results}.

Формы колебаний, определенные методами операционного модального анализа, приведены на рисунках~\ref{subfig:reflector-ssi-cov-mode-2}~--~\ref{subfig:reflector-ssi-cov-mode-16}.

\begin{figure}[!htb]
	\centerfloat
	\includegraphics[width = 0.84\linewidth]{reflector-signals}
	\caption{Временные сигналы акселерометров вдоль направления Y} \label{fig:reflector-signals}
\end{figure}

\begin{figure}[!htb]
	\centerfloat
	\includegraphics[width = 0.84\linewidth]{reflector-spectrums}
	\caption{Спектральная плотность мощности временных сигналов акселерометров вдоль направления Y} \label{fig:reflector-spectrums}
\end{figure}

\begin{longtblr}[
	caption = {Результаты определения частот и логарифмических декрементов колебаний методами операционного модального анализа}, 
	label = {tab:reflector-results}, 
]{
	colspec = {|c|c|c|c||c|c|c|},
	hlines
}
	\SetCell[r = 2]{c} Тон & \SetCell[c = 3]{c} Частота, Гц && & \SetCell[c = 3]{c} Логарифмический декремент && \\
	& SSI-COV & ERA & SSI-DD & SSI-COV & ERA & SSI-DD \\ \hline
	1 & 66.052 & 65.962 & 66.403 & 0.054 & 0.042 & 0.045 \\
	2 & 91.287 & 90.910 & 91.843 & 0.030 & 0.052 & 0.042 \\
	3 & 102.540 & --- & --- & 0.056 & --- & --- \\
	4 & 114.770 & --- & --- & 0.063 & --- & --- \\
	5 & 122.370 & --- & --- & 0.094 & --- & --- \\
	6 & 127.850 & --- & --- & 0.049 & --- & --- \\
	7 & 157.560 & --- & --- & 0.121 & --- & --- \\
	8 & 203.850 & --- & --- & 0.060 & --- & --- \\
	9 & 208.270 & --- & --- & 0.062 & --- & --- \\
	10 & 227.700 & --- & --- & 0.057 & --- & --- \\
	11 & 243.360 & --- & --- & 0.048 & --- & --- \\
	12 & 273.450 & 273.310 & --- & 0.033 & 0.026 & --- \\
	13 & 283.300 & --- & 281.990 & 0.040 & --- & 0.033 \\
	14 & 325.360 & --- & 324.020 & 0.034 & --- & 0.056 \\
\end{longtblr}

На основе таблицы~\ref{tab:reflector-results} можем заключить, что метод \name{SSI-COV} позволяет выделить наибольшее число тонов колебаний. Оценим сходимость модальных характеристик, определяемых этим методом: частот~\tabref{tab:reflector-conv-time-frequency} и логарифмических декрементов колебаний~\tabref{tab:reflector-conv-time-decrement}, варьируя длительность временных сигналов от $ 5 $ до $ 50 $ секунд. 

\def\sfReflector{0.48\textwidth}

\begin{figure}[H]
	\centering
	\begin{subfigure}[b]{\sfReflector}
		\includegraphics[width = \textwidth]{reflector-ssi-cov-mode-2}
		\caption{91.29 Гц} \label{subfig:reflector-ssi-cov-mode-2}
	\end{subfigure}
	\hfill
	\begin{subfigure}[b]{\sfReflector}
		\includegraphics[width = \textwidth]{reflector-ssi-cov-mode-6}
		\caption{127.85 Гц}
	\end{subfigure}
	\begin{subfigure}[b]{\sfReflector}
		\includegraphics[width = \textwidth]{reflector-ssi-cov-mode-13}
		\caption{283.30 Гц}
	\end{subfigure}	
	\hfill
	\begin{subfigure}[b]{\sfReflector}
		\includegraphics[width = \textwidth]{reflector-ssi-cov-mode-16}
		\caption{329.54 Гц} \label{subfig:reflector-ssi-cov-mode-16}
	\end{subfigure}	
	\caption{Пример форм колебаний рефлектора, определенных методом \name{SSI-COV} (a~--~г)}
\end{figure}

\begin{longtblr}[
	caption = {Cходимость частот собственных колебаний в зависимости от длины временного сегмента}, 
	label = {tab:reflector-conv-time-frequency}
]{
	colspec = {|c|c|c|c|c|c|c|},
	hlines
}
	\SetCell[r = 2]{c} Тон & \SetCell[c = 6]{c} Длительность сегмента, c &&&&& \\
	& 5 & 10 & 20 & 30 & 40 & 50 \\ \hline
	1 & 66.096 & 65.993 & 66.102 & 66.04 & 66.042 & 66.052 \\
	2 & 91.335 & 91.193 & 91.174 & 91.259 & --- & 91.287 \\
	3 & --- & --- & --- & 102.87 & 102.64 & 102.54 \\
	4 & 115.25 & --- & 114.79 & 114.97 & 114.21 & 114.77 \\
	5 & 124.26 & 119.06 & 122.3 & 122.09 & 122.24 & 122.37 \\
	6 & --- & --- & 130.13 & --- & 128.34 & 127.85 \\
\end{longtblr}

\begin{longtblr}[
	caption = {Cходимость логарифмического декремента колебаний в зависимости от длины временного сегмента}, 
	label = {tab:reflector-conv-time-decrement}
]{
	colspec = {|c|c|c|c|c|c|c|}, 
	hlines
}
	\SetCell[r = 2]{c} Тон & \SetCell[c = 6]{c} Длительность сегмента, c &&&&& \\
	& 5 & 10 & 20 & 30 & 40 & 50 \\ \hline
	1 & 0.060 & 0.041 & 0.052 & 0.052 & 0.051 & 0.054 \\
	2 & 0.027 & 0.027 & 0.025 & 0.027 & --- & 0.030 \\
	3 & --- & --- & --- & 0.060 & 0.057 & 0.056 \\
	4 & 0.081 & --- & 0.070 & 0.073 & 0.068 & 0.063 \\
	5 & 0.086 & 0.090 & 0.098 & 0.101 & 0.096 & 0.094 \\
	6 & --- & --- & 0.022 & --- & 0.043 & 0.049 \\
\end{longtblr}

Дополнительно оценим влияния частоты дискретизации на устойчивость численных значений модальных характеристик, определяемых методом \name{SSI-COV}. Для этого временные отклики прореживаются с интервалами, которые соответствует диапазону частот дискретизации от $ 12800 $ до $ 2560 $ Гц. Результаты по частотам и логарифмическим декрементам сведены в таблицах~\ref{tab:reflector-conv-sample-frequency} и \ref{tab:reflector-conv-sample-decrement} соответственно. 

\begin{longtblr}[
	caption = {Cходимость частот собственных колебаний в зависимости от частоты дискретизации}, 
	label = {tab:reflector-conv-sample-frequency}
]{
	colspec = {|c|c|c|c|c|c|}, 
	hlines
}
	\SetCell[r = 2]{c} Тон & \SetCell[c = 5]{c} Частота дискретизации, Гц &&&& \\
	& 12800 & 6400 & 4267 & 3200 & 2560 \\ \hline
	1 & 66.052 & 65.874 & 65.933 & 65.891 & 65.841 \\
	2 & 91.287 & 91.292 & 91.201 & 91.136 & 91.049 \\
	3 & 102.539 & 102.476 & 102.458 & 102.763 & 103.266 \\
	4 & 114.772 & 114.007 & 114.044 & 114.134 & 114.196 \\
	5 & 122.375 & 123.003 & 123.034 & 123.123 & 123.257 \\
	6 & 127.851 & 129.596 & 129.586 & 129.580 & 129.599 \\
\end{longtblr}

\begin{longtblr}[
	caption = {Cходимость логарифмического декремента колебаний в зависимости от частоты дискретизации}, 
	label = {tab:reflector-conv-sample-decrement}, 
]{
	colspec = {|c|c|c|c|c|c|}, 
	hlines
}
	\SetCell[r = 2]{c} Тон & \SetCell[c = 5]{c} Частота дискретизации, Гц &&&& \\
	& 12800 & 6400 & 4267 & 3200 & 2560 \\ \hline
	1 & 0.054 & 0.034 & 0.030 & 0.032 & 0.029 \\
	2 & 0.030 & 0.025 & 0.031 & 0.035 & 0.037 \\
	3 & 0.056 & 0.062 & 0.068 & 0.064 & 0.055 \\
	4 & 0.063 & 0.053 & 0.052 & 0.052 & 0.053 \\
	5 & 0.094 & 0.061 & 0.057 & 0.056 & 0.050 \\
	6 & 0.049 & 0.020 & 0.021 & 0.022 & 0.022 \\
\end{longtblr}

Можем видеть, что временного сегмента длительностью $ 20 $ секунд достаточно для стабилизации численных значений определяемых модальных характеристик.

В случае изменения частоты дискретизации частота собственных колебаний меняется незначительно, в то время как логарифмический декремент колебаний меняется в разы. 

\subsection{Апробация на динамически подобной модели}

\fixme{Описание модели и типа ударного воздействия. Гистограммы для сопоставления результатов экспериментального и операционного модального анализа}

\subsection{Оценка модальных параметров летательных аппаратов по результатам полетов в неспокойной атмосфере}

\fixme{Коротко о сути подходов: развертка, широкополосное и импульсное возбуждения. Алгоритм автоматической разметки границ.}

\section{Выводы по главе \thechapter}