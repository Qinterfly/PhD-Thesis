% --------------------------------------- %
% --- Основные сведения о диссертации --- %
% --------------------------------------- %

% --- Данные автора --- %

\newcommand{\thesisAuthorLastName}{Лакиза}
\newcommand{\thesisAuthorOtherNames}{Павел Анатольевич}
\newcommand{\thesisAuthorInitials}{П.\,А.}
\newcommand{\thesisAuthor}             
{%
	\texorpdfstring{
        \thesisAuthorLastName~\thesisAuthorOtherNames  % Так будет отображаться на титульном листе или в тексте, где будет использоваться переменная
    }{%
        \thesisAuthorLastName, \thesisAuthorOtherNames % Эта запись для свойств pdf-файла. В таком виде, если pdf будет обработан программами для сбора библиографических сведений, будет правильно представлена фамилия.
    }
}
\newcommand{\thesisAuthorShort}{\thesisAuthorInitials~\thesisAuthorLastName}

% --- Данные диссертации --- %

\newcommand{\thesisTitle}{Коррекция расчетных моделей летательных аппаратов по результатам модальных испытаний}
\newcommand{\thesisSpecialtyNumber}{2.5.14}
\newcommand{\thesisSpecialtyTitle}{Прочность и тепловые режимы летательных аппаратов}
\newcommand{\thesisDegree}{кандидата технических наук}
\newcommand{\thesisDegreeShort}{канд.~тех.~наук}
\newcommand{\thesisCity}{Новосибирск}
\newcommand{\thesisYear}{2023}
\newcommand{\thesisFirstOrganization}{Федеральное автономное учреждение <<Сибирский научно-исследовательский институт авиации им.~С.\,А.~Чаплыгина>>}
\newcommand{\thesisSecondOrganization}{Федеральное государственное бюджетное образовательное учреждение высшего образования <<Новосибирский государственный технический университет>>}

% --- Данные научного руководителя --- %

\newcommand{\supervisorFio}{Бернс Владимир Андреевич}
\newcommand{\supervisorRegalia}{доктор технических наук,~профессор}
\newcommand{\supervisorFioShort}{В.\,А.~Бернс}
\newcommand{\supervisorRegaliaShort}{д-р техн.~наук,~проф.}

% --- Данные первого оппонента --- %

\newcommand{\opponentOneFio}{\fixme{Иголкин Александр Алексеевич}}
\newcommand{\opponentOneRegalia}{\fixme{доктор технических наук, доцент}}
\newcommand{\opponentOneJobPlace}{\fixme{федеральное государственное автономное образовательное учреждение высшего образования <<Самарский национальный исследовательский университет имени академика С.П. Королева>>, кафедра <<Автоматические системы энергетических установок>>}}
\newcommand{\opponentOneJobPost}{\fixme{профессор}}

% --- Данные второго оппонента --- %

\newcommand{\opponentTwoFio}{\fixme{Бужинский Валерий Алексеевич}}
\newcommand{\opponentTwoRegalia}{\fixme{доктор физико-математических наук, старший научный сотрудник}}
\newcommand{\opponentTwoJobPlace}{\fixme{акционерное общество <<Центральный научно-исследовательский институт машиностроения>>, отдел <<Динамика ракетно-космической техники>>}}
\newcommand{\opponentTwoJobPost}{\fixme{старший научный сотрудник}}

% --- Данные о ведущей организации --- %

\newcommand{\leadingOrganizationTitle}{\fixme{Федеральное государственное бюджетное образовательное учреждение высшего образования <<Московский государственный технический университет имени Н.Э.Баумана (национальный исследовательский университет)>>}}

% --- Данные о защите --- %

\newcommand{\defenseDate}{\fixme{30 июня 2023~г.~в~11 часов}}

% --- Данные диссертационного совета --- %

\newcommand{\defenseCouncilNumber}{\fixme{Д\,24.2.347.03}}
\newcommand{\defenseCouncilTitle}{\fixme{Новосибирский государственный технический университет}}
\newcommand{\defenseCouncilAddress}{\fixme{г. Новосибирск, проспект К.Маркса, 20}}
\newcommand{\defenseCouncilPhone}{\fixme{+7~(383)~346-06-12}}
\newcommand{\defenseSecretaryFio}{\fixme{Тюрин Андрей Геннадиевич}}
\newcommand{\defenseSecretaryRegalia}{\fixme{канд.~техн.~наук}}

% --- Данные автореферата --- %

\newcommand{\synopsisLibrary}{\fixme{Новосибирского государственного технического университета и на сайте \url{http://www.nstu.ru}}}
\newcommand{\synopsisDate}{\fixme{22 декабря} 2022~года}

% --- Ключевые слова для метаданных PDF диссертации и автореферата --- %

\providecommand{\keywords}%            
{}
