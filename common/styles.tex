% --------------------------------------------------------------------- %
% --- Стили, которые являются общими для диссертации и автореферата --- %
% --------------------------------------------------------------------- %

% --- Шаблон --- %

\DeclareRobustCommand{\fixme}{\textcolor{red}}
\AtBeginDocument{
    \setlength{\parindent}{2.5em} % Абзацный отступ. Должен быть одинаковым по всему тексту
    							  %  и равен пяти знакам (ГОСТ Р 7.0.11-2011, 5.3.7)
}

% --- Форматирование стандартных таблиц --- %

\DeclareCaptionLabelSeparator{tabsep}{\tablabelsep} 

\captionsetup[table]{
    format = \tabformat,                % Формат подписи (plain|hang)
    font = normal,                      % Нормальные размер, цвет, стиль шрифта
    skip = .0pt,                        % Отбивка под подписью
    parskip = .0pt,                     % Отбивка между параграфами подписи
    position = above,                   % Положение подписи
    justification = \tabjust,           % Центровка
    indent = \tabindent,                % Смещение строк после первой
    labelsep = tabsep,                  % Разделитель
    singlelinecheck = \tabsinglecenter, % Не выравнивать по центру, если умещается в одну строку
}

% --- Форматирование таблиц с гибкими настройками --- %

% Стиль
\UseTblrLibrary{varwidth}

\captionsetup[longtblr]{
    format = \tabformat,                % Формат подписи (plain|hang)
    font = normal,                      % Нормальные размер, цвет, стиль шрифта
    skip = .0pt,                        % Отбивка под подписью
    parskip = .0pt,                     % Отбивка между параграфами подписи
    position = above,                   % Положение подписи
    justification = \tabjust,           % Центровка
    indent = \tabindent,                % Смещение строк после первой
    singlelinecheck = \tabsinglecenter, % Не выравнивать по центру, если умещается в одну строку
}

% Настройки заголовка первого блока таблиц
\DefTblrTemplate{caption-sep}{default}{}
\DefTblrTemplate{caption}{default}{
	\UseTblrTemplate{caption-tag}{default}
	\UseTblrTemplate{caption-sep}{default}---\,
	\UseTblrTemplate{caption-text}{default} 
}

% Настройки перенесенного заголовка
\DefTblrTemplate{contfoot-text}{default}{}
\DefTblrTemplate{conthead-text}{default}{(продолжение)}
\DefTblrTemplate{capcont}{default}{
	\UseTblrTemplate{caption}{default}
    \UseTblrTemplate{conthead-text}{default}
}

% --- Форматирование рисунков --- %

\DeclareCaptionLabelSeparator{figsep}{\figlabelsep}

\captionsetup[figure]{
    format = plain,             % Формат подписи (plain|hang)
    font = normal,              % Нормальные размер, цвет, стиль шрифта
    skip = .0pt,                % Отбивка под подписью
    parskip = .0pt,             % Отбивка между параграфами подписи
    position = below,           % Положение подписи
    singlelinecheck = true,     % Выравнивание по центру, если умещается в одну строку
    justification = centerlast, % Центровка
    labelsep = figsep,          % Разделитель
}

% Предкомпиляция рисунков
\ifnumequal{\value{imgprecompile}}{1}{
	\tikzexternalize
}{}

% --- Форматирование подрисунков --- %

% Нумерация
\DeclareCaptionSubType{figure}
\renewcommand\thesubfigure{\asbuk{subfigure}} 

% Выбор размера шрифта в зависимости от типа документа
\ifsynopsis
	\DeclareCaptionFont{norm}{\fontsize{10pt}{11pt}\selectfont}
	\newcommand{\subfigureskip}{2.pt}
\else
	\DeclareCaptionFont{norm}{\fontsize{14pt}{16pt}\selectfont}
	\newcommand{\subfigureskip}{0.pt}
\fi

% Стиль
\captionsetup[subfloat]{
    labelfont = norm,          % Нормальный размер подписей подрисунков
    textfont = norm,           % Нормальные размер, цвет, стиль шрифта
    labelsep = space,          % Разделитель
    labelformat = brace,       % Одна скобка справа от номера
    justification = centering, % Центровка
    singlelinecheck = true,    % Выравнивание по центру, если умещается в одну строку
    skip = \subfigureskip,     % Отбивка над подписью
    parskip = .0pt,            % Отбивка между параграфами подписи
    position = below,          % Положение подписи
}

% --- Настройки ссылок на рисунки, таблицы и др. --- %

% * Команды \cref...format отвечают за форматирование при помощи команды \cref
% * Команды \labelcref...format отвечают за форматирование при помощи команды \labelcref
% Формат
\crefdefaultlabelformat{#2#1#3}

% Уравнение
\crefformat{equation}{(#2#1#3)}                                                 % Одиночная ссылка с приставкой
\labelcrefformat{equation}{(#2#1#3)}                                            % Одиночная ссылка без приставки
\crefrangeformat{equation}{(#3#1#4) \cyrdash~(#5#2#6)}                          % Диапазон ссылок с приставкой
\labelcrefrangeformat{equation}{(#3#1#4) \cyrdash~(#5#2#6)}                     % Диапазон ссылок без приставки
\crefmultiformat{equation}{(#2#1#3)}{ и~(#2#1#3)}{, (#2#1#3)}{ и~(#2#1#3)}      % Перечисление ссылок с приставкой
\labelcrefmultiformat{equation}{(#2#1#3)}{ и~(#2#1#3)}{, (#2#1#3)}{ и~(#2#1#3)} % Перечисление без приставки

% Подуравнение
\crefformat{subequation}{(#2#1#3)}                                                 % Одиночная ссылка с приставкой
\labelcrefformat{subequation}{(#2#1#3)}                                            % Одиночная ссылка без приставки
\crefrangeformat{subequation}{(#3#1#4) \cyrdash~(#5#2#6)}                          % Диапазон ссылок с приставкой
\labelcrefrangeformat{subequation}{(#3#1#4) \cyrdash~(#5#2#6)}                     % Диапазон ссылок без приставки
\crefmultiformat{subequation}{(#2#1#3)}{ и~(#2#1#3)}{, (#2#1#3)}{ и~(#2#1#3)} 	   % Перечисление ссылок с приставкой
\labelcrefmultiformat{subequation}{(#2#1#3)}{ и~(#2#1#3)}{, (#2#1#3)}{ и~(#2#1#3)} % Перечисление без приставки

% Глава
\crefformat{chapter}{#2#1#3}                                           % Одиночная ссылка с приставкой
\labelcrefformat{chapter}{#2#1#3}                                      % Одиночная ссылка без приставки
\crefrangeformat{chapter}{#3#1#4 \cyrdash~#5#2#6}                      % Диапазон ссылок с приставкой
\labelcrefrangeformat{chapter}{#3#1#4 \cyrdash~#5#2#6}                 % Диапазон ссылок без приставки
\crefmultiformat{chapter}{#2#1#3}{ и~#2#1#3}{, #2#1#3}{ и~#2#1#3}      % Перечисление ссылок с приставкой
\labelcrefmultiformat{chapter}{#2#1#3}{ и~#2#1#3}{, #2#1#3}{ и~#2#1#3} % Перечисление без приставки

% Параграф
\crefformat{section}{#2#1#3}                                           % Одиночная ссылка с приставкой
\labelcrefformat{section}{#2#1#3}                                      % Одиночная ссылка без приставки
\crefrangeformat{section}{#3#1#4 \cyrdash~#5#2#6}                      % Диапазон ссылок с приставкой
\labelcrefrangeformat{section}{#3#1#4 \cyrdash~#5#2#6}                 % Диапазон ссылок без приставки
\crefmultiformat{section}{#2#1#3}{ и~#2#1#3}{, #2#1#3}{ и~#2#1#3}      % Перечисление ссылок с приставкой
\labelcrefmultiformat{section}{#2#1#3}{ и~#2#1#3}{, #2#1#3}{ и~#2#1#3} % Перечисление без приставки

% Приложение
\crefformat{appendix}{#2#1#3}                                           % Одиночная ссылка с приставкой
\labelcrefformat{appendix}{#2#1#3}                                      % Одиночная ссылка без приставки
\crefrangeformat{appendix}{#3#1#4 \cyrdash~#5#2#6}                      % Диапазон ссылок с приставкой
\labelcrefrangeformat{appendix}{#3#1#4 \cyrdash~#5#2#6}                 % Диапазон ссылок без приставки
\crefmultiformat{appendix}{#2#1#3}{ и~#2#1#3}{, #2#1#3}{ и~#2#1#3}      % Перечисление ссылок с приставкой
\labelcrefmultiformat{appendix}{#2#1#3}{ и~#2#1#3}{, #2#1#3}{ и~#2#1#3} % Перечисление без приставки

% Рисунок
\crefformat{figure}{#2#1#3}                                           % Одиночная ссылка с приставкой
\labelcrefformat{figure}{#2#1#3}                                      % Одиночная ссылка без приставки
\crefrangeformat{figure}{#3#1#4 \cyrdash~#5#2#6}                      % Диапазон ссылок с приставкой
\labelcrefrangeformat{figure}{#3#1#4 \cyrdash~#5#2#6}                 % Диапазон ссылок без приставки
\crefmultiformat{figure}{#2#1#3}{ и~#2#1#3}{, #2#1#3}{ и~#2#1#3}      % Перечисление ссылок с приставкой
\labelcrefmultiformat{figure}{#2#1#3}{ и~#2#1#3}{, #2#1#3}{ и~#2#1#3} % Перечисление без приставки

% Таблица
\crefformat{table}{#2#1#3}                                           % Одиночная ссылка с приставкой
\labelcrefformat{table}{#2#1#3}                                      % Одиночная ссылка без приставки
\crefrangeformat{table}{#3#1#4 \cyrdash~#5#2#6}                      % Диапазон ссылок с приставкой
\labelcrefrangeformat{table}{#3#1#4 \cyrdash~#5#2#6}                 % Диапазон ссылок без приставки
\crefmultiformat{table}{#2#1#3}{ и~#2#1#3}{, #2#1#3}{ и~#2#1#3}      % Перечисление ссылок с приставкой
\labelcrefmultiformat{table}{#2#1#3}{ и~#2#1#3}{, #2#1#3}{ и~#2#1#3} % Перечисление без приставки

% Листинг
\crefformat{lstlisting}{#2#1#3}                                           % Одиночная ссылка с приставкой
\labelcrefformat{lstlisting}{#2#1#3}                                      % Одиночная ссылка без приставки
\crefrangeformat{lstlisting}{#3#1#4 \cyrdash~#5#2#6}                      % Диапазон ссылок с приставкой
\labelcrefrangeformat{lstlisting}{#3#1#4 \cyrdash~#5#2#6}                 % Диапазон ссылок без приставки
\crefmultiformat{lstlisting}{#2#1#3}{ и~#2#1#3}{, #2#1#3}{ и~#2#1#3}      % Перечисление ссылок с приставкой
\labelcrefmultiformat{lstlisting}{#2#1#3}{ и~#2#1#3}{, #2#1#3}{ и~#2#1#3} % Перечисление без приставки

% Листинг
\crefformat{ListingEnv}{#2#1#3}                                           % Одиночная ссылка с приставкой
\labelcrefformat{ListingEnv}{#2#1#3}                                      % Одиночная ссылка без приставки
\crefrangeformat{ListingEnv}{#3#1#4 \cyrdash~#5#2#6}                      % Диапазон ссылок с приставкой
\labelcrefrangeformat{ListingEnv}{#3#1#4 \cyrdash~#5#2#6}                 % Диапазон ссылок без приставки
\crefmultiformat{ListingEnv}{#2#1#3}{ и~#2#1#3}{, #2#1#3}{ и~#2#1#3}      % Перечисление ссылок с приставкой
\labelcrefmultiformat{ListingEnv}{#2#1#3}{ и~#2#1#3}{, #2#1#3}{ и~#2#1#3} % Перечисление без приставки

% --- Графические стили --- %

\tikzstyle{dim<->} = [thick, latex-latex]              % Двусторонний размер
\tikzstyle{dim->} = [-latex]                           % Односторонний размер
\tikzstyle{axis} = [thick, -latex', color = black]     % Линия оси координат
\tikzstyle{symLine} = [thick, dash dot, color = black] % Линия симметрии
\tikzstyle{vector} = [very thick, -latex]              % Вектор

% --- Настройки гиперссылок --- %

\hypersetup{
    linktocpage = true,                                            % Ссылки с номера страницы в оглавлении, списке таблиц и рисунков
    plainpages  = false,                                           % Требовать представления в арабской форме
    colorlinks  = true,                                            % Ссылки отображаются раскрашенным текстом
    linkcolor   = blue,                                            % Цвет ссылок типа ref, eqref и подобных
    filecolor   = magenta,                                         % Цвет ссылки на файл    
    urlcolor    = cyan,                                            % Цвет ссылки на сайт
    citecolor   = green,                                           % Цвет ссылки на цитату
    pdftitle    = {\thesisTitle},                                  % Заголовок
    pdfauthor   = {\thesisAuthor},                                 % Автор
    pdfsubject  = {\thesisSpecialtyNumber\ \thesisSpecialtyTitle}, % Тема
    pdfkeywords = {\keywords},                                     % Ключевые слова
    pdflang     = {ru},
}

% Подключение настроек черновика
\ifnumequal{\value{draft}}{1}{
    \hypersetup{
        draft,
    }
}{}

% --- Списки --- %

% Используем короткое тире (endash) для ненумерованных списков
% (ГОСТ 2.105-95, пункт 4.1.7, требует дефиса, но так лучше смотрится)
\renewcommand{\labelitemi}{\normalfont\bfseries{--}}

% Перечисление строчными буквами русского алфавита (ГОСТ 2.105-95, 4.1.7)
\makeatletter
\AddEnumerateCounter{\asbuk}{\russian@alph}{щ} % Управляем списками/перечислениями через пакет enumitem, а он 'не знает' про asbuk, потому 'учим' его
\makeatother
%\renewcommand{\theenumi}{\asbuk{enumi}}     % Первый уровень нумерации
%\renewcommand{\labelenumi}{\theenumi)}      % Первый уровень нумерации
\renewcommand{\theenumii}{\asbuk{enumii}}    % Второй уровень нумерации
\renewcommand{\labelenumii}{\theenumii)}     % Второй уровень нумерации
\renewcommand{\theenumiii}{\arabic{enumiii}} % Третий уровень нумерации
\renewcommand{\labelenumiii}{\theenumiii)}   % Третий уровень нумерации

\setlist{nosep,%                             % Единый стиль для всех списков (пакет enumitem), без дополнительных интервалов.
    labelindent=\parindent,leftmargin=*%     % Каждый пункт, подпункт и перечисление записывают с абзацного отступа (ГОСТ 2.105-95, 4.1.8)
}

% --- Правильная нумерация приложений, рисунков и формул --- %

\makeatletter
\if@uni@ode
  \def\russian@Alph#1{\ifcase#1\or
    А\or Б\or В\or Г\or Д\or Е\or Ж\or
    И\or К\or Л\or М\or Н\or
    П\or Р\or С\or Т\or У\or Ф\or Х\or
    Ц\or Ш\or Щ\or Э\or Ю\or Я\else\@ctrerr\fi}
\else
  \def\russian@Alph#1{\ifcase#1\or
    \CYRA\or\CYRB\or\CYRV\or\CYRG\or\CYRD\or\CYRE\or\CYRZH\or
    \CYRI\or\CYRK\or\CYRL\or\CYRM\or\CYRN\or
    \CYRP\or\CYRR\or\CYRS\or\CYRT\or\CYRU\or\CYRF\or\CYRH\or
    \CYRC\or\CYRSH\or\CYRSHCH\or\CYREREV\or\CYRYU\or
    \CYRYA\else\@ctrerr\fi}
\fi
\if@uni@ode
  \def\russian@alph#1{\ifcase#1\or
    а\or б\or в\or г\or д\or е\or ж\or
    и\or к\or л\or м\or н\or
    п\or р\or с\or т\or у\or ф\or х\or
    ц\or ш\or щ\or э\or ю\or я\else\@ctrerr\fi}
\else
  \def\russian@alph#1{\ifcase#1\or
    \cyra\or\cyrb\or\cyrv\or\cyrg\or\cyrd\or\cyre\or\cyrzh\or
    \cyri\or\cyrk\or\cyrl\or\cyrm\or\cyrn\or
    \cyrp\or\cyrr\or\cyrs\or\cyrt\or\cyru\or\cyrf\or\cyrh\or
    \cyrc\or\cyrsh\or\cyrshch\or\cyrerev\or\cyryu\or
    \cyrya\else\@ctrerr\fi}
\fi
\makeatother

% --- Выбор правильной формы существительного при числительном --- %

\makeatletter
\def\formtotal#1#2#3#4#5{%
    \newcount\@c
    \@c\totvalue{#1}\relax
    \newcount\@last
    \newcount\@pnul
    \@last\@c\relax
    \divide\@last 10
    \@pnul\@last\relax
    \divide\@pnul 10
    \multiply\@pnul-10
    \advance\@pnul\@last
    \multiply\@last-10
    \advance\@last\@c
    #2%
    \ifnum\@pnul=1#5\else%
    \ifcase\@last#5\or#3\or#4\or#4\or#4\else#5\fi
    \fi
}
\makeatother

\newcommand{\formbytotal}[5]{\total{#1}~\formtotal{#1}{#2}{#3}{#4}{#5}}