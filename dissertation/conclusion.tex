
\chapter*{Основные выводы и заключение} 
\addcontentsline{toc}{chapter}{Основные выводы и заключение}

В диссертационной работе рассмотрены вопросы развития методов экспериментального модального анализа применительно к задаче коррекции расчетных моделей. Разработаны методики и их программные реализации для осуществления коррекции, освобождения и синтеза конечно-элементных моделей. В ходе проведенных исследований получены следующие основные результаты.


\begin{enumerate}
	\item Разработана методика коррекции конечно-элементных моделей летательных аппаратов по результатам модальных испытаний, основанная на дополнении исходной модели корректирующими конечными элементами.
	\item Исследования сходимости алгоритма и чувствительности методики коррекции расчетных моделей показали, что результаты коррекции устойчивы к погрешностям эксперимента. 
	\item Разработан способ определения собственных частот и форм колебаний свободной конструкции по результатам испытаний этой конструкции с наложенными связями.
	\item Развита методика расчетно-экспериментального модального анализа конструкций по результатам испытаний их составных частей. Разработана программа и обоснованы граничные условия в испытаниях составных частей. Создано программное обеспечение, реализующее полный цикл операций для решения задачи синтеза глобальных расчетных моделей из скорректированных моделей составных частей.
	\item Формирование глобальной матрицы демпфирования конструкций по результатам испытаний их составных частей осуществляется в несколько этапов: по соотношениям между вынужденными монофазными и собственные колебаниями подтверждается диагональный вид матриц демпфирования составных частей в главных координатах, определяются обобщенные характеристики демпфирования. На основании гипотезы Е.\,С.~Сорокина строятся начальные приближения матриц демпфирования составных частей в физической системе координат. Эти матрицы уточняются решением задачи коррекции. Формирование глобальной матрицы осуществляется посредством ассемблирования матриц демпфирования составных частей. 
	\item С целью получения исходных данных для коррекции создан комплекс программ, позволяющий проводить обработку и представление результатов модального анализа непосредственно в процессе испытаний. Разработано программное обеспечение для контроля параметров технического состояния изделий в процессе испытаний.
	\item Эффективность разработанных методик и программного обеспечения подтверждена результатами решения практических задач коррекции расчетных моделей конструкций.
\end{enumerate}

\textbf{Рекомендации и перспективы дальнейшей разработки темы}

Перспективным направлением дальнейших исследований является развитие разработанных вычислительных алгоритмов для ускорения сходимости в задачах коррекции. Это особенно актуально при коррекции расчетных моделей с высокой степенью детализации. Было отмечено, что обобщенное отношение Рэлея отличается от значений собственных чисел, получаемых при решении обобщенной проблемы собственных значений. Причина состоит в численных округлениях при расчете собственных форм колебаний с нулевыми собственными значениями.

Прикладной аспект дальнейшей работы выражается в создании вспомогательных программ, позволяющих учитывать информацию о корректирующих элементах в конечно-элементных пакетах, использованных для создания исходных моделей. Это обеспечит взаимодействие встроенными средствами со скорректированными расчетными моделями, в том числе их применение для решения задач вынужденных колебаний и нелинейной динамики.

Перспективным также является создание программного обеспечения для контроля параметров технического состояния изделий по эксплуатационным вибрациям.

Таким образом, все поставленные в диссертации задачи решены и цель исследования достигнута.

Перспективным направлением дальнейших исследований является развитие разработанных в настоящей диссертации вычислительных алгоритмов для улучшения сходимости в задачах коррекции. Это особенно актуально при коррекции расчетных моделей с высокой степенью детализации, которая выражается в эмпирической оценке: число степеней свободы модели больше или равно одному миллиону. В этом случае было замечено, что обобщенное отношение Рэлея отличается от значений собственных чисел, получаемых при решении обобщенной проблемы собственных значений. Это происходит вследствие численных округлений при нахождении собственных форм колебаний с нулевыми собственными значениями.

Прикладной аспект дальнейшей работы выражается в создании вспомогательных утилит, позволяющих учитывать информацию о корректирующих элементах в конечно-элементных пакетах, использованных для создания исходных моделей. Это обеспечит взаимодействие встроенными средствами с скорректированными расчетными моделями, в том числе их применение для решения задач о вынужденных колебаниях и нелинейной динамики.
