% ----------------------------- %
% --- Стили для презентации --- %
% ----------------------------- %

% --- Шаблон --- %

\DeclareRobustCommand{\fixme}{\textcolor{red}}
\DeclareRobustCommand{\highlight}{\textcolor{blue}}

% --- Путь к изображениям --- %

\graphicspath{{images/presentation}{images/partReview/}{images/partModalAnalysis/}{images/partModelUpdating/}{images/partAprobation/}{images/appendix/}}

% --- Тема оформления --- %

\usetheme{Warsaw}

% --- Макет страниц --- %

\setbeamersize{text margin left = 0.5cm, text margin right = 0.5cm} % Размер полей слайдов

% --- Настройки навигации --- %

\beamertemplatenavigationsymbolsempty % Отключение стрелок

% --- Форматирование текста --- %

% Выравнивание по ширине
\apptocmd{\frame}{}{\justifying}{} 
\apptocmd{\itemize}{\justifying}{}{}
\apptocmd{\enumerate}{\justifying}{}{}

% Настройка переносов
\hyphenpenalty=10000 % Запрет переносов
\righthyphenmin=2    % Перенос двух последний букв
\sloppy              % Выравнивание текста по ширине

% --- Параметры шрифтов --- %

% Размеры шрифтов
\setbeamerfont{institute}{size = \small}
\setbeamerfont{author}{size = \large}
\setbeamerfont{title}{size = \large}
\setbeamerfont{subtitle}{size = \normalsize}
\setbeamerfont{date}{size = \normalsize}
\setbeamerfont{bibliography item}{size = \small}
\setbeamerfont{bibliography entry author}{size = \small}
\setbeamerfont{bibliography entry title}{size = \small}
\setbeamerfont{bibliography entry location}{size = \small}
\setbeamerfont{bibliography entry note}{size = \small}

% Шрифты с засечками для формул
\usefonttheme[onlymath]{serif} 

% --- Цвета структурных элементов --- %

\setbeamercolor{bibliography item}{fg = black}
\setbeamercolor{bibliography entry author}{fg = black}
\setbeamercolor{bibliography entry title}{fg = black}
\setbeamercolor{bibliography entry location}{fg = black}
\setbeamercolor{bibliography entry note}{fg = black}

% --- Рисунки --- %

\fboxsep=0.5mm                        % Величина смещения рамки
\fboxrule=1pt                         % Толщина линии рамки
\setbeamertemplate{caption}[numbered] % Включение нумерации
\setlength\abovecaptionskip{0pt}	  % Отступ до подписки рисунка

% --- Настройки библиографии --- % 

\setbeamertemplate{bibliography item}{\insertbiblabel} % Нумерация списка статей

% --- Настройки гиперссылок --- %

\hypersetup{
    unicode = true, % Не латинские символы в закладках
}

% --- Настройка списков --- %

\makeatletter
\newcommand*{\rom}[1]{\expandafter\@slowromancap\romannumeral#1@}
\makeatother

\newcommand{\itemi}{\item[\checkmark]}

% --- Настройка сносок --- %

\renewcommand{\thefootnote}{[\arabic{footnote}]}

% --- Разметка верхнего колонтитула --- %

\setbeamertemplate{headline}{
	\usebeamerfont{headline}%
    \begin{beamercolorbox}[wd = \paperwidth, colsep = 1.5pt]{lower separation line head}
    \end{beamercolorbox}
}
\setbeamerfont{frametitle}{size=\fontsize{13}{15}\selectfont}


% --- Разметка нижнего колонтитула (основное содержание) --- %

\setbeamertemplate{footline}
{
  \hfill%
  \usebeamercolor[fg]{page number in head/foot}%
  \usebeamerfont{page number in head/foot}%
  \insertframenumber\kern1em\vskip5pt%
}
\setbeamerfont{page number in head/foot}{size={\fontsize{10}{12}}}
\newcommand{\beginSkip}{\vspace{1em}}

% --- Разметка нижнего колонтитула (приложение) --- %

\makeatletter
\g@addto@macro\appendix{
    \setbeamertemplate{footline}{}
}
\makeatother

% --- Таблицы --- %

\newenvironment{comtblr}[1]
{
	\begin{tblr}
	{
		colspec = {|X[m, t]|X[m, t]|}, 
		width   = \textwidth,
		hlines,
		column{1} = {#1}
	}
	\SetCell[]{c} \textbf{Замечания} & \SetCell[]{c} \textbf{Ответы на замечания} \\ \hline
}
{	
	\end{tblr}
	\vskip0pt plus 1filll
} 	




