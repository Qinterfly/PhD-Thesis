% ----------------------------- %
% --- Стили для презентации --- %
% ----------------------------- %

% --- Шаблон --- %

\DeclareRobustCommand{\fixme}{\textcolor{red}}

% --- Путь к изображениям --- %

\graphicspath{{images/presentation}{images/partReview/}{images/partModalAnalysis/}{images/partModelUpdating/}{images/partAprobation/}{images/appendix/}}

% --- Тема оформления --- %

\usetheme{Warsaw}

% --- Макет страниц --- %

\setbeamersize{text margin left = 1cm, text margin right = 1cm} % Размер полей слайдов

% --- Настройки навигации --- %

\beamertemplatenavigationsymbolsempty % Отключение стрелок

% --- Форматирование текста --- %

% Выравнивание по ширине
\apptocmd{\frame}{}{\justifying}{} 
\apptocmd{\itemize}{\justifying}{}{}
\apptocmd{\enumerate}{\justifying}{}{}

% Настройка переносов
\hyphenpenalty=10000 % Запрет переносов
\righthyphenmin=2    % Перенос двух последний букв
\sloppy              % Выравнивание текста по ширине

% --- Параметры шрифтов --- %

% Размеры шрифтов
\setbeamerfont{institute}{size = \small}
\setbeamerfont{author}{size = \large}
\setbeamerfont{title}{size = \large}
\setbeamerfont{subtitle}{size = \normalsize}
\setbeamerfont{date}{size = \normalsize}
\setbeamerfont{bibliography item}{size = \small}
\setbeamerfont{bibliography entry author}{size = \small}
\setbeamerfont{bibliography entry title}{size = \small}
\setbeamerfont{bibliography entry location}{size = \small}
\setbeamerfont{bibliography entry note}{size = \small}

% Шрифты с засечками для формул
\usefonttheme[onlymath]{serif} 

% --- Цвета структурных элементов --- %

\setbeamercolor{bibliography item}{fg = black}
\setbeamercolor{bibliography entry author}{fg = black}
\setbeamercolor{bibliography entry title}{fg = black}
\setbeamercolor{bibliography entry location}{fg = black}
\setbeamercolor{bibliography entry note}{fg = black}

% --- Настройки библиографии --- % 

\setbeamertemplate{bibliography item}{\insertbiblabel} % Нумерация списка статей

% --- Настройки гиперссылок --- %

\hypersetup{
    unicode = true, % Не латинские символы в закладках
}

% --- Настройка списков --- %

\makeatletter
\newcommand*{\rom}[1]{\expandafter\@slowromancap\romannumeral#1@}
\makeatother

\newcommand{\itemi}{\item[\checkmark]}

% --- Разметка нижнего колонтитула --- %

\setbeamertemplate{footline}{
    \leavevmode%
    \hbox{%
    	% Автор
        \begin{beamercolorbox}[wd=0.28\paperwidth, ht=2.25ex, dp=1ex, center]{author in head/foot}%
            \usebeamerfont{author in head/foot}\thesisAuthor
        \end{beamercolorbox}%
        % Тема и страницы
        \begin{beamercolorbox}[wd=0.72\paperwidth, ht=2.25ex, dp=1ex, right]{title in head/foot}%
            \thesisTitleShort\hfill\insertframenumber{}\,/\,\inserttotalframenumber\hspace*{2ex}
        \end{beamercolorbox}}%
    \vskip0pt%
}

% --- Разметка верхнего колонтитула --- %

\setbeamertemplate{headline}{
    \begin{beamercolorbox}[wd=\paperwidth, colsep=1.5pt]{lower separation line head}
    \end{beamercolorbox}
} 

