% -------------------------------- %
% --- Задание новых переменных --- %
% -------------------------------- %

% --- Общая характеристика работы --- %

\newcommand{\actualityTXT}{Актуальность темы.}
\newcommand{\progressTXT}{Степень разработанности темы.}
\newcommand{\aimTXT}{Целью}
\newcommand{\tasksTXT}{задачи}
\newcommand{\noveltyTXT}{Научная новизна:}
\newcommand{\influenceTXT}{Практическая значимость}
\newcommand{\methodsTXT}{Методология и методы исследования.}
\newcommand{\defpositionsTXT}{Основные положения, выносимые на~защиту:}
\newcommand{\reliabilityTXT}{Достоверность}
\newcommand{\probationTXT}{Апробация работы.}
\newcommand{\contributionTXT}{Личный вклад.}
\newcommand{\publicationsTXT}{Публикации.}

% --- Заголовки библиографии --- %

% Для автореферата
\newcommand{\bibtitleauthor}{Публикации автора по теме диссертации}

% Для стиля библиографии `\insertbiblioauthorgrouped`
\newcommand{\bibtitleauthorvak}{В изданиях из списка ВАК РФ}
\newcommand{\bibtitleauthorscopus}{В изданиях, входящих в международную базу цитирования Scopus}
\newcommand{\bibtitleauthorwos}{В изданиях, входящих в международную базу цитирования Web of Science}
\newcommand{\bibtitleauthorother}{В прочих изданиях}
\newcommand{\bibtitleauthorconf}{В сборниках трудов конференций}
\newcommand{\bibtitleauthorpatent}{Зарегистрированные патенты}
\newcommand{\bibtitleauthorprogram}{Зарегистрированные программы для ЭВМ}

% Для стиля библиографии `\insertbiblioauthorimportant`
\newcommand{\bibtitleauthorimportant}{Наиболее значимые \protect\MakeLowercase\bibtitleauthor}

% Для списка литературы в диссертации и списка чужих работ в автореферате
\newcommand{\bibtitlefull}{Список литературы} % (ГОСТ Р 7.0.11-2011, 4)