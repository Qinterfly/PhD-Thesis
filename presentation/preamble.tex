
% Титульный слайд
\begin{frame}[noframenumbering, plain]
    \setcounter{framenumber}{1}
    \maketitle
\end{frame}

\section{Введение}

\begin{frame}{Актуальность и цель диссертационной работы}
	\textbf{\underline{Цель диссертационной работы}}: разработка методики коррекции расчетных моделей летательных аппаратов по результатам модальных испытаний.
	\vfill
	\begin{columns}
		\hfill
		\begin{column}{0.40\textwidth}
			\centering
			\includegraphics[width = \textwidth]{example-ka-62} \\ \vspace{0.5em}
			\includegraphics[width = \textwidth]{example-s-70}
		\end{column}
		\begin{column}{0.57\textwidth}
			\centering
			\includegraphics[width = \textwidth]{example-su-30}	
		\end{column}
		\hfill
	\end{columns}	
\end{frame}

\begin{frame}{Задачи исследования}
	\begin{enumerate}
		\item Разработать методику коррекции расчетных динамических моделей ЛА по экспериментально определенным модальным характеристикам.
		\item Оценить сходимость и чувствительность методики коррекции к погрешностям в результатах модальных испытаний.
		\item Создать алгоритмы и реализующие их программы для обработки и представления результатов экспериментального модального анализа в процессе испытаний.
		\item Изучить методы операционного модального анализа. Реализовать численные алгоритмы для определения модальных характеристик ЛА по результатам акустических и летных испытаний.
		\item Изучить методы вибродиагностики конструкций. Создать алгоритмы и реализующее их программное обеспечение для контроля конструктивно-производственных дефектов в ЛА в процессе модальных испытаний.
		\item Внедрить разработанные в диссертационной работе методики в практику  модальных испытаний ЛА. Использовать методику коррекции для уточнения расчетных динамических моделей.
	\end{enumerate}
\end{frame}

\begin{frame}{Обзор публикаций по теме исследования}
	\begin{itemize}
		\item \textbf{Стохастические методы коррекции}: \\ Beck~J.\,L., Katafygiotis~L.\,S., Boulkaibet~I., Vanik~M.\,W., Goller~B., Schueller~G.\,I., Au~S.\,K., Marwala~T., Yuen~K.\,V., Worden~K., Hensman~J.\,J., Cheung~S.\,H., Mthembu~L., Yan~W.\,J.
		\item \textbf{Детерминированные методы коррекции}: \\ Bakir~P.\,G., Friswell~M.\,I., Baruch~M., Mottershead~J.\,E., Ewins~D.\,J., Berman~E.\,G., Allen~M.\,S., Link~M., Park~D.\,C., Caesar~B., Min~C.\,H., Sipple~J.\,D., Gupta~A.
		\item \textbf{Методы регуляризации}: \\ Ahmadian~H., Fregolent~A., Natke~H.\,G., Visser~W.\,J., Titurus~B., Imregun~M., D'Ambrogio~W., Gladwell~G.\,M.\,L., Ismail~F., Hansen~P.\,C., Bartilson~D.\,T., Smyth~A.\,W.
		\item \textbf{Теоретические и практические аспекты методов модальных испытаний}: \\ Резник~А.\,Л., Смыслов~В.\,И., Микишев~Г.\,Н., Рабинович~Б.\,И., Бернс~В.\,А., Dat~R., Clerc~D., Kennedy~C.\,C., Pancu~C.\,D.\,P., Heylen~W., Lammens~S., Sas~P.
	\end{itemize}
\end{frame}