% -------------------------------- %
% --- Задание новых переменных --- %
% -------------------------------- %

% --- Общая характеристика работы --- %

\newcommand{\actualityTXT}{Актуальность темы исследования}
\newcommand{\progressTXT}{Степень разработанности темы исследования}
\newcommand{\aimTXT}{Целью диссертационный работы}
\newcommand{\tasksTXT}{Задачи исследования}
\newcommand{\noveltyTXT}{Научная новизна}
\newcommand{\influenceTXT}{Теоретическая и практическая значимость}
\newcommand{\methodsTXT}{Методология и методы исследования}
\newcommand{\defpositionsTXT}{Положения, выносимые на~защиту}
\newcommand{\reliabilityTXT}{Степень достоверности и апробация результатов работы}
\newcommand{\probationTXT}{Апробация работы.}
\newcommand{\contributionTXT}{Личный вклад автора}
\newcommand{\pasportTXT}{Соответствие паспорту заявленной специальности}
\newcommand{\publicationsTXT}{Публикации}

% --- Заголовки библиографии --- %

% Для автореферата
\newcommand{\bibtitleauthor}{Публикации автора по теме диссертации}

% Для стиля библиографии `\insertbiblioauthorgrouped`
\newcommand{\bibtitleauthorvak}{В изданиях из списка ВАК РФ}
\newcommand{\bibtitleauthorscopus}{В изданиях, входящих в международную базу цитирования Scopus}
\newcommand{\bibtitleauthorwos}{В изданиях, входящих в международную базу цитирования Web of Science}
\newcommand{\bibtitleauthorother}{В прочих изданиях}
\newcommand{\bibtitleauthorconf}{В сборниках трудов конференций}
\newcommand{\bibtitleauthorpatent}{Зарегистрированные патенты}
\newcommand{\bibtitleauthorprogram}{Зарегистрированные программы для ЭВМ}

% Для стиля библиографии `\insertbiblioauthorimportant`
\newcommand{\bibtitleauthorimportant}{Наиболее значимые \protect\MakeLowercase\bibtitleauthor}

% Для списка литературы в диссертации и списка чужих работ в автореферате
\newcommand{\bibtitlefull}{Список литературы} % (ГОСТ Р 7.0.11-2011, 4)

% --- Формулы и их сокращения --- %

\newcommand{\rarr}{$\rightarrow$}             % Стрелка вправо
\newcommand{\mat}[1]{\mathbf{#1}}             % Жирные матричные символы
\newcommand{\set}[1]{\mathbb{#1}}             % Множество
\newcommand{\trans}[1]{#1 ^ \mathsf{T}}       % Транспонирование
\newcommand{\rbrackets}[1]{\left( #1 \right)} % Круглые скобки
\newcommand{\sbrackets}[1]{\left[ #1 \right]} % Квадратные скобки
\newcommand{\cbrackets}[1]{\{ #1 \}}		  % Фигурные скобки
\newcommand{\abs}[1]{\left| #1 \right|}       % Модуль числа
\newcommand{\imag}{\operatorname{Im}}         % Мнимая часть числа
\newcommand{\real}{\operatorname{Re}}         % Действительная часть числа
\newcommand{\internal}[1]{#1 ^ {\text{int}}}  % Обозначение внутреннего элемента
\newcommand{\external}[1]{#1 ^ {\text{ext}}}  % Обозначение внешнего элемента
% Частная производная n-го порядка по одному аргументу
\newcommand{\partDer}[3]{ 
\ifthenelse{ \equal{#3}{1} \OR \equal{#3}{} }
	{ \frac{\partial #1}{\partial #2} }
	{ \frac{\partial^{#3} #1}{\partial #2^{#3}} }
}

% Названия
\newcommand{\name}[1]{\texttt{#1}} 

% Рисунки
\newcommand{\figref}[1]{(рисунок~\ref{#1})} 

% Таблицы
\newcommand{\tabref}[1]{(таблица~\ref{#1})}

% Приложение
\newcommand{\appref}[1]{(приложение~\ref{#1})}

% --- Начертание математических выражений --- %

% Русская традиция начертания математических знаков
\renewcommand{\le}{\ensuremath{\leqslant}}
\renewcommand{\leq}{\ensuremath{\leqslant}}
\renewcommand{\ge}{\ensuremath{\geqslant}}
\renewcommand{\geq}{\ensuremath{\geqslant}}
\renewcommand{\emptyset}{\varnothing}

% Русская традиция начертания математических функций
\renewcommand{\tan}{\operatorname{tg}}
\renewcommand{\cot}{\operatorname{ctg}}
\renewcommand{\csc}{\operatorname{cosec}}

% Русская традиция начертания греческих букв (греческие буквы вертикальные, через пакет upgreek)
\renewcommand{\epsilon}{\ensuremath{\upvarepsilon}}
\renewcommand{\phi}{\ensuremath{\upvarphi}}
%\renewcommand{\kappa}{\ensuremath{\varkappa}}
\renewcommand{\alpha}{\upalpha}
\renewcommand{\beta}{\upbeta}
\renewcommand{\gamma}{\upgamma}
\renewcommand{\delta}{\updelta}
\renewcommand{\varepsilon}{\upvarepsilon}
\renewcommand{\zeta}{\upzeta}
\renewcommand{\eta}{\upeta}
\renewcommand{\theta}{\uptheta}
\renewcommand{\vartheta}{\upvartheta}
\renewcommand{\iota}{\upiota}
\renewcommand{\kappa}{\upkappa}
\renewcommand{\lambda}{\uplambda}
\renewcommand{\mu}{\upmu}
\renewcommand{\nu}{\upnu}
\renewcommand{\xi}{\upxi}
\renewcommand{\pi}{\uppi}
\renewcommand{\varpi}{\upvarpi}
\renewcommand{\rho}{\uprho}
%\renewcommand{\varrho}{\upvarrho}
\renewcommand{\sigma}{\upsigma}
%\renewcommand{\varsigma}{\upvarsigma}
\renewcommand{\tau}{\uptau}
\renewcommand{\upsilon}{\upupsilon}
\renewcommand{\varphi}{\upvarphi}
\renewcommand{\chi}{\upchi}
\renewcommand{\psi}{\uppsi}
\renewcommand{\omega}{\upomega}
