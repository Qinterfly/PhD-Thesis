
\par\null\par

{\actuality} 

Решение проблемы безопасной и эффективной эксплуатации авиационной и космической техники начинается на этапе проектирования. Для этой цели разрабатываются различные расчетные модели летательных аппаратов (ЛА). Так, например, расчетные динамические модели используются для обеспечения аэроупругой устойчивости самолетов и управляемости космических аппаратов, определении реакции ЛА на динамическое воздействие. Расчетные модели, построенные по технической документации изделий, позволяют сделать первоначальную оценку динамических характеристик ЛА. Однако такие модели содержат неизбежные погрешности моделирования, обусловленные дискретизацией модели, неточностью задания свойств материалов, геометрических характеристик и граничных условий. Невозможность в полной мере учесть в расчетах особенности реальной конструкции приводит к необходимости экспериментального определения модальных параметров ЛА с последующей коррекцией расчетных моделей, поэтому разработка методов коррекции моделей по результатам модальных испытаний является актуальной задачей.

Целью модальных испытаний ЛА является определение характеристик собственных тонов (мод) колебаний конструкций (собственных частот и форм, обобщенных масс и демпфирования). Они проводятся на всех этапах создания ЛА. Испытаниям подвергаются динамически подобные модели ЛА, опытные и серийные образцы авиационной и космической техники. Этап экспериментальных исследований динамических характеристик предполагает испытания не только ЛА в целом, но и их составных частей. Скорректированные по результатам испытаний расчетные модели позволяют повысить эффективность работ по доводке изделий исходя из требований их безопасной и эффективной эксплуатации. 

{\progress}

Известные методы коррекции могут быть разделены на две категории: стохастические и детерминированные. В основе стохастических методов лежит представление о том, что экспериментальные данные являются случайными и содержат неизбежные ошибки измерения, обусловленные как объективными, так и субъективными факторами. В зависимости от типов ошибок измерения в работах Beck~J.\,L., Katafygiotis~L.\,S., Boulkaibet~I., Vanik~M.\,W., Goller~B., Schueller~G.\,I., Au~S.\,K., Marwala~T., Yuen~K.\,V., Worden~K., Hensman~J.\,J., Cheung~S.\,H., Mthembu~L., Yan~W.\,J. и др. были разработаны различные методы коррекции. Детерминированные методы коррекции обычно сводятся к итерационной процедуре минимизации целевой функции, равной сумме квадратов разностей между измеренными в эксперименте данными и соответствующими данными, полученными с помощью расчетной модели (Bakir~P.\,G., Friswell~M.\,I., Baruch~M., Mottershead~J.\,E., Ewins~D.\,J., Berman~E.\,G., Allen~M.\,S., Link~M., Park~D.\,C., Caesar~B., Min~C.\,H., Sipple~J.\,D., Gupta~A. и др.).

Решению важных практических задач применения модальных испытаний для целей коррекции расчетных динамических моделей авиационных конструкций и изделий ракетно-космической техники посвящены работы Смыслова~В.\,И., Кузнецова~О.\,А., Межина~В.\,С., Обухова~В.\,В., Бобылева~С.\,С. и Авершьевой~А.\,В. 

Практическая реализация методов коррекции нередко приводит к тому, что результирующая система уравнений оказывается плохо обусловленной. Для борьбы с этой проблемой существуют техники регуляризации, наиболее часто используемые исследователями: Ahmadian~H., Fregolent~A., Natke~H.\,G., Visser~W.\,J., Titurus~B., Imregun~M., D'Ambrogio~W., Gladwell~G.\,M.\,L., Ismail~F., Hansen~P.\,C., Bartilson~D.\,T., Smyth~A.\,W.

Теоретическое обоснование методов модальных испытаний и вопросы их практического применения изложены, например, в работах Резника~А.\,Л., Смыслова~В.\,И., Микишева~Г.\,Н., Рабиновича~Б.\,И., Бернса~В.\,А., Dat~R., Clerc~D., Kennedy~C.\,C., Pancu~C.\,D.\,P., Heylen~W., Lammens~S., Sas~P. и др.

По результатам анализа публикаций отмечено, что известные методы коррекции расчетных моделей не являются универсальными и не учитывают в полной мере особенностей конструкций ЛА и модальных испытаний авиационной и космической техники. В основу разработанной в диссертации методики положен детерминированный подход. Целевой функцией является сумма квадратов разностей между целевыми (найденными экспериментально) и расчетными собственными частотами ЛА. Методика не изменяет портреты и симметрию матриц этих моделей.

{\aim} является разработка методики коррекции расчетных моделей летательных аппаратов по результатам модальных испытаний. 

{\tasks}
\begin{enumerate}[beginpenalty = 10000] 
	\item Разработать методику коррекции расчетных динамических моделей ЛА по экспериментально определенным модальным характеристикам.
	\item Оценить сходимость и чувствительность методики коррекции к погрешностям в результатах модальных испытаний.
	\item Создать алгоритмы и реализующие их программы для обработки и представления результатов экспериментального модального анализа в процессе испытаний.
	\item Изучить методы операционного модального анализа. Реализовать численные алгоритмы для определения модальных характеристик ЛА по результатам акустических и летных испытаний.
	\item Изучить методы вибродиагностики конструкций. Создать алгоритмы и реализующие их программное обеспечение для контроля конструктивно-производственных дефектов в конструкциях ЛА в процессе модальных испытаний.
	\item Внедрить разработанные в диссертационной работе методики в практику  модальных испытаний ЛА. Использовать методику коррекции для уточнения расчетных динамических моделей.
\end{enumerate}

{\novelty}
\begin{enumerate}[beginpenalty = 10000] 
	\item Разработана новая методика коррекции конечно-элементных моделей ЛА, заключающаяся в добавлении корректирующих конечных элементов, параметры которых определяются по результатам модальных испытаний.
	\item Создан способ определения частот и форм собственных колебаний свободной конструкции по результатам испытаний этой конструкции с наложенными связями.
	\item Обоснована методика формирования глобальной матрицы демпфирования конструкций по результатам испытаний их составных частей.
	\item Развита методика испытаний составных частей ЛА для достоверного построения их матриц жесткости.
\end{enumerate}

{\influence}

Теоретическую значимость представляют:

\begin{enumerate}[beginpenalty = 10000] 
	\item Методика коррекции конечно-элементных моделей ЛА посредством добавления корректирующих элементов, характеристики которых определяются по результатам экспериментального модального анализа.
	\item Способ определения модальных параметров свободной конструкции по результатам испытания этой конструкции с наложенными связями. 
\end{enumerate}

Практическая значимость результатов работы состоит в разработке и развитии методик, позволяющих повысить достоверность расчетных моделей ЛА и, как следствие, обеспечить безопасную и эффективную эксплуатацию авиационной и космической техники. Применение этих методик в совокупности с созданным программным обеспечением позволяет повысить информативность, расширить область использования результатов модального анализа ЛА и снизить объем работ по доводке их конструкций.

Результаты проведенных в диссертации исследований использованы в модальных испытаниях самолётов \mbox{Су-30}, \mbox{Су-34}, \mbox{Як-130}, \mbox{Як-152}, \mbox{МС-21}; в конструкторско-технологической доводке изделий \mbox{Су-57} и \mbox{С-70}, а также при проектировании  гирдеров для модульных секций накопителя ЦКП <<СКИФ>>, о чём имеются акты об их использовании и внедрении~(приложения~\ref{struct:acts-usage} и \ref{struct:acts-implement}).

{\methods}

При построении расчетных моделей использовался метод конечных элементов. В разработке методики коррекции расчетных моделей применялись методы оптимизации. Целевые данные для коррекции расчетных моделей получены методами экспериментального модального анализа. В оценке чувствительности методики коррекции расчетных моделей к погрешностям эксперимента использовался метод статистического моделирования.  

{\defpositions}
\begin{enumerate}[beginpenalty = 10000] 
	\item Методика коррекции конечно-элементных моделей ЛА, заключающаяся в добавлении корректирующих конечных элементов, параметры которых определяются по результатам модальных испытаний.
	\item Способ определения частот и форм собственных колебаний свободной конструкции по результатам испытаний этой конструкции с наложенными связями.
	\item Методика испытаний составных частей ЛА для достоверного построения их матриц жесткости.
	\item Алгоритмы и реализующие их программы для обработки и представления результатов экспериментального модального анализа в процессе испытаний. 
\end{enumerate}

\newpage

{\reliability} 

Достоверность и обоснованность результатов работы определяется применением основных положений механики, анализом погрешностей определяемых параметров, оценкой чувствительности разрабатываемой методики и исследованиями сходимости ее алгоритма. Результаты экспериментальных исследований получены с использованием апробированных методик и современного прецизионного оборудования.

Основные положения и результаты работы докладывались и обсуждались на XXIV Международном симпозиуме <<Динамические и технологические проблемы механики конструкций и сплошных сред>> имени А.~Г.~Горшкова (г.~Москва, 2018), Международной научно-практической конференции <<Решетнёвские чтения>> (г.~Красноярск, XXII (2018), XXIII (2019), XXV (2021), XXVI (2022)), 17-ой Российско-Китайской научно-технической конференции <<Фундаментальные задачи аэродинамики, динамики, прочности и безопасности полетов ЛА>> (г.~Жуковский, 2021); Юбилейной научно-технической конференция, посвящённой 80-летнему юбилею СибНИА (г.~Новосибирск, 2021), школе-семинаре <<Проблемы прочности авиационных конструкций и материалов>> (г.~Новосибирск, 2021 и 2022), 58-ой Международной научной студенческой конференции МНСК (г.~Новосибирск, 2020), 6-ой Международной научно-технической конференции <<Динамика и виброакустика машин>> (г.~Самара, 2022), научно-технической конференции <<Прочность конструкций летательных аппаратов>> (г.~Жуковский, 2018 и 2022). 

{\contribution} заключается в создании методик коррекции, освобождения и синтеза расчетных моделей конструкций; разработке алгоритмов и реализующих их программ; проведении расчетов и участии в экспериментальных исследованиях, анализе их результатов; формулировке выводов.

{\pasport}

Тема и содержание диссертационной работы соответствуют паспорту научной специальности 2.5.14~---~<<Прочность и тепловые режимы летательных аппаратов>> по пункту 1~---~<<Методы определения внешних силовых и тепловых нагрузок, действующих на объекты авиационной, ракетной и космической техники на этапах транспортировки, применения и эксплуатации>>, пункту 2~---~<<Обеспечение прочности объектов авиационной, ракетной и космической техники на основе современных аналитических и численных методов, методов натурного и полунатурного моделирования в условиях стационарных и нестационарных внешних воздействий>> и пункту 6~---~<<Организация, экономика и оптимизация процессов обеспечения прочности, термопрочности и тепловых режимов летательных аппаратов>>.

{\publications} 

Основные результаты по теме диссертации изложены в~27~печатных изданиях, 2~из которых изданы в журналах, рекомендованных ВАК, 2~---~в~периодических научных журналах, индексируемых \textit{Web~of~Science} и \textit{Scopus}, 23~---~в прочих изданиях и сборниках трудов международных и всероссийских научно-технических конференций. Зарегистрирован патент на изобретение. Получены~4~свидетельства о государственной регистрации программ для ЭВМ.
