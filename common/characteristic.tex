
{\actuality} 

Решение проблемы безопасной и эффективной эксплуатации авиационной и космической техники начинается на этапе проектирования. Для этой цели разрабатываются различные расчетные модели летательных аппаратов (ЛА). Так, например, динамические расчетные модели используются для обеспечения аэроупругой устойчивости самолетов и управляемости космических аппаратов, определении реакции ЛА на динамическое воздействие. Расчетные модели, построенные по технической документации изделий, позволяют сделать первоначальную оценку динамических характеристик ЛА. 

Целью модальных испытаний ЛА является определение характеристик собственных тонов (мод) колебаний конструкций (собственных частот и форм, обобщенных масс и демпфирования). Они проводятся на всех этапах создания ЛА. Испытаниям подвергаются динамически подобные модели ЛА, опытные и серийные образцы авиационной и космической техники. Этап экспериментальных исследований динамических характеристик предполагает испытания не только ЛА в целом, но и их составных частей. Скорректированные по результатам испытаний расчетные модели позволяют повысить эффективность работ по доводке изделий исходя из требований их безопасной и эффективной эксплуатации. 

Конечно-элементные модели (КЭ-модели) широко используются для проведения статистических и динамических расчетов во многих областях техники. Однако такие модели в ряде случаев содержат неизбежные погрешности моделирования, обусловленные дискретизацией модели, неточностью задания свойств материалов, геометрических характеристик и граничных условий. Невозможность в полной мере учесть в расчетах особенности реальной конструкции приводит к необходимости экспериментального определения модальных параметров ЛА с последующей коррекцией расчетных моделей, поэтому разработка методов коррекции моделей по результатам модальных испытаний является актуальной задачей. 

{\progress}

\fixme{Известные методы коррекции могут быть разделены на две категории: стохастические и детерминированные. В основе стохастических методов лежит представление о том, что экспериментальные данные являются случайными и содержат неизбежные ошибки измерения, обусловленные как объективными, так и субъективными факторами. В зависимости от типов ошибок измерения в работах Huang B., Chen H. A, Bao N., Wang C., Boulkaibet I., Mthembu L., Marwala T., Wang T., He H., Yan W., Chen G.P., Deng Z., Guo Z., Zhang X., Zhai X., Fei C.W., Choy Y.S., Wang J.J., Fang S.E., Zhang Q.H., Ren W.X., Xin Y., Hao H., Li J., Lam H.F., Yang J., Au S.K. и др. были разработаны различные методы коррекции. Детерминированные методы коррекции обычно сводятся к итерационной процедуре минимизации целевой функции, равной сумме квадратов разностей между измеренными в эксперименте данными и соответствующими данными, полученными с помощью расчетной модели (Mottershead J.E., Link M., Friswell M.I., Bakir P.G., Reynders E., Roeck G.D., Min C.H., Hong S., Park S.Y., Park D.C., Hernandez E.M., Bernal D., Chen L., Guo Y., Li L. и др.). 
Практическая реализация методов коррекции нередко приводит к тому, что результирующая система уравнений оказывается плохо обусловленной [30]. Для борьбы с этой проблемой существуют техники регуляризации, наиболее часто используемые исследователями: Hua X.G., Ni Y.Q., Ko J.M., Reumers P., Hoorickx C.V., Schevenels M., Lombaert G., Mantilla-Gaviria I., Leonardi M., Balbastre-Tejedor J.V., Reyes E., Weber B., Paultre P., Proulx J.
Теоретическое обоснование методов модальных испытаний и вопросы их практического применения изложены, например, в работах Резника А. Л., Смыслова В. И., Микишева Г. Н., Рабиновича Б. И., Бернса В. А., Dat R., Clerc D., Kennedy C. C., Pancu C. D. P., Heylen W., Lammens S., Sas P и др. 
По результатам анализа публикаций отмечено, что известные методы коррекции расчетных моделей не являются универсальными и не учитывают в полной мере особенностей конструкций ЛА и модальных испытаний авиационной и космической техники. В основу разработанной в диссертации методики положен детерминированный подход. Целевой функцией является сумма квадратов разностей между целевыми (найденными экспериментально) и расчетными собственными частотами ЛА. Методика не имеет ограничений по числу степеней свободы КЭ-моделей и не изменяет портреты и симметрию матриц этих моделей.}

{\aim} диссертационной работы: разработка методики коррекции расчетных моделей летательных аппаратов по результатам модальных испытаний. 

{\tasks}
\begin{enumerate}[beginpenalty = 10000] 
	\item Разработать методику коррекции расчетных динамических моделей ЛА по экспериментально определенным модальным характеристикам.
	\item Изучить методы экспериментального модального анализа. Создать программную платформу для обработки и представления результатов модального анализа непосредственно в процессе испытаний. 
	\item Изучить методы операционного модального анализа. Разработать программное обеспечение для определения модальных характеристик ЛА по результатам акустических испытаний и полетов в неспокойной атмосфере.
	\item Изучить методы вибродиагностики конструкций. Разработать программное обеспечение для контроля технологических дефектов в конструкциях ЛА в процессе модальных испытаний. 
	\item Оценить сходимость и чувствительность методики коррекции к погрешностям в результатах модальных испытаний. 
	\item Решить практические задачи коррекции расчетных моделей конструкций аэрокосмического и гражданского назначения.
\end{enumerate}

{\novelty}
\begin{enumerate}[beginpenalty = 10000] 
	\item Разработан новый подход к коррекции расчетных динамических моделей путем добавления корректирующих конечных элементов, параметры которых определяются из решения задачи оптимизации по целевым модальным характеристикам.
	\item Развита методика синтеза достоверной расчетной модели ЛА из высокодетализированных моделей составных частей, скорректированных по результатам модальных испытаний.
	\item Получены новые результаты применения методов коррекции, синтеза, идентификации и диагностирования дефектов к конструкциям аэрокосмического и гражданского назначения.
\end{enumerate}

{\influence} состоит в разработке и применении методик для повышения достоверности расчетных моделей ЛА и, как следствие, снижения объема работ по доводке их конструкций, обеспечения безопасности и эффективности эксплуатации. Разработанная программная платформа позволяет повысить информативность и расширить область использования результатов экспериментального и операционного модального анализа ЛА.

Результаты проведенных в диссертации исследований использованы в конструкторско-технологической доводке самолётов Су-30, Як-130, Як-152, МС-21, изделий Су-57 и С-70, а также при проектировании гирдеров для модульных секций накопителя ЦКП <<СКИФ>>, о чём имеются акты внедрения (приложения А, Б и В).

{\methods}

При построении расчетных моделей использовался метод конечных элементов. Исходные данные для коррекции расчетных моделей получены методами экспериментального и операционного модального анализа. В разработке методики коррекции расчетных моделей и оценке чувствительности к погрешностям эксперимента использовались методы вычислительной математики и статистического моделирования. 

{\defpositions}
\begin{enumerate}[beginpenalty = 10000] 
	\item Методика коррекции расчетных динамических моделей путем добавления корректирующих конечных элементов, параметры которых определяются из решения задачи оптимизации по целевым модальным характеристикам.
	\item Методика синтеза достоверной расчетной модели ЛА из высокодетализированных моделей составных частей, скорректированных по результатам модальных испытаний.
	\item Программная платформа для обработки и представления результатов операционного и экспериментального модального анализа, коррекции и синтеза расчетных моделей ЛА.
	\item Результаты исследования сходимости алгоритма и чувствительности методики коррекции расчетных моделей к погрешностям эксперимента. 
	\item Результаты применения методов коррекции, синтеза, идентификации и диагностирования дефектов к конструкциям аэрокосмического и гражданского назначения.
\end{enumerate}

{\reliability} 

\fixme{Достоверность и обоснованность результатов работы определяется применением основных положений механики; анализом погрешностей определяемых параметров; оценкой чувствительности разрабатываемой методики и исследованиями сходимости ее алгоритма. Результаты экспериментальных исследований получены с использованием апробированных методик и современного прецизионного оборудования.}

\fixme{Основные положения и результаты работы докладывались и обсуждались на XXIV Международном симпозиуме <<Динамические и технологические проблемы механики конструкций и сплошных сред>> имени А.~Г.~Горшкова (г.~Москва, 2018), Международной научно-практической конференции <<Решетнёвские чтения>> (г.~Красноярск, XXII (2018), XXIII (2019), XXV (2021), XXVI (2022)); 17-ой Российско-Китайской научно-технической конференции <<Фундаментальные задачи аэродинамики, динамики, прочности и безопасности полетов ЛА>> (г.~Жуковский, 2021), Юбилейной научно-технической конференция, посвящённая 80-летнему юбилею СибНИА (г.~Новосибирск, 2021), школе-семинаре <<Проблемы прочности авиационных конструкций и материалов>> (г.~Новосибирск, 2016, 2017); III Международной научно-технической конференции <<Динамика и виброакустика машин>> (г.~Самара, 2022); научно-технической конференции <<Прочность конструкций летательных аппаратов>> (г.~Жуковский, 2018, 2022).}

{\contribution} \fixme{заключается в разработке методики коррекции расчетных моделей конструкций, проведении расчетных и экспериментальных исследований и анализе их результатов, формулировке выводов.}

{\pasport}

\fixme{2.5.14}

{\publications} 

\fixme{Основные результаты по теме диссертации изложены в~XX~печатных изданиях, X~из которых изданы в журналах, рекомендованных ВАК, X~---~в тезисах докладов.}
