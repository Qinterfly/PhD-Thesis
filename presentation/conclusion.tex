
\section{Научная новизна}

\begin{frame}{Научная новизна}
	\begin{enumerate}
		\item Разработана новая методика коррекции конечно-элементных моделей ЛА, заключающаяся в добавлении корректирующих конечных элементов, параметры которых определяются по результатам модальных испытаний.
		\item Создан способ определения частот и форм собственных колебаний свободной конструкции по результатам испытаний этой конструкции с наложенными связями.
		\item Обоснована методика формирования глобальной матрицы демпфирования конструкций по результатам испытаний их составных частей.
		\item Развита методика испытаний составных частей ЛА для достоверного построения их матриц жесткости.
	\end{enumerate}
\end{frame}

\section{Заключение}

\begin{frame}{Заключение}
	\begin{enumerate}
		\item Разработана методика коррекции конечно-элементных моделей летательных аппаратов по результатам модальных испытаний, основанная на дополнении исходной модели корректирующими конечными элементами.
		\item Исследования сходимости алгоритма и чувствительности методики коррекции расчетных моделей показали, что результаты коррекции устойчивы к погрешностям эксперимента. 
		\item Разработан способ определения собственных частот и форм колебаний свободной конструкции по результатам испытаний этой конструкции с наложенными связями.
		\item Развита методика расчетно-экспериментального модального анализа конструкций по результатам испытаний их составных частей. Разработана программа и обоснованы граничные условия в испытаниях составных частей. Создано программное обеспечение, реализующее полный цикл операций для решения задачи синтеза глобальных расчетных моделей из скорректированных моделей составных частей.
	\end{enumerate}
\end{frame}

\begin{frame}{Заключение}
	\begin{enumerate}
		\setcounter{enumi}{4}
		\item Формирование глобальной матрицы демпфирования конструкций по результатам испытаний их составных частей осуществляется в несколько этапов: по соотношениям между вынужденными монофазными и собственными колебаниями подтверждается диагональный вид матриц демпфирования составных частей в главных координатах, определяются обобщенные характеристики демпфирования. На основании гипотезы Е.\,С.~Сорокина строятся начальные приближения матриц демпфирования составных частей в физической системе координат. Эти матрицы уточняются решением задачи коррекции. Формирование глобальной матрицы осуществляется посредством ассемблирования матриц демпфирования составных частей. 
		\item С целью получения исходных данных для коррекции создан комплекс программ, позволяющий проводить обработку и представление результатов модального анализа непосредственно в процессе испытаний. Разработано программное обеспечение для контроля параметров технического состояния изделий в процессе испытаний.
		\item Эффективность разработанных методик и программного обеспечения подтверждена результатами решения практических задач коррекции расчетных моделей конструкций.
	\end{enumerate}
\end{frame}