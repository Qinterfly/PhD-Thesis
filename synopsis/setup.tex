% ------------------------------------------------- %
% --- Упрощенные настройки шаблона автореферата --- %
% ------------------------------------------------- %

% --- Инициализирование переменных --- %

\newlength{\secskip}
\newlength{\belowcaptskip}
\newcounter{showperssign}
\newcounter{showsecrsign}
\newcounter{showopplead}

% --- Задание интервалов --- %

\setlength{\secskip}{3pt}        % Интервал между заголовками и текстом 
\setlength{\belowcaptskip}{-10pt} % Интервал после подписью и текстом

% --- Список публикаций --- %

\makeatletter
\@ifundefined{c@usefootcite}{
  \newcounter{usefootcite}
  \setcounter{usefootcite}{0} % 0 --- два списка литературы;
                              % 1 --- список публикаций автора + цитирование
                              %       других работ в сносках
}{}
\makeatother

\makeatletter
\@ifundefined{c@bibgrouped}{
  \newcounter{bibgrouped}
  \setcounter{bibgrouped}{0}  % 0 --- единый список работ автора;
                              % 1 --- сгруппированные работы автора
}{}
\makeatother

% --- Область упрощённого управления оформлением --- %

% Управление зазором между подрисуночной подписью и основным текстом
\setlength{\belowcaptionskip}{10pt plus 20pt minus 2pt}

% --- Подпись таблиц --- %

% Смещение строк подписи после первой
\newcommand{\tabindent}{0cm}

% Тип форматирования таблицы
\newcommand{\tabformat}{plain}

% выравнивание по центру подписи, состоящей из одной строки
% true  --- выравнивать
% false --- не выравнивать
\newcommand{\tabsinglecenter}{false}

% выравнивание подписи таблиц
% justified   --- выравнивать как обычный текст
% centering   --- выравнивать по центру
% centerlast  --- выравнивать по центру только последнюю строку
% centerfirst --- выравнивать по центру только первую строку
% raggedleft  --- выравнивать по правому краю
% raggedright --- выравнивать по левому краю
\newcommand{\tabjust}{justified}

% Разделитель записи «Таблица #» и названия таблицы
\newcommand{\tablabelsep}{~\cyrdash\ }

% --- Подпись рисунков --- %

% Разделитель записи «Рисунок #» и названия рисунка
\newcommand{\figlabelsep}{~\cyrdash\ }  % (ГОСТ 2.105, 4.3.1)
                                        % "--- здесь не работает
                                        
% --- Параметры отобржания информации о защите --- %                                       

% Демонстрация подписи диссертанта на автореферате
\setcounter{showperssign}{1}  % 0 --- не показывать;
                              % 1 --- показывать
% Демонстрация подписи учёного секретаря на автореферате
\setcounter{showsecrsign}{1}  % 0 --- не показывать;
                              % 1 --- показывать
% Демонстрация информации об оппонентах и ведущей организации на автореферате
\setcounter{showopplead}{1}   % 0 --- не показывать;
                              % 1 --- показывать

% --- Задание цветов гиперссылок --- %

\definecolor{linkcolor}{rgb}{0.9, 0, 0}
\definecolor{citecolor}{rgb}{0, 0.6, 0}
\definecolor{urlcolor}{rgb}{0, 0, 1}
