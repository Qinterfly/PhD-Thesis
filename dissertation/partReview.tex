\chapter{Проблемы создания расчетных динамических моделей конструкций по результатам модальных испытаний}

\section{Методы коррекции} 

Конечно-элементные модели широко используются для проведения статических и динамических расчетов во многих областях техники. Однако такие модели в ряде случаев содержат неизбежные погрешности моделирования, обусловленные дискретизаций модели, неточностью задания свойств материалов, геометрических характеристик и граничных условий \cite{lib:modelUpdating:Bartilson}. Для устранения погрешностей моделирования были разработаны различные методы коррекции КЭ-моделей применительно к разным конструкциям: топливным бакам ракет-носителей \cite{lib:modelUpdating:Li&Tian}, соплу ракетного двигателя \cite{lib:modelUpdating:Yan&Li}, композитному самолету \cite{lib:modelUpdating:Zhao&Gupta}, архитектурным сооружениям \cite{lib:modelUpdating:Girardi&Padovani}, пластинчатым теплообменникам \cite{lib:modelUpdating:Guo&Wang}. В основе этих методов лежит минимизация разницы между ключевыми характеристиками реальной конструкции и параметрами КЭ-модели. В качестве таких параметров могут выступать, например, частоты собственных колебаний и отклик конструкции на динамическое воздействие \cite{lib:modelUpdating:Petersen&Oiseth}. Кроме того, методы коррекции нередко используются для оценки поврежденностей различных конструкций: дамб \cite{lib:modelUpdating:Bayraktar&Sevim}, мостов \cite{lib:modelUpdating:Cong&Thoi, lib:modelUpdating:Polanco}, контейнеров для хранения радиоактивных отходов \cite{lib:modelUpdating:Eiras}. Обзор подходов к решению проблемы мониторинга поврежденности приведен в \cite{lib:modelUpdating:Simoen}. 

Следует отметить, что погрешность описания механических свойств реальной конструкции посредством метода конечных элементов зачастую обусловлена особенностями моделирования узлов сопряжения, например, болтовых и заклепочных соединений. Однако описание нелинейного поведения таких узлов посредством построения нелинейной КЭ-модели в ряде случаев может оказаться вычислительно затратным. Во избежание этого такие узлы сопряжения могут быть заменены специальными элементами с эквивалентными жесткостными и диссипативными характеристиками. Так, в работах \cite{lib:modelUpdating:Lacayo, lib:modelUpdating:Yuan} болтовые узлы сопряжения двухбалочной модели были заменены гистерезисными элементами Iwan’a, параметры которых были определены методом коррекции. 

Известные методы коррекции могут быть разделены на две категории: стохастические и детерминированные. В основе стохастических методов лежит представление о том, что экспериментальные данные являются случайными и содержат неизбежные ошибки измерения, обусловленные как объективными факторами (температура, влажность, вибрации, оборудование), так и субъективными (опыт проведения экспериментов) \cite{lib:modelUpdating:Huang}. В зависимости от типов ошибок измерения были разработаны различные методы коррекции: метод Монте-Карло симуляции \cite{lib:modelUpdating:Bao, lib:modelUpdating:Boulkaibet}, методы пертурбаций \cite{lib:modelUpdating:Wang&He, lib:modelUpdating:Deng}, методы случайных поверхностей отклика \cite{lib:modelUpdating:Zhai&Fei, lib:modelUpdating:Fang} и Байесовcкие методы \cite{lib:modelUpdating:Xin&Hao, lib:modelUpdating:Lam}. Однако, стохастические методы коррекции на несколько порядков вычислительно затратнее детерминированных методов. Заметим, что вопрос выбора параметров коррекции для стохастических методов не является однозначным \cite{lib:modelUpdating:Yuan&Liang}.

Детерминированные методы коррекции обычно сводятся к итерационной процедуре коррекции параметров КЭ-модели. Среди итерационных методов наиболее популярным и интуитивно понятным является подход, основанный на создании матрицы чувствительности \cite{lib:modelUpdating:Mottershead, lib:modelUpdating:Bakir, lib:modelUpdating:Min, lib:modelUpdating:Hernandez}. В основе этого метода лежит проблема минимизации целевой функции, равной сумме квадратов разностей между измеренными в эксперименте данными и соответствующими данными, полученными с помощью модели. Удобство такого подхода заключается в том, что в целевой функции одновременно могут фигурировать разнородные параметры, такие, как собственные частоты и отклики конструкций на динамическое воздействие. Чтобы уравновесить вклад этих данных в результирующую функцию авторы работы \cite{lib:modelUpdating:Chen&Guo} вводят весовые коэффициенты, значения которых определят методом анализа иерархий.
 
Нередко применение методов коррекции приводит к тому, что результирующая система уравнений является плохо обусловленной \cite{lib:modelUpdating:Hua}. Для борьбы с этой проблемой две техники регуляризации наиболее часто используются исследователями: регуляризация Тихонова \cite{lib:modelUpdating:Li&Law, lib:modelUpdating:Reumers} и сингулярное разложение \cite{lib:modelUpdating:Mantilla-Gaviria}. Особенности применения обоих подходов на примере модели бетонного сооружения показаны в работе \cite{lib:modelUpdating:Weber}.

Известен метод коррекции расчетной динамической модели композитной конструкции, основанный на кластерном анализе. Создается исходная конечно-элементная модель конструкции. Экспериментальная частота и форма собственных колебаний определяются в модальных испытаниях. Строится матрица чувствительности для корректируемых параметров. Алгоритм кластеризации по слоям используется для группировки параметров коррекции. Среди всех кластеров для коррекции выбирается тот, которому соответствует наибольшее значение матрицы чувствительности. Решается проблема собственных значений для расчетной модели. Разности между расчетными и экспериментальными собственными частотами и формами закладываются в целевую функцию, для которой решается задача оптимизации \cite{lib:modelUpdating:CN107357992A}.

Недостатками метода являются: 
\begin{itemize}
	\item ограниченная область применимости (конструкции, изготовленные из слоистых материалов); 
	\item в целевой функции одновременно присутствуют два параметра (собственная частота и собственная форма колебаний), погрешности экспериментального определения которых различаются на порядок; 
	\item характеристики демпфирования колебаний не корректируются и не определяются. 
\end{itemize}

Известен метод коррекции нелинейной конечно-элементной модели конструкции, основанный на создание комплексной матрицы чувствительности. На первом шаге предлагаемого метода создается нелинейная конечно-элементная модель. Затем методом пертурбаций с комплексным шагом изменяются параметры коррекции с целью вычисления матрицы чувствительности нелинейного динамического отклика конструкции. Измеряется отклик реальной конструкции в процессе испытаний, создается целевая функция для нелинейной коррекции КЭ-модели. Наконец, решается задача среднеквадратичной минимизации целевой функции, описывающей отличие расчетного динамического отклика от экспериментального \cite{lib:modelUpdating:CN109885896A}. 

Недостатками метода являются: 
\begin{itemize}
	\item не указана область применимости метода (виды нелинейных конструкций); 
	\item метод не позволяет непосредственное определение резонансных частот колебаний конструкции. 
\end{itemize}

Известен метод коррекции полноразмерной конечно-элементной модели из условий близости параметров модели соответствующим параметрам реальной конструкции. В зависимости от цели исследования предлагаются две вариации метода применительно к лопаткам двигателя. В одной из них параметрами коррекции является массовые плотности лопаток, а в другой~---~упругие свойства материала. По результатам коррекции сопоставляются динамические отклики скорректированной КЭ-модели и реальной конструкции и вычисляются относительные коэффициенты разбалансировки лопаток \cite{lib:modelUpdating:WO2019209410A1}. Область применимости метода ограничена малоразмерными конструкциями, изготовленными из однородного материала. 

Известен метод коррекции расчетных моделей конструкций, основанный на вычислении критериев модального соответствия между экспериментальными и расчетными формами собственных колебаний деформаций. Создается конечно-элементная модель, с использованием которой проводится модальный анализ для определения форм собственный колебаний деформаций конструкции. Эти формы определяются и в модальных испытаниях. Вычисляется критерии модального соответствия между расчетными и экспериментальными формами. В дальнейшем осуществляется выбор форм колебаний по критерию модального соответствия и назначение параметров коррекции по пороговому значению чувствительности. В качестве параметров коррекции выступают как геометрические характеристики конструкции, так и характеристики материалов, из которых она изготовлена \cite{lib:modelUpdating:CN106529055A}. Область применимости метода ограничена достаточно простыми малоразмерными конструкциями.
 
Известен метод мониторинга поврежденности большепролетного моста, основанный на коррекции характеристик его конечно-элементной модели. Мониторинг осуществляется в несколько этапов. Сначала экспериментально определяются частоты собственных колебаний, которые закладываются в уравнения колебаний моста. Затем производится коррекция расчетной модели конструкции Байесовским методом. Величины параметров коррекции определяются по пикам распределения вероятностей. Оценивается разность между текущими параметрами коррекции и параметрами коррекции, соответствующими обучающей выборке. Результирующие значения оцениваются и положения дефектов локализуется \cite{lib:modelUpdating:CN107687872A}. 

Недостатком метода является то, что его реализация связана с большим объемом вычислений. Кроме того, параметры дефекта являются локальными (дифференциальными) характеристиками конструкции, а собственные частоты~---~интегральными. Поэтому однозначность решения задачи мониторинга поврежденности моста требует дополнительного контроля. 

\section{Методы ассемблирования}

Динамический анализ конструкций обычно выполняется в двух областях: временной \cite{lib:coupling:Dong&Shuo, lib:coupling:Gram-Schmidt} и частотной \cite{lib:coupling:Peeters}. В общем случае мы можем утверждать, что любая подконструкция во временной области можем быть связана с любой другой подконструкций во временной области, а любая подконструкция в частотной области может быть связана с любой другой подконструкцией в частотной области \cite{lib:coupling:Valk}. Возможные взаимоотношения между подконструкциями различных типов представлены на рисунке~\ref{fig:schemeCoupling}.

Независимо от того, моделируются ли подконструкции во временной или частотной области, при стыковке подконструкций должны выполняться следующие условия:
\begin{enumerate}[noitemsep]
	\item Совместность перемещений стыковочных степеней свободы.
	\item Выполнение уравнений равновесия. 
\end{enumerate}

\begin{figure}[!htb]
	\centering
	\includegraphics[width = 0.8\linewidth]{schemeCoupling}
	\caption{Методы стыковки подконструкций во временной и частотной областях} \label{fig:schemeCoupling}
\end{figure}

Наиболее распространенными методами ассемблирования, которые удовлетворяют описанным условиям, на текущий момент являются:
\begin{itemize}
	\item \name{Primal assembly}: выбор уникального набора степеней свободы (DOFs) при котором обе подконструкции имеют одинаковый набор интерфейсных узлов, что приводит к автоматическому выполнению уравнений совместности перемещений и равновесия \cite{lib:coupling:Fregolent}.
	\item \name{Dual assembly}: выбор такой комбинации степеней свободы при которой уравнения равновесия  могут быть удовлетворены априори \cite{lib:coupling:DAmbrogio}.
\end{itemize}

Отметим, что стыковка во временной области предполагает собой ассемблирование матриц, описывающих свойства конструкций, по соответствующим степеням свободы. Выбранные степени свободы должны однозначно определять поведение результирующей конструкции (Maximum Rank Coordinate Choice) \cite{lib:coupling:Allen&Mayes}. Учет физических свойств закрепления конструкций может быть достигнут с использованием весовых функций соединения узлов (Modal constraints for fixture and subsystem) \cite{lib:coupling:Allen}. 

Ассемблирование с использованием частотных откликов конструкций (FRF) зачастую осуществляется следующим образом:
\begin{equation*}
	\mat{H}_{Cij} = \lfloor \mat{H}_\mat{A} \rfloor_{is} ([\mat{H}_\mat{A}]_{ss} + [\mat{H}_\mat{B}]_{ss})^{-1} \{ \mat{H}_\mat{B} \}_{sj},
\end{equation*}
где $\mat{A}, \mat{B}$~---~ассемблируемые конструкции; $\mat{C}$~---~результат композиции структур; $\mat{H}$~---~FRF матрица; $\lfloor \mat{H} \rfloor$~---~строка FRF матрицы; $\{ \mat{H} \}$~--~столбец FRF матрицы; $ i, j $~---~внеинтерфейсные DOFs; $ s $~---~интерфейсные DOFs.

Однако, если матрицы являются плохо обусловленными и/или содержат шум, могут возникнуть проблемы с вычислением обратной матрицы \cite{lib:coupling:Gialamas}. В этом случае возможно вычислением псевдообратной матрицы с использованием сингулярного разложения матриц (Singular Value Decomposition~--~SVD). В соответствии с этой теорией $ (n \times m) $ комплексная матрица $ [\mat{G}] $ может быть представлена в виде:
\begin{equation}
	[\mat{G}] = [\mat{U}][\mat{\Sigma}][\mat{V}] ^ \mat{H}
\end{equation}
где, \\
$ [\mat{U}] $~---~унитарная $(n \times n)$ матрица (т.е $ [\mat{U}][\mat{U}] ^ \mat{H} = [\mat{U}] ^ \mat{H}[\mat{U}] = [\mat{I}] $). Столбцы этой матрицы носят название левых сингулярных векторов, \\
$[\mat{\Sigma}]$~---~действительная $(n \times m)$ псевдодиагональная матрица. Её элементы носят название сингулярных значений, которые расположены в убывающем порядке,\\
$[\mat{V}]$~---~унитарная ${(n \times n)}$ матрица. Столбцы этой матрицы называют правыми сингулярными векторами. 

Результирующая псевдообратная матрица может быть найдена из следующего соотношения:
\begin{equation}
	[\mat{G}]^{+} = [\mat{V}]_r [\mat{\Sigma}]_r ^ {-1} [\mat{U}]_r ^ \mat{H}
\end{equation}
где индекс $ r $ означает срез по ненулевым сингулярным числам.

Амплитудно-частотные отклики конструкций также могут быть использованы для определения их динамических свойств \cite{lib:coupling:Xi, lib:coupling:Malekjafarian}.

\section{Методы операционного модального анализа}

Методы операционного модального анализа используются для оценки модальных характеристик конструкций. В ряде случаев схема многоточечного возбуждения не может быть реализована в силу значительных размеров конструкций и/или воздействие не может быть напрямую измерено. В этих случаях в качестве исходных данных используются только сигналы откликов. 

Принципиальные допущения, лежащие в основе методов операционного модального анализа:
\begin{enumerate}[noitemsep]
	\item Воздействие представляет собой стационарный белый шум.
	\item Система является стационарной.
	\item Система является наблюдаемой: расположение датчиков на конструкции позволяет определить все тона колебаний, лежащие в интересующем частотном диапазоне. 
\end{enumerate}

Методы операционного модального анализа делятся на две категории: частотные и временные методы \cite{lib:oma:Magalhaes}. При проведении анализа во временной области импульсные характеристики аппроксимируются как корреляционные функции. Основной недостаток таких методов~---~импульсная характеристика с отдельно взятого датчика является суперпозицией колебаний по всем тонам совместно. Методы анализа в частотной области основываются на предположении о том, что спектральное разложение процесса происходит по частотам собственных колебаний конструкций \cite{lib:oma:Brincker}. 

\subsection{Методы идентификации в частотной области}

Методы идентификации в частотной области \figref{fig:schemeFrequencyDomainOMA} делятся на непараметрические и параметрические. Последние используют оптимизационный алгоритм для минимизации невязки между параметрической моделью и измеренными откликами. 

\begin{figure}[H]
	\centering
	% Стили
	\tikzstyle{baseBlock}=[rectangle, draw = black, rounded corners, text width = 10em, text centered, minimum height = 1.5em, drop shadow]
	\tikzstyle{edge from parent}=[draw, very thick, color=black!90, -latex']
	% Дерево
	\begin{tikzpicture}
	[
		scale = 1,
		edge from parent fork down,
		level 1/.style = {baseBlock, sibling distance = 20em},
		level 2/.style = {baseBlock, sibling distance = 12em},
	]
	    \node {\underline{Частотная область (FD)}}[sibling distance = 12em, level distance = 5em]
			child { node {Непараметрические}
				child { node {Frequency domain decomposition (FDD)} } }
			child { node {Параметрические}
				child { node {Maximum likelihood estimator (MLE)} }
				child { node [yshift = -0.7em] {Least squares complex frequency (LSCF) estimator}
				child { node [yshift = -2em] {Poly-reference least squares complex frequency estimator (p-LSCF)} } }
			};
	\end{tikzpicture}
	\caption{Методы идентификации в частотной области} \label{fig:schemeFrequencyDomainOMA}
\end{figure}

Наиболее широкое применение для анализа конструкций нашел параметрический метод \name{p-LSCF} \cite{lib:oma:Guillaume}. Недостатком этого метода является невозможность идентификации близких тонов колебаний. Описанного недостатка лишен метод \name{PolyMAX}, который представляет собой развитие \name{p-LSCF} и используется в коммерческих продуктах, распространяемых компанией \name{LMS}. Подробное описание метода приведено в \cite{lib:oma:Peeters&Auweraer}

\subsection{Методы идентификации во временной области}

Вследствие ограничения частотного разрешения, а также эффекта растекания спектра, нередко используют методы идентификации во временной области \figref{fig:schemeTimeDomainOMA}. Их дополнительным преимуществом является отсутствие необходимости определения передаточных функций системы.

\begin{figure}[H]
	\centering
	% Стили
	\tikzstyle{baseBlock}=[rectangle, draw = black, rounded corners, text width = 10em, text centered, minimum height = 1.5em, drop shadow]
	\tikzstyle{edge from parent}=[draw, very thick, color=black!90, -latex']
	% Дерево
	\begin{tikzpicture}
	[
		scale = 1,
		edge from parent fork down,
		style = {baseBlock}
	]
	    \node {\underline{Временная область (TD)}}[sibling distance = 10em, level distance = 7em]
	    	child { node {Eigensystem realization algorithm (ERA)} }
			child { node {Autoregressive moving-average (ARMA)} }
			child { node {Stochastic subspace identification (SSI) }
				child { node [yshift = -0.7em] {Covariance-driven stochastic subspace identification (SSI-COV)} } 
				child { node {Data-driven stochastic subspace identification (SSI-DD)} } }
			child { node {Least squares \\ complex estimator (LSCE)} };

	\end{tikzpicture}
	\caption{Методы идентификации во временной области}\label{fig:schemeTimeDomainOMA}
\end{figure}

\subsubsection{Модель авторегрессии~---~скользящего среднего (ARMA)}

Авторегрессионные модели являются стохастическими разностными моделями, которые описывают временные данные в параметрической форме \cite{lib:oma:Chen}. Векторная авторегрессионая модель порядка $ q $, обозначаемая как $ AR(q) $, для ускорений $ \ddot{u}(k) $ в дискретный момент времени $ k $:

\begin{equation}
	\ddot{u}(k) = \sum_{j\,=\,1} ^ q L_j^{(q)} \ddot{u}(k - j) + v(k). \label{eq:representationVAR}
\end{equation}

В выражении \eqref{eq:representationVAR} $ L_j^{(q)} $ является матрицей коэффициентов для лага $ j $ и $ v(k) $~---~вектора ошибок гауссовского шума. Параметры авторегрессионной модели находятся посредством минимизации невязок между временными данными и их параметрическим представлением. Для нахождения модальных параметров векторная авторегрессионная модель впоследствии преобразуется к модели в пространстве состояний:

\begin{equation}
	\ddot{U}_q(k) = R_q \ddot{U}_q(k - 1) + Y(k),
\end{equation} 
где $ \ddot{U}_q(k) = \begin{bmatrix} \ddot{u}(k) & \ddot{u}(k - 1) & \dots & \ddot{u}(k - q + 1) \end{bmatrix}, $

\begin{equation}
	R_q = 
	\begin{pmatrix}
		L_1^{(q)} & L_1^{(q)} & \dots & L_{q-1}^{(q)} & L_q^{(q)} \\
		I_p & 0 & \dots & 0 & 0 \\ 
		0 & I_p & \dots & 0 & 0 \\ 
		\vdots & \vdots & \vdots & \ddots & \vdots \\ 
		0 & 0 & \dots & I_p & 0 
	\end{pmatrix},
\end{equation}

\begin{equation}
	Y(k) = \begin{bmatrix} v(k)^T & 0_{1 \times p} & \dots & 0_{1 \times p} \end{bmatrix}.
\end{equation}

Матрица $ R_q $ описывает динамические свойства колеблющийся конструкции. Модальные параметры находятся посредством решения модальной декомпозиции:

\begin{equation}
	R_q = \psi_q \Lambda_q \psi_q^{-1}.
\end{equation}
где $ \Lambda_q $~---~это комплексная матрица диагональная матрица собственных значений $ \lambda_j $ и $ \psi_q $~---~это матрица, содержащая собственные вектора $ R_q $. Частота $ \omega_j $ и относительный коэффициент демпфирования $ \xi_j $ определяются исходя из собственного значения каждого тона колебаний:

\begin{gather}
	\omega_j = \frac{|\log(\lambda_j)|}{\Delta t}, \\
	\xi_j = \frac{Re(\log(\lambda_j))}{|\log(\lambda_j)|},
\end{gather}
где $ \Delta t $~---~шаг дискретизации по времени.

\subsubsection{Методы идентификации стохастической модели в пространстве состояний~(SSI)}

Методы идентификации стохастической модели в пространстве состояний широко используются для анализа динамических параметров механических систем. Существует две разновидности этого подхода: на основе ковариационной (корреляционной) матрицы \name{SSI-COV} и на основе построения проекций исходных сигналов \name{SSI-DD} \cite{lib:oma:Rainieri}. 

Как показано в работе \cite{lib:oma:Peeters}, эти подходы являются тесно связанными. Однако, метод \name{SSI-COV} основывается на более простых предположениях и позволяет быстрее обрабатывать данные откликов, тогда как \name{SSI-DD} позволяет проводить расширенную постобработку получаемых результатов, например, оценивать вклад каждого тона колебаний.

Стохастическая модель в пространстве состояний строится на основе уравнений движения системы в модальных координатах:

\begin{equation}
	\mat{M} \ddot{\mat{\eta}}(t) + \mat{C}_2 \dot{\mat{\eta}}(t) + \mat{K} \eta(t) = \mat{u}(t),
\end{equation}
где $ \mat{M} $, $ \mat{C}_2 $, $ \mat{K} $~---~матрицы масс, демпфирования и жесткости соответственно. Вектор $ \mat{u}(t) $ содержит неизвестные воздействия. Вектор отклика $ \mat{y}(t) $ запишется:
\begin{equation}
	\mat{y}(t) = \mat{L}_a \ddot{\mat{\eta}}(t) + \mat{L}_v \dot{\mat{\eta}}(t) + \mat{L}_d \eta(t) + \mat{v}(t),
\end{equation}
где $ \mat{L}_a $, $ \mat{L}_v $ и $ \mat{L}_d $~---~это матрицы, описывающие связь модальных функций с ускорениями, скоростями и перемещениями. Вектор $ \mat{v}(t) $ описывает шумовую составляющую системы. Таким образом, стохастическая модель в пространстве состояний запишется:
\begin{equation}
	\dot{\mat{s}}(t) = \mat{P}_c \mat{s}(t) + \mat{B}_c \mat{u}(t), \label{eq:stochasticModelSSI}
\end{equation}
где 
\begin{gather}
	\mat{s}(t) = 
	\begin{bmatrix} 
		\dot{\mat{\eta}}(t) \\ 
		\mat{\eta}(t) 
	\end{bmatrix}, \\
	\mat{P}_c =
	\begin{pmatrix}
		-\mat{M} ^ {-1} \mat{C}_2 & -\mat{M} ^ {-1} \mat{K} \\
		 \mat{I} & 0
	\end{pmatrix}, \\
	\mat{B}_c = 
	\begin{bmatrix}
		\mat{M} ^ {-1} \\
		0	
	\end{bmatrix}.
\end{gather}

Система может быть записана в дискретном виде $ t = k \Delta t $ на основании того, что шум в отклике $ \mat{v}(t) $ и возбуждении $ \mat{u}(t) $ является белым шумом:
\begin{equation}
	\begin{cases}
		\mat{s}(k + 1) = \mat{P} \mat{s}(k) + \mat{w}_1(k), \\
		\mat{y}(k) = \mat{C} \mat{s}(k) + \mat{w}_2(k),
	\end{cases}
\end{equation}
где $ \mat{w}_1(k) $ и $ \mat{w}_2(k) $~---~это независимые шумовые слагаемые. $ \mat{P} $ и $ \mat{C} $ задаются как:
\begin{gather}
	\mat{P} = e ^ {\mat{P}_c \Delta t}, \\
	\mat{C} = 
	\begin{bmatrix} 
		\mat{L}_v & -\mat{L}_a \mat{M} ^ {-1} \mat{C}_2 & \mat{L}_d - \mat{L}_a \mat{M} ^ {-1} \mat{K}
	\end{bmatrix}.
\end{gather}

Целью модальной идентификации является нахождение собственных значений $ \chi_n $ и собственных векторов $ \mat{\phi}_n $ через которые определяются собственные частоты $ f_n $, формы колебаний $ \mat{a}_n $ и относительные коэффициенты демпфирования:
\begin{equation}
	\mu_n = \frac{\ln(\chi_n)}{\Delta t}, \ f_n = \frac{|\mu_n|}{2 \pi}, \ \zeta = \frac{-Re(\mu_n)}{|\mu_n|}, \ a_n = \mat{C} \mat{\phi}_n.
\end{equation}

\subsubsection{Алгоритм построения реализации по собственным колебаниям в пространстве состояний~(ERA)}

По аналогии с \eqref{eq:stochasticModelSSI} стохастическая модель в пространстве состояний запишется \cite{lib:oma:Juang}:
\begin{gather}
	\mat{x}_{k + 1} = \mat{A} \mat{x}_k + \mat{M} \mat{y}_k, \\
	\mat{y}_k = \mat{C} \mat{x}_k + \mat{v}_k.
\end{gather}

Для получения импульсных характеристик по откликам на случайное воздействие, к временным сигналам применяется метод \name{Natural Excitation Technique} (NExT) \cite{lib:oma:Lin}. 

Расчет начинается с построения блочной матрицы обобщенной ганкелевой матрицы $ (r \times s) $:
\begin{equation}
	\mat{H}_{rs}(k - 1) = 
	\begin{pmatrix}
		Y(k) & Y(k + t_1) & \dots & Y(k + t_{s - 1}) \\
		Y(j_1 + k) & Y(j_1 + k + t_1) & \dots & Y(j_1 + k + t_{s - 1}) \\
		\vdots & \vdots & & \vdots \\
		Y(j_{r - 1} + k) & Y(j_{r - 1} + k + t_1) & \dots & Y(j_{r - 1} + k + t_{s - 1})
	\end{pmatrix}. \label{eq:matrixHankel}
\end{equation}
где $j_i(i = 1, \hdots, r-1) $ и $ t_i(i = 1, \hdots, s - 1) $ являются случайными целыми числами.

Для системы $ n $-го порядка, используя сингулярное разложение матрицы \eqref{eq:matrixHankel}, найдем несингулярные матрицы $ \mat{P} $ и $ \mat{Q} $ такие, что:
\begin{equation}
	\mat{H}_{rs}(0) = \mat{P} \mat{D} \trans{\mat{Q}}.
\end{equation}

Тогда собственные значения $ \mat{z} $ и собственные вектора $ \mat{\psi} $ находятся в результате решения обобщенной проблемы собственных значений:
\begin{equation}
	\mat{\psi} ^ {-1} \left[ \mat{D} ^ {-\frac{1}{2}} \trans{\mat{P}} \mat{H}_{rs}(k) \mat{Q} \mat{D} ^ {-\frac{1}{2}} \right] \psi = \mat{z},
\end{equation}
где $ \mat{D} $~---~диагональная матрица, состоящая из ненулевых сингулярных чисел ганкелевой матрицы.

\section{Методы экспериментального модального анализа}

\fixme{Описать кратко основные}

\section{Выводы по главе \thechapter}