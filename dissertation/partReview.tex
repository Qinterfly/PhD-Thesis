\chapter{Методы верификации расчетных динамических моделей конструкций по результатам модальных испытаний}

\section{Методы и средства модальных испытаний}

\fixme{Сначала методы и средства классического экспериментального модального анализа. Затем переход к обнаружению дефектов в модальных испытаниях}

\section{Методы верификации расчетных динамических моделей} 

Конечно-элементные модели широко используются для проведения статических и динамических расчетов во многих областях техники. Однако такие модели в ряде случаев содержат неизбежные погрешности моделирования, обусловленные дискретизаций модели, неточностью задания свойств материалов, геометрических характеристик и граничных условий \cite{lib:modelUpdating:Bartilson}. Для устранения погрешностей моделирования были разработаны различные методы коррекции КЭ-моделей применительно к разным конструкциям: топливным бакам ракет-носителей \cite{lib:modelUpdating:Li&Tian}, соплу ракетного двигателя \cite{lib:modelUpdating:Yan&Li}, композитному самолету \cite{lib:modelUpdating:Zhao&Gupta}, архитектурным сооружениям \cite{lib:modelUpdating:Girardi&Padovani}, пластинчатым теплообменникам \cite{lib:modelUpdating:Guo&Wang}. В основе этих методов лежит минимизация разницы между ключевыми характеристиками реальной конструкции и параметрами КЭ-модели. В качестве таких параметров могут выступать, например, частоты собственных колебаний и отклик конструкции на динамическое воздействие \cite{lib:modelUpdating:Petersen&Oiseth}. Кроме того, методы коррекции нередко используются для оценки поврежденностей различных конструкций: дамб \cite{lib:modelUpdating:Bayraktar&Sevim}, мостов \cite{lib:modelUpdating:Cong&Thoi, lib:modelUpdating:Polanco}, контейнеров для хранения радиоактивных отходов \cite{lib:modelUpdating:Eiras}. Обзор подходов к решению проблемы мониторинга поврежденности приведен в \cite{lib:modelUpdating:Simoen}. 

Следует отметить, что погрешность описания механических свойств реальной конструкции посредством метода конечных элементов зачастую обусловлена особенностями моделирования узлов сопряжения, например, болтовых и заклепочных соединений. Однако описание нелинейного поведения таких узлов посредством построения нелинейной КЭ-модели в ряде случаев может оказаться вычислительно затратным. Во избежание этого такие узлы сопряжения могут быть заменены специальными элементами с эквивалентными жесткостными и диссипативными характеристиками. Так, в работах \cite{lib:modelUpdating:Lacayo, lib:modelUpdating:Yuan} болтовые узлы сопряжения двухбалочной модели были заменены гистерезисными элементами Iwan’a, параметры которых были определены методом коррекции. 

Известные методы коррекции могут быть разделены на две категории: стохастические и детерминированные. В основе стохастических методов лежит представление о том, что экспериментальные данные являются случайными и содержат неизбежные ошибки измерения, обусловленные как объективными факторами (температура, влажность, вибрации, оборудование), так и субъективными (опыт проведения экспериментов) \cite{lib:modelUpdating:Huang}. В зависимости от типов ошибок измерения были разработаны различные методы коррекции: метод Монте-Карло симуляции \cite{lib:modelUpdating:Bao, lib:modelUpdating:Boulkaibet}, методы пертурбаций \cite{lib:modelUpdating:Wang&He, lib:modelUpdating:Deng}, методы случайных поверхностей отклика \cite{lib:modelUpdating:Zhai&Fei, lib:modelUpdating:Fang} и Байесовcкие методы \cite{lib:modelUpdating:Xin&Hao, lib:modelUpdating:Lam}. Однако, стохастические методы коррекции на несколько порядков вычислительно затратнее детерминированных методов. Заметим, что вопрос выбора параметров коррекции для стохастических методов не является однозначным \cite{lib:modelUpdating:Yuan&Liang}.

Детерминированные методы коррекции обычно сводятся к итерационной процедуре коррекции параметров КЭ-модели. Среди итерационных методов наиболее популярным и интуитивно понятным является подход, основанный на создании матрицы чувствительности \cite{lib:modelUpdating:Mottershead, lib:modelUpdating:Bakir, lib:modelUpdating:Min, lib:modelUpdating:Hernandez}. В основе этого метода лежит проблема минимизации целевой функции, равной сумме квадратов разностей между измеренными в эксперименте данными и соответствующими данными, полученными с помощью модели. Удобство такого подхода заключается в том, что в целевой функции одновременно могут фигурировать разнородные параметры, такие, как собственные частоты и отклики конструкций на динамическое воздействие. Чтобы уравновесить вклад этих данных в результирующую функцию авторы работы \cite{lib:modelUpdating:Chen&Guo} вводят весовые коэффициенты, значения которых определят методом анализа иерархий.
 
Нередко применение методов коррекции приводит к тому, что результирующая система уравнений является плохо обусловленной \cite{lib:modelUpdating:Hua}. Для борьбы с этой проблемой две техники регуляризации наиболее часто используются исследователями: регуляризация Тихонова \cite{lib:modelUpdating:Li&Law, lib:modelUpdating:Reumers} и сингулярное разложение \cite{lib:modelUpdating:Mantilla-Gaviria}. Особенности применения обоих подходов на примере модели бетонного сооружения показаны в работе \cite{lib:modelUpdating:Weber}.

\section{Выводы по главе \thechapter}