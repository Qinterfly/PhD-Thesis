\chapter{Результаты модальных испытаний как исходные данные для верификации расчетных моделей конструкций}

\section{Методика определения модальных параметров по результатам экспериментального модального анализа}

\fixme{Здесь нужно использовать описание методики, изложенное в изобретении}

\section{Погрешности экспериментального модального анализа}

\fixme{Знание погрешностей необходимо для оценки <<глубины>> коррекции}
 
\section{Первичная обработка результатов испытаний}

\fixme{Эта обработка необходима в том числе и для устранения некоторых погрешностей}

\section{Диагностика дефектов конструкций по результатам испытаний}

\fixme{Дефектов нет в расчетных моделях, поэтому их нужно обнаруживать в реальной конструкции, а затем либо устранять, либо учитывать}

\subsection{Использование нелинейных искажений портретов колебаний}

\fixme{Привести таблицу сравнений чувствительности портретов и собственных частот к дефектам на примере панели}

\subsection{Обнаружение трещин}

\subsection{Контроль люфтов и зазоров}

\subsection{Выявление повышенного трения в подвижных соединениях}

\section{Обработка и представление результатов в процессе испытаний}

\fixme{Вот здесь и пригодилась твоя программа экспресс обработки результатов испытаний, так как между испытаниями и первым вылетом нет времени для составления полновесного отчета. Но главное не это! Обработка и представление результатов испытаний непосредственно в процессе испытаний позволит оперативно составить заключение о полноте экспериментальных данных, необходимых для коррекции расчетной модели объекта испытаний.}

\section{Операционный модальный анализ}

\subsection{Методика декомпозиции виброускорений}

\fixme{Приводится изложение метода для постоянных амплитуд. Оценивается чувствительность в зависимости от зашумленности сигнала}

\subsection{Тестовые примеры}

\subsubsection{Упруго-массовая система}

Проведем первичное тестирование методов операционного модального анализа на примере упруго-массовой системы, обладающей шестью степенями свободы \figref{fig:elastic-system-scheme} \cite{lib:oma:NAFID}.

\begin{figure}[H]
	\centerfloat
	\includegraphics[width = 0.7\linewidth]{elastic-system-scheme}
	\caption{Механическая система с шестью степенями свободы} \label{fig:elastic-system-scheme}
\end{figure}

Матрицы жесткости, масс и демпфирования в этом случае равны:
\begin{gather}
	\mat{K} = 
	\begin{pmatrix}
		k_1 + k_2 & -k_2 & 0 & 0 & 0 & 0 \\
		-k_2 & k_2 + k_3 + k_8 + k_9 & -k_3 & -k_9 & 0 & 0 \\
		0 & -k_3 & k_3 + k_4 & -k_4 & 0 & 0 \\
		0 & -k_9 & -k_4 & k_4 + k_5 + k_9 & -k_5 & 0 \\
		0 & 0 & 0 & -k_5 & k_5 + k_6 & 0 \\
		0 & 0 & 0 & 0 & 0 & k_6 + k_7 \\
	\end{pmatrix}, \\
	\mat{M} = 
	\begin{pmatrix}
		m_1 & 0 & 0 & 0 & 0 & 0 \\
		0 & m_2 & 0 & 0 & 0 & 0 \\
		0 & 0 & m_3 & 0 & 0 & 0 \\
		0 & 0 & 0 & m_4 & 0 & 0 \\
		0 & 0 & 0 & 0 & m_5 & 0 \\
		0 & 0 & 0 & 0 & 0 & m_6 \\	
	\end{pmatrix}, \
	\mat{C} = 
	\begin{pmatrix}
		c_1 & 0 & 0 & 0 & 0 & 0 \\
		0 & c_3 & 0 & 0 & 0 & -c_3 \\
		0 & 0 & c_2 & 0 & -c_2 & 0 \\
		0 & 0 & 0 & 0 & 0 & 0 \\
		0 & 0 & 0 & -c_2 & c_2 & 0 \\
		0 & 0 & -c_3 & 0 & 0 & c_3 \\	
	\end{pmatrix}.
\end{gather}

Примем физические параметры системы равными: 
\begin{itemize}[noitemsep]
	\item $ m_1 = m_2 = m_5 = 2 $ кг,
	\item $ m_3 = m_4 = m_6 = 1 $ кг,
	\item $ k_5 = k_8 = k_9 = 2.01 \ \sfrac{\text{кН}}{\text{м}} $,
	\item $ k_1 = k_2 = k_3 = k_4 = k_5 = k_6 = k_7 = 1 \ \sfrac{\text{кН}}{\text{м}} $,
	\item $ c_1 = c_3 = 400 \ \sfrac{(\text{Н} \cdot \text{с})}{\text{м}} $,
	\item $ c_2 = 200 \ \sfrac{(\text{Н} \cdot \text{с})}{\text{м}} $.
\end{itemize}

Будем моделировать динамический отклик \figref{elastic-system-signals} невозмущенной механической системы ($ \eta = 0 $) на случайные внешние воздействия ($ f_a = 0.5 $ Н), приложенные к каждой из масс. Шаг дискретизации составляет $ 1 $ мс, а общее время моделирования~---~$ 1000 $ c.

Спектральная плотность мощности, построенная на основе откликов каждой массы, приведена на рисунке~\ref{elastic-system-spectrums}.

\begin{figure}[!htb]
	\centerfloat
	\includegraphics[width = 1\linewidth]{elastic-system-signals}
	\caption{Временные сигналы ускорений каждой из масс} \label{elastic-system-signals}
\end{figure}

\begin{figure}[!htb]
	\centerfloat
	\includegraphics[width = 1\linewidth]{elastic-system-spectrums}
	\caption{Спектральная плотность мощности по всем каналам измерений} \label{elastic-system-spectrums}
\end{figure}

Теоретические и определенные частоты собственных колебаний и относительные коэффициенты демпфирования сведены в таблицах~\ref{tab:elastic-system-frequencies} и \ref{tab:elastic-damping-ratios}.

\begin{longtblr}[
	caption = {Результаты определения частот собственных колебаний механической системы с шестью степенями свободы}, 
	label = {tab:elastic-system-frequencies}, 
]{
	colspec = {|c|c|c|c|c|c|},
	hlines
}
	\SetCell[r = 2]{c} Тон & \SetCell[c = 5]{c} Частота, Гц &&&& \\
	& Теория & SSI-COV & SSI-DD & LSCE & ERA \\ \hline
	1 & 93.1856 & 93.1873 & 93.1851 & 93.1979 & 93.1737 \\
	2 & 149.6246 & 149.5825 & 149.6334 & 149.5321 & 149.2415 \\
	3 & 187.9729 & 188.1742 & 188.1098 & 187.5424 & 187.2981 \\
	4 & 246.0442 & 246.0608 & 246.1392 & --- & --- \\
	5 & 289.0728 & 289.1116 & 289.0207 & 289.2388 & 288.8544 \\
	6 & 395.9591 & 395.9343 & 395.9575 & 395.9110 & 395.9927 \\
\end{longtblr}

\begin{longtblr}[
	caption = {Результаты определения относительных коэффициентов демпфирования механической системы с шестью степенями свободы}, 
	label = {tab:elastic-damping-ratios}, 
]{
	colspec = {|c|c|c|c|c|c|},
	hlines
}
	\SetCell[r = 2]{c} Тон & \SetCell[c = 5]{c} Коэффициент демпфирования, \% &&&& \\
	& Теория & SSI-COV & SSI-DD & LSCE & ERA \\ \hline
	1 & 1.3844 & 1.3768 & 1.3860 & 1.3784 & 1.4192 \\
	2 & 12.4453 & 12.3011 & 12.3983 & 12.4082 & 12.1034 \\
	3 & 14.7696 & 14.7151 & 14.8216 & 14.1395 & 14.0355 \\
	4 & 10.9748 & 10.9288 & 10.9691 & --- & --- \\
	5 & 3.4296 & 3.4147 & 3.4093 & 3.4559 & 3.4397 \\
	6 & 0.7436 & 0.7521 & 0.7483 & 0.7531 & 0.7493 \\
\end{longtblr}

\section{Имитационная модель беспилотного летательного аппарата}

Рассмотрим имитационную модель беспилотного летательного аппарата \name{XQ-58 Valkyrie}~\figref{fig:x58-geometry} \cite{lib:misc:x58}. На основании геометрической модели планера была создана конечно-элементная модель \name{Ansys} \figref{x58-mesh}. При этом жесткостные характеристики элементов планера подбирались таким образом, чтобы приблизить спектр частот собственных колебаний к тому, который наблюдается на летательных аппаратах схожей компоновки.

\begin{figure}[!htb]
	\centerfloat
	\includegraphics[width = 0.75\linewidth]{x58-geometry}
	\caption{Геометрическая модель беспилотного летательного аппарата} \label{fig:x58-geometry}
\end{figure}

На каждом из элементов планера была размещена сеть виртуальных датчиков, связанных между собой треугольными полигонами. Пространственная схема расположения датчиков, общее количество которых составило $ 101 $, приведена на рисунке \ref{fig:x58-sensors}.

Для определения модальных характеристик планера к законцовкам крыла было приложено однократное импульсное воздействие длительностью $ 0.04 $ с. Данные динамических откликов записывались с частотой дискретизации $ 2 $ кГц по каждому пространственному направлению во всех датчиках в течение $ 5 $ с. Временные зависимости ускорений вдоль направления Y, полученные в нескольких точках левой консоли крыла, приведены на рисунке~\ref{fig:x58-signals}. Распределение спектральной плотности мощности, соответствующей этим сигналам, показано на рисунке~\ref{fig:x58-spectrums}.

\begin{figure}[!htb]
	\centerfloat
	\includegraphics[width = 0.75\linewidth]{x58-mesh}
	\caption{КЭ-модель беспилотного летательного аппарата} \label{fig:x58-mesh}
\end{figure}

\begin{figure}[!htb]
	\centerfloat
	\includegraphics[width = 0.75\linewidth]{x58-sensors}
	\caption{Схема размещения датчиков} \label{fig:x58-sensors}
\end{figure}

\begin{figure}[!htb]
	\centerfloat
	\includegraphics[width = 1\linewidth]{x58-signals}
	\caption{Временные сигналы ускорений в нескольких точках левой консоли крыла} \label{fig:x58-signals}
\end{figure}

\begin{figure}[!htb]
	\centerfloat
	\includegraphics[width = 1\linewidth]{x58-spectrums}
	\caption{Спектральная плотность мощности в нескольких точках левой консоли крыла} \label{fig:x58-spectrums}
\end{figure}

Необходимо заметить, что не все методы операционного модального анализа, которые рассматриваются в настоящей работе, достаточно эффективны с численной точки зрения для одновременной обработки всего массива данных откликов. Так, посредством метода \name{ERA} удается обработать лишь один элемент планера одномоментно. Наиболее высокопроизводительным и точным применительно к рассматриваемой задачи оказался метод \name{SSI-COV}. В таблице~\ref{tab:x58-ssi-cov-results} сведены частоты $ f $ и логарифмические декременты колебаний $ \delta $, определенные по конечно-элементной модели и методом \name{SSI-COV}. По этим данным вычислены погрешности определения модальных характеристик, которые приведены в двух последних столбцах. При этом определенные формы колебаний \name{SSI-COV} совпадают с расчетными \name{Ansys}. Сравнение форм колебаний на примере пятого тона собственных колебаний приведено на рисунке~\ref{fig:x58-mode-compare}. 

\begin{longtblr}[
	caption = {Результат определения модальных характеристик методом SSI-COV}, 
	label = {tab:x58-ssi-cov-results}, 
]{
	colspec = {|c|c|c|c|c|c|c|},
	hlines
}
	\SetCell[r=2]{c} Тон & \SetCell[c=2]{c} Ansys && \SetCell[c=2]{c} SSI-COV && \SetCell[c=2]{c} Погрешность, \% & \\
	& $ f $, Гц & $ \delta $ & $ f $, Гц & $ \delta $ & $ \Delta \overline{f} $ & $ \Delta \overline{\delta} $ \\ \hline
	1 & 1.8343 & \SetCell[r=11]{c} 0.0628 & 1.8343 & 0.0628 & -0.0018 & 0.0018 \\ 
	2 & 7.0324 & & 7.0322 & 0.0628 & -0.0035 & 0.0050 \\ 
	3 & 10.2304 & & 10.2300 & 0.0628 & -0.0042 & -0.0141 \\ 
	4 & 21.7268 & & 21.7180 & 0.0628 & -0.0405 & -0.0730 \\ 
	5 & 25.9578 & & 25.9430 & 0.0628 & -0.0571 & -0.1016 \\ 
	6 & 30.8802 & & 30.8560 & 0.0627 & -0.0784 & -0.1732 \\ 
	7 & 35.5900 & & 35.5530 & 0.0627 & -0.1039 & -0.2035 \\ 
	8 & 38.4500 & & 38.4030 & 0.0627 & -0.1223 & -0.2369 \\ 
	9 & 40.3791 & & 40.3250 & 0.0626 & -0.1340 & -0.2942 \\ 
	10 & 40.4966 & & 40.4420 & 0.0627 & -0.1348 & -0.2608 \\ 
	11 & 52.6507 & & 52.5310 & 0.0626 & -0.2274 & -0.4486 \\ 
\end{longtblr}

\begin{figure}[!htb]
	\centering
	\begin{subfigure}{0.49\textwidth}
		\includegraphics[width = 1\textwidth]{x58-ansys-mode-5}
		\caption{\name{Ansys}}
	\end{subfigure}
	\begin{subfigure}{0.49\textwidth}
		\includegraphics[width = 1\textwidth]{x58-ssi-cov-mode-5}
		\caption{\name{SSI-COV}}
	\end{subfigure}
     \caption{Сопоставление расчетных и определенных форм собственных колебаний тона №5} \label{fig:x58-mode-compare}
\end{figure}

Остальные формы колебаний, определенные методом \name{SSI-COV}, приведены на рисунках~\ref{subfig:x58-ssi-cov-mode-1}~--~\ref{subfig:x58-ssi-cov-mode-4}.

\def\sfX58{0.48\textwidth}

\begin{figure}[!htb]
	\centering
	\begin{subfigure}[b]{\sfX58}
		\includegraphics[width = \textwidth]{x58-ssi-cov-mode-1}
		\caption{1.83 Гц (траекторный)} \label{subfig:x58-ssi-cov-mode-1}
	\end{subfigure}
	\hfill
	\begin{subfigure}[b]{\sfX58}
		\includegraphics[width = \textwidth]{x58-ssi-cov-mode-2}
		\caption{7.03 Гц}
	\end{subfigure}
	\begin{subfigure}[b]{\sfX58}
		\includegraphics[width = \textwidth]{x58-ssi-cov-mode-3}
		\caption{10.23 Гц}
	\end{subfigure}	
	\hfill
	\begin{subfigure}[b]{\sfX58}
		\includegraphics[width = \textwidth]{x58-ssi-cov-mode-4}
		\caption{21.71 Гц} \label{subfig:x58-ssi-cov-mode-4}
	\end{subfigure}	
	\caption{Пример форм колебаний имитационной модели по методу \name{SSI-COV} (a~--~г)}
\end{figure}

Для количественной оценки соответствия определенных форм колебаний их расчетным аналогам, воспользуемся критерием модального соответствия. Численная оценка качества выделения форм колебаний методом \name{SSI-COV} для первых двенадцати тонов колебаний приведена на рисунке~\ref{fig:x58-mac}. Из рисунка видно, что определенные формы колебаний практически совпадают с расчетными.

\begin{figure}[!htb]
	\centerfloat
	\includegraphics[width = 1\linewidth]{x58-mac}
	\caption{Критерий модального соответствия расчетных форм колебаний их определенным аналогам} \label{fig:x58-mac}
\end{figure}

Стабилизационная диаграмма, построенная по методу \name{SSI-COV}, показана на рисунке~\ref{fig:x58-ssi-cov}.

\begin{figure}[H]
	\centerfloat
	\includegraphics[width = 1\linewidth]{x58-ssi-cov}
	\caption{Стабилизационная диаграмма по методу \name{SSI-COV}} \label{fig:x58-ssi-cov}
\end{figure}

\subsection{Определение модальных характеристик по результатам акустических испытаний}

Определим модальные характеристики рефлектора \figref{fig:reflector-experiment} по результатам отклика на шумовое акустическое воздействие. Для записи откликов использовались акселерометры~\figref{fig:reflector-sensors}, размещенные на поверхности рефлектора в соответствии со схемой, приведенной на рисунке~\ref{fig:reflector-scheme}.

\begin{figure}[!htb]
	\centerfloat
	\includegraphics[width = 1\linewidth]{reflector-experiment}
	\caption{Рефлектор в сборе с испытательной оснасткой} \label{fig:reflector-experiment}
\end{figure}

\begin{figure}[!htb]
	\centerfloat
	\includegraphics[width = 0.9\linewidth]{reflector-sensors}
	\caption{Датчики ускорения, размещенные на поверхности рефлектора} \label{fig:reflector-sensors}
\end{figure}

\begin{figure}[!htb]
	\centerfloat
	\includegraphics[width = 0.9\linewidth]{reflector-scheme}
	\caption{Схема расстановки датчиков по рефлектору} \label{fig:reflector-scheme}
\end{figure}

Длительность шумового воздействия составила $ 53 $ секунды при частоте дискретизации сигналов равной $ 12800 $ Гц. Пример временных сигналов акселерометров вдоль направления Y показан на рисунке~\ref{fig:reflector-signals}. Спектральная плотность мощности исследуемых сигналов приведена на рисунке~\ref{fig:reflector-spectrums}. Обработка сигналов осуществлялась последовательно вдоль каждого из пространственных направлений тремя методами операционного модального анализа: \name{SSI-COV}, \name{ERA} и \name{SSI-DD}. При этом сигнал шумового воздействия не использовался. Частоты и логарифмические декременты колебаний, полученные посредством каждого из методов, сведены в таблице~\ref{tab:reflector-results}.

Формы колебаний, определенные методами операционного модального анализа, приведены на рисунках~\ref{subfig:reflector-ssi-cov-mode-2}~--~\ref{subfig:reflector-ssi-cov-mode-16}.

\begin{figure}[!htb]
	\centerfloat
	\includegraphics[width = 0.84\linewidth]{reflector-signals}
	\caption{Временные сигналы акселерометров вдоль направления Y} \label{fig:reflector-signals}
\end{figure}

\begin{figure}[!htb]
	\centerfloat
	\includegraphics[width = 0.84\linewidth]{reflector-spectrums}
	\caption{Спектральная плотность мощности временных сигналов акселерометров вдоль направления Y} \label{fig:reflector-spectrums}
\end{figure}

\begin{longtblr}[
	caption = {Результаты определения частот и логарифмических декрементов колебаний методами операционного модального анализа}, 
	label = {tab:reflector-results}, 
]{
	colspec = {|c|c|c|c||c|c|c|},
	hlines
}
	\SetCell[r = 2]{c} Тон & \SetCell[c = 3]{c} Частота, Гц && & \SetCell[c = 3]{c} Логарифмический декремент && \\
	& SSI-COV & ERA & SSI-DD & SSI-COV & ERA & SSI-DD \\ \hline
	1 & 66.052 & 65.962 & 66.403 & 0.054 & 0.042 & 0.045 \\
	2 & 91.287 & 90.910 & 91.843 & 0.030 & 0.052 & 0.042 \\
	3 & 102.540 & --- & --- & 0.056 & --- & --- \\
	4 & 114.770 & --- & --- & 0.063 & --- & --- \\
	5 & 122.370 & --- & --- & 0.094 & --- & --- \\
	6 & 127.850 & --- & --- & 0.049 & --- & --- \\
	7 & 157.560 & --- & --- & 0.121 & --- & --- \\
	8 & 203.850 & --- & --- & 0.060 & --- & --- \\
	9 & 208.270 & --- & --- & 0.062 & --- & --- \\
	10 & 227.700 & --- & --- & 0.057 & --- & --- \\
	11 & 243.360 & --- & --- & 0.048 & --- & --- \\
	12 & 273.450 & 273.310 & --- & 0.033 & 0.026 & --- \\
	13 & 283.300 & --- & 281.990 & 0.040 & --- & 0.033 \\
	14 & 325.360 & --- & 324.020 & 0.034 & --- & 0.056 \\
\end{longtblr}

На основе таблицы~\ref{tab:reflector-results} можем заключить, что метод \name{SSI-COV} позволяет выделить наибольшее число тонов колебаний. Оценим сходимость модальных характеристик, определяемых этим методом: частот~\tabref{tab:reflector-conv-time-frequency} и логарифмических декрементов колебаний~\tabref{tab:reflector-conv-time-decrement}, варьируя длительность временных сигналов от $ 5 $ до $ 50 $ секунд. 

\def\sfReflector{0.48\textwidth}

\begin{figure}[H]
	\centering
	\begin{subfigure}[b]{\sfReflector}
		\includegraphics[width = \textwidth]{reflector-ssi-cov-mode-2}
		\caption{91.29 Гц} \label{subfig:reflector-ssi-cov-mode-2}
	\end{subfigure}
	\hfill
	\begin{subfigure}[b]{\sfReflector}
		\includegraphics[width = \textwidth]{reflector-ssi-cov-mode-6}
		\caption{127.85 Гц}
	\end{subfigure}
	\begin{subfigure}[b]{\sfReflector}
		\includegraphics[width = \textwidth]{reflector-ssi-cov-mode-13}
		\caption{283.30 Гц}
	\end{subfigure}	
	\hfill
	\begin{subfigure}[b]{\sfReflector}
		\includegraphics[width = \textwidth]{reflector-ssi-cov-mode-16}
		\caption{329.54 Гц} \label{subfig:reflector-ssi-cov-mode-16}
	\end{subfigure}	
	\caption{Пример форм колебаний рефлектора, определенных методом \name{SSI-COV} (a~--~г)}
\end{figure}

\begin{longtblr}[
	caption = {Cходимость частот собственных колебаний в зависимости от длины временного сегмента}, 
	label = {tab:reflector-conv-time-frequency}
]{
	colspec = {|c|c|c|c|c|c|c|},
	hlines
}
	\SetCell[r = 2]{c} Тон & \SetCell[c = 6]{c} Длительность сегмента, c &&&&& \\
	& 5 & 10 & 20 & 30 & 40 & 50 \\ \hline
	1 & 66.096 & 65.993 & 66.102 & 66.04 & 66.042 & 66.052 \\
	2 & 91.335 & 91.193 & 91.174 & 91.259 & --- & 91.287 \\
	3 & --- & --- & --- & 102.87 & 102.64 & 102.54 \\
	4 & 115.25 & --- & 114.79 & 114.97 & 114.21 & 114.77 \\
	5 & 124.26 & 119.06 & 122.3 & 122.09 & 122.24 & 122.37 \\
	6 & --- & --- & 130.13 & --- & 128.34 & 127.85 \\
\end{longtblr}

\begin{longtblr}[
	caption = {Cходимость логарифмического декремента колебаний в зависимости от длины временного сегмента}, 
	label = {tab:reflector-conv-time-decrement}
]{
	colspec = {|c|c|c|c|c|c|c|}, 
	hlines
}
	\SetCell[r = 2]{c} Тон & \SetCell[c = 6]{c} Длительность сегмента, c &&&&& \\
	& 5 & 10 & 20 & 30 & 40 & 50 \\ \hline
	1 & 0.060 & 0.041 & 0.052 & 0.052 & 0.051 & 0.054 \\
	2 & 0.027 & 0.027 & 0.025 & 0.027 & --- & 0.030 \\
	3 & --- & --- & --- & 0.060 & 0.057 & 0.056 \\
	4 & 0.081 & --- & 0.070 & 0.073 & 0.068 & 0.063 \\
	5 & 0.086 & 0.090 & 0.098 & 0.101 & 0.096 & 0.094 \\
	6 & --- & --- & 0.022 & --- & 0.043 & 0.049 \\
\end{longtblr}

Дополнительно оценим влияния частоты дискретизации на устойчивость численных значений модальных характеристик, определяемых методом \name{SSI-COV}. Для этого временные отклики прореживаются с интервалами, которые соответствует диапазону частот дискретизации от $ 12800 $ до $ 2560 $ Гц. Результаты по частотам и логарифмическим декрементам сведены в таблицах~\ref{tab:reflector-conv-sample-frequency} и \ref{tab:reflector-conv-sample-decrement} соответственно. 

\begin{longtblr}[
	caption = {Cходимость частот собственных колебаний в зависимости от частоты дискретизации}, 
	label = {tab:reflector-conv-sample-frequency}
]{
	colspec = {|c|c|c|c|c|c|}, 
	hlines
}
	\SetCell[r = 2]{c} Тон & \SetCell[c = 5]{c} Частота дискретизации, Гц &&&& \\
	& 12800 & 6400 & 4267 & 3200 & 2560 \\ \hline
	1 & 66.052 & 65.874 & 65.933 & 65.891 & 65.841 \\
	2 & 91.287 & 91.292 & 91.201 & 91.136 & 91.049 \\
	3 & 102.539 & 102.476 & 102.458 & 102.763 & 103.266 \\
	4 & 114.772 & 114.007 & 114.044 & 114.134 & 114.196 \\
	5 & 122.375 & 123.003 & 123.034 & 123.123 & 123.257 \\
	6 & 127.851 & 129.596 & 129.586 & 129.580 & 129.599 \\
\end{longtblr}

\begin{longtblr}[
	caption = {Cходимость логарифмического декремента колебаний в зависимости от частоты дискретизации}, 
	label = {tab:reflector-conv-sample-decrement}, 
]{
	colspec = {|c|c|c|c|c|c|}, 
	hlines
}
	\SetCell[r = 2]{c} Тон & \SetCell[c = 5]{c} Частота дискретизации, Гц &&&& \\
	& 12800 & 6400 & 4267 & 3200 & 2560 \\ \hline
	1 & 0.054 & 0.034 & 0.030 & 0.032 & 0.029 \\
	2 & 0.030 & 0.025 & 0.031 & 0.035 & 0.037 \\
	3 & 0.056 & 0.062 & 0.068 & 0.064 & 0.055 \\
	4 & 0.063 & 0.053 & 0.052 & 0.052 & 0.053 \\
	5 & 0.094 & 0.061 & 0.057 & 0.056 & 0.050 \\
	6 & 0.049 & 0.020 & 0.021 & 0.022 & 0.022 \\
\end{longtblr}

Можем видеть, что временного сегмента длительностью $ 20 $ секунд достаточно для стабилизации численных значений определяемых модальных характеристик.

В случае изменения частоты дискретизации частота собственных колебаний меняется незначительно, в то время как логарифмический декремент колебаний меняется в разы. 

\subsection{Оценка модальных параметров летательных аппаратов по результатам полетов в неспокойной атмосфере}

\fixme{Флаттерные кривые}

\section{Выводы по главе \thechapter}