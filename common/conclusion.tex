
\begin{enumerate}
	\item С применением методов оптимизации разработана методика коррекции, основанная на дополнении исходной расчетной модели корректирующими конечными элементами. Проработана возможность коррекции как упругих характеристик, так и параметров демпфирования. Для формирования начального приближения матрицы демпфирования используется гипотеза Е.\,С.~Сорокина. Разработанный подход использован для решения практических задач коррекции  расчетных моделей.
	\item Методом статистического моделирования исследована сходимость алгоритма коррекции по отношению к погрешностям в результатах модальных испытаний. На модельных задачах показано, что искажения форм колебаний по результатам коррекции оказываются на порядок ниже погрешностей в целевых значениях собственных частот.
	\item Развита методика синтеза глобальных конечно-элементных моделей конструкций на основе верифицированных по результатам модальных испытаний моделей составных частей. Обоснована программа проведения экспериментов, позволяющая получать наиболее полные характеристики составных частей конструкций, в том числе при наложении дополнительных связей. Разработано программное обеспечение, реализующие полный цикл операций для решения задачи синтеза.
	\item Создан комплекс программ, позволяющий проводить обработку и представление результатов модального анализа непосредственно в процессе испытаний. Используя разработанное программное обеспечение, продемонстрированы практические результаты обнаружения производственно-конструктивных дефектов в конструкциях ЛА по результатам модальных испытаний.
\end{enumerate}

