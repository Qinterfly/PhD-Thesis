
\chapter*{Основные выводы и заключение} 
\addcontentsline{toc}{chapter}{Основные выводы и заключение}

Перспективным направлением дальнейших исследований является развитие разработанных в настоящей диссертации вычислительных алгоритмов для улучшения сходимости в задачах коррекции. Это особенно актуально при коррекции расчетных моделей с высокой степенью детализации, которая выражается в эмпирической оценке: число степеней свободы модели больше или равно одному миллиону. В этом случае было замечено, что обобщенное отношение Рэлея отличается от значений собственных чисел, получаемых при решении обобщенной проблемы собственных значений. Это происходит вследствие численных округлений при нахождении собственных форм колебаний с нулевыми собственными значениями.

Прикладной аспект дальнейшей работы выражается в создании вспомогательных утилит, позволяющих учитывать информацию о корректирующих элементах в конечно-элементных пакетах, использованных для создания исходных моделей. Это обеспечит взаимодействие встроенными средствами с скорректированными расчетными моделями, в том числе их применение для решения задач о вынужденных колебаниях и нелинейной динамики.
