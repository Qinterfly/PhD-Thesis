% ---------------------------------------------------------------------- %
% --- Пакеты, которые являются общими для диссертации и автореферата --- %
% ---------------------------------------------------------------------- %

% --- Логические условия --- %

\newif\ifsynopsis     % Условие, проверяющее, что документ является авторефератом
\usepackage{etoolbox}

% --- Комментирование текста --- %

\usepackage{comment}    

% --- Поля и разметка страницы --- %

\usepackage{pdflscape} % Для включения альбомных страниц
\usepackage{geometry}  % Для последующего задания полей

% --- Математические пакеты --- %

\usepackage{amsthm, amsmath, amscd} % Математические дополнения от AMS
\usepackage{amsfonts, amssymb}      % Математические дополнения от AMS
\usepackage{mathtools}              % Добавляет окружение multlined
\usepackage{xfrac}                  % Красивые дроби
\usepackage{upgreek}                %  Русская традиция начертания греческих букв
\usepackage[
    locale = DE,
    list-separator       = {;\,},
    list-final-separator = {;\,},
    list-pair-separator  = {;\,},
    list-units           = single,
    range-units          = single,
    range-phrase         = {\text{\ensuremath{-}}},
    fraction-function    = \sfrac,
    separate-uncertainty,
    ]{siunitx}[=v2] % Размерности SI
\sisetup{inter-unit-product = \ensuremath{{}\cdot{}}}

% --- Кодировки и шрифты --- %

% Кириллица в нумерации subequations
\patchcmd{\subequations}{\def\theequation{\theparentequation\alph{equation}}}
{\def\theequation{\theparentequation\asbuk{equation}}}
{\typeout{subequations patched}}{\typeout{subequations not patched}}

% Установки для размера шрифта 14 pt
\newlength{\curtextsize}
\newlength{\bigtextsize}
\setlength{\bigtextsize}{13.9pt}
\makeatletter
\setlength{\curtextsize}{\f@size pt}
\makeatother

% Решение проблемы копирования текста в буфер кракозябрами
\ifnumequal{\value{usealtfont}}{0}{}{
    \input glyphtounicode.tex
    \input glyphtounicode-cmr.tex %from pdfx package
    \pdfgentounicode=1
}
    
% Улучшенный поиск русских слов в полученном pdf-файле
\usepackage{cmap}   
\ifnumequal{\value{usealtfont}}{2}{}{
    \defaulthyphenchar=127  % Если стоит до fontenc, то переносы
                            % не впишутся в выделяемый текст при
                            % копировании его в буфер обмена
}

% Дополнительные текстовые символы
\usepackage{textcomp}

% Языковые пакеты
\usepackage[T1, T2A]{fontenc}        % Поддержка русских букв
\usepackage[utf8]{inputenc}          % Кодировка utf8
\usepackage[english, russian]{babel} % Языки: русский, английский

% Подключение русифицированных шрифтов XCharter
\ifnumequal{\value{usealtfont}}{2}{
	\usepackage[scaled = 0.914]{XCharter} 
	\usepackage[charter, vvarbb, scaled = 1.048]{newtxmath}
	\setDisplayskipStretch{-0.078}
}{}

% --- Оформление абзацев --- %

\indentafterchapter % Красная строка после заголовков типа chapter
\usepackage{indentfirst}

% --- Цвета --- %

\usepackage[dvipsnames, table, hyperref]{xcolor}

% --- Таблицы --- %

\usepackage{threeparttable} % Автоматическая ширина подписи таблицы

% --- Общее форматирование --- %

\usepackage{soulutf8} % Поддержка переносоустойчивых подчёркиваний и зачёркиваний

% --- Оптимизация расстановки переносов и длины последней строки абзаца --- %

\usepackage[hyphenation, lastparline]{impnattypo}

% --- Работа со списками --- %

\usepackage{enumitem}

% --- Векторная графика --- %

\usepackage{tikz}

% --- Гиперссылки --- %

\usepackage{hyperref}

% --- Изображения --- %

\usepackage{graphicx}   % Подключаем пакет работы с графикой
\usepackage{caption}    % Подписи рисунков и таблиц
\usepackage{subcaption} % Подписи подрисунков и подтаблиц
\usepackage{pdfpages}   % Добавление внешних pdf файлов

% --- Счётчики --- %

\usepackage{aliascnt}                 % Псевдонимы счетчиков
\usepackage[figure,table]{totalcount} % Счётчик рисунков и таблиц
\usepackage{totcount}                 % Пакет создания счётчиков на основе последнего номера
                                      %  подсчитываемого элемента (может требовать дважды
                                      %  компилировать документ)
\usepackage{totpages}                 % Счётчик страниц, совместимый с hyperref (ссылается
                                      %  на номер последней страницы). Желательно ставить
                                      %  последним пакетом в преамбуле

% --- Управление групповыми ссылками --- %

\usepackage[russian]{cleveref}

% --- При работе с черновиком --- %

\ifnumequal{\value{draft}}{1}{
    \usepackage[firstpage]{draftwatermark}
    \SetWatermarkText{DRAFT}
    \SetWatermarkFontSize{14pt}
    \SetWatermarkScale{15}
    \SetWatermarkAngle{45}
}{}